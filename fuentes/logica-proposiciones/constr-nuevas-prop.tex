



Exponemos ahora la forma de crear proposiciones nuevas haciendo uso de
varias proposiciones y operadores lógicos. Hasta ahora, solo se han empleado
proposiciones simples en la definición de los operadores lógicos para poder
construir proposiciones compuestas. Los operadores descritos solo actúan
sobre una o dos proposiciones. Cuando se dispone de más de dos
proposiciones, hay que emplear paréntesis, corchetes o llaves para indicar
las proposiciones que son afectadas por cada conector.

Hay que tener cuidado con la notación del operador negación, $\neg$. La
sintaxis a este respecto, en la lógica simbólica, consiste en que este
afecta únicamente a la proposición que le sucede. Así pues, en la expresión
$\neg p$ la negación afecta únicamente a $p$. Pero, si se tiene, por
ejemplo, la proposición compuesta $\neg p \land q$, la proposición afectada
por la negación es $p$, que quizás no es lo que deseaba. Para negar la
proposición $p \land q$, en forma global, se hace uso de paréntesis. Así, se
tendría que escribir $\neg (p \land q)$. Ambas proposiciones no son
equivalentes, tal y como se muestra en la tabla \ref{tbl:prioridad-neg}.

\begin{table}[h]%
  \caption{Prioridad en notación de negación}
  \label{tbl:prioridad-neg}%
  \centering
  $$
    \begin{array}{|c|c||c|c|c|c|}
      \hline
      p & q & \neg p & p \land q & \neg p \land q & \neg (p \land q) \\
      \hline
      \hline
      0 & 0 & 1 & 0 & 0 & 1 \\
      0 & 1 & 1 & 0 & 1 & 1 \\
      1 & 0 & 0 & 0 & 0 & 1 \\
      1 & 1 & 0 & 1 & 0 & 0 \\
      \hline
    \end{array}
  $$
\end{table}

En notaciones alternativas del operador negación, como, por ejemplo,
$\overline{p}$, se evita este problema y se tiene, por tanto, una notación
que hace menor uso de paréntesis y, por lo tanto, resulta más cómoda. Así,
se tendría que $\neg (p \land q)$ se representaría por $\overline{p \land
q}$.

En lo que respecta a la sintaxis de los operadores conjunción y disyunción,
no hay regla de prioridad sobre estos, con lo que expresiones como

$p \land q \lor r$ no son gramaticales, ya que se podría interpretar de dos
formas distintas: como $(p \land q) \lor r$ o como $p \land (q \lor r)$. Se
rompe la ambigüedad haciendo uso de paréntesis, como en estas dos últimas
expresiones. Al contrario de lo que sucede en la aritmética con la suma y el
producto, no existen a este respecto reglas de prioridad entre estos dos
operadores.

A este respecto, la notación que se suele seguir en ciencias de la
computación o en el diseño lógico digital parece también, al igual que vimos
para el operador negación, más eficiente. En esta, se suele representar al
operador conjunción con el producto: signo `$\cdot$' o uso de la notación en
aposición. La disyunción se representa como una suma: `$+$'. Se sigue la
regla de otorgar prioridad al producto respecto a la suma, con lo que se
evitaría el uso de paréntesis en muchos casos.

Por otro lado, en general, el orden de escritura de las proposiciones es
relevante. Así, $p \rightarrow q$ y $q \rightarrow p$ son dos proposiciones
no equivalentes, tal y como se puede ver en la
tabla~\ref{tbl:condicional-no-conmut}, en la que se ``enfrentan'' sus tablas
de verdad.

\begin{table}[h]%
  \caption{El condicional no es conmutativo}
  \label{tbl:condicional-no-conmut}%
  \centering
  $$
    \begin{array}{|c|c||c|c|}
      \hline
      p & q & p \rightarrow q & q \rightarrow p \\
      \hline
      \hline
      0 & 0 & 1 & 1 \\
      0 & 1 & 1 & 0 \\
      1 & 0 & 0 & 1 \\
      1 & 1 & 1 & 1 \\
      \hline
    \end{array}
  $$
\end{table}

El valor de cualquier proposición simple, sea verdadera o falsa, se obtiene
directamente de su enunciado. A veces, no resulta evidente la determinación
del valor de una proposición compuesta, puesto que este valor depende de los
valores que tomen las proposiciones simples que la componen.

Hay dos operadores que se deben destacar. Por un lado, al $C_0$, que toma el
valor 0 para cualquier combinación de datos de entrada, se le conoce como
\emph{contradicción}. Lo representaremos simbólicamente como un cero en
negrita: `$\contrad$'. Por el otro, se tiene el operador $C_{15}$, que toma
siempre el valor 1. A este se le conoce como una \emph{tautología}. Lo
designaremos por un uno en negrita: `$\tautol$'.

Advierta que no es lo mismo $\tautol$ que 1. El primero será una proposición
lógica, mientras que, el segundo, un valor de verdad. TKTK

% TODO Cuando presento las variables lógicas, $p$, mencionar las constantes
% lógicas: 0 y 1.

La forma de determinar si dos proposiciones son equivalentes es hacer la
operación bicondicional sobre estas. Si el resultado da una tautilogía, se
afirmará que ambas proposiciones son equivalentes. En caso contrario, no lo
son.

La forma habitual de manipular expresiones lógicas para sistematizar el
razonamiento es la siguiente. Si dos proposiciones $p$ y $q$ son
equivalentes y $p$ forma parte de una tercera proposición $r$, entonces
puede sustituirse $p$ por $q$ en la expresión de $r$, pues la nueva
proposición obtenida es equivalente a $r$. Desde el punto de vista lógico,
$p$ y $q$ pueden sustituirse el uno al otro; por eso coloquialmente se
expresa diciendo que $p$ y $q$ son proposiciones iguales.




