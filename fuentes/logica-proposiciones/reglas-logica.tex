



Para evitar sobrecargar la notación simbólica con paréntesis, adoptaremos la
regla de ahorrarnos los paréntesis, corchetes o llaves al escribir la
equivalencia entre dos proposiciones, marcada con el símbolo
`$\Longleftrightarrow$'. Las expresiones situadas a la derecha e izquierda
del mismo constituyen las proposiciones equivalentes, aunque vayan sin
paréntesis. Por ejemplo, se escribe

$$ p \lor (q \land r) \iff (p \lor q) \land (p \lor r) $$

\noindent para indicar que las proposiciones $p \lor (q \land r)$ y \( (p
\lor q) \land (p \lor r) \) son equivalentes, en lugar de

$$ [p \lor (q \land r)] \iff [(p \lor q) \land (p \lor r)] $$

\noindent No obstante, a veces se usará una expresión de este último estilo.






\subsection{Reglas lógicas equivalentes con una proposición}

Con una proposición simple $p$ y el conector negación, se pueden escribir
aparentemente muchas proposiciones nuevas. Por ejemplo, $\neg p$, $\neg(\neg
p)$, $\neg(\neg(\neg p))$, etc., aunque en realidad no son necesarios los
paréntesis; se puede poner perfectamente $\neg p$, $\neg \neg p$, $\neg \neg
\neg p$, etc.

\begin{table}[h]%
  \caption{Múltiples negaciones}
  \label{tbl:multipl-neg}%
  \centering
  $$
    \begin{array}{|c||c|c|}
      \hline
      p & \neg p & \neg \neg p \\
      \hline
      \hline
      0 & 1 & 0 \\
      1 & 0 & 1 \\
      \hline
    \end{array}
  $$
\end{table}

En realidad, tal y como vamos a demostrar ahora, en toda esa lista de
proposiciones se tienen únicamente dos proposiciones: $p$ y $\neg p$. Las
demás serán equivalentes a una de estas. Puede verlo fácilmente para $\neg
\neg p$ en la tabla \ref{tbl:multipl-neg}. Por tanto, se da que

$$ p \iff \neg \neg p $$

Coloquialmente, esta regla (también calificada de ley) se expresa diciendo
que ``una doble negación afirma''. Podemos sustituir $\neg \neg p$ por $p$ o
viceversa allí donde aparezcan. Lo mismo ocurre con las proposiciones $\neg
p$ y $\neg \neg \neg p$, que también son equivalentes. En general, se emplea
la expresión más corta, aunque algunas veces puede interesar hacer uso de
una más extensa.

Con una única proposición $p$, solo hay cuatro posibles tablas de verdad,
luego solo se pueden expresar cuatro proposiciones esencialmente distintas,
es decir, que no sean equivalentes entre sí, como se aprecia en la
tabla~\ref{tbl:tablas-una-prop}.

\begin{table}[h]%
  \caption{Tablas para una sola proposición}
  \label{tbl:tablas-una-prop}%
  \centering
  $$
    \begin{array}{|c||c|c|c|c|}
      \hline
      p & 0 & p & \neg p & 1 \\
      \hline
      \hline
      0 & 0 & 0 & 1 & 1 \\
      1 & 0 & 1 & 0 & 1 \\
      \hline
    \end{array}
  $$
\end{table}






\subsubsection{Reglas de simplificación}

Con una única proposición $p$ y un conector distinto de \neg, se pueden
escribir muchas proposiciones nuevas, pero entre ellas se cumplen las
siguientes leyes de simplificación:

\begin{minipage}[t]{.45\textwidth}
  \centering
  \begin{align*}
    p \lor p &\iff p \\
    p \land p &\iff p \\
  \end{align*}
\end{minipage}
\begin{minipage}[t]{.45\textwidth}
  \centering
  \begin{align*}
    p \leftrightarrow p &\iff \tautol \\
    p \rightarrow p &\iff \tautol \\
  \end{align*}
\end{minipage}

\noindent Por tanto, en una expresión de una proposición lógica compuesta en
la que aparezca $p \lor p$, esta parte se puede sustituir por $p$.

Con una única proposición $p$ y varios conectores distintos, se pueden
escribir proposiciones nuevas, como, por ejemplo, $p \lor \neg p$, $p \land
\neg p$, \ldots





\subsubsection{Ley del tercio excluso}

La proposición $p \lor \neg p$ es una tautología:

$$ p \lor \neg p \iff \tautol $$

Esta ley se expresa coloquialmente diciendo que siempre se cumple una
proposición o su negación. Por ejemplo,

\begin{center}
  El número $\pi$ es racional o irracional.
\end{center}





\subsubsection{Ley de contradicción}

La proposición $p \land \neg p$ es una contradicción:

$$ p \land \neg p \iff \contrad $$

Esto se expresa coloquialmente diciendo que nunca se cumple una proposición
y su negación, por ejemplo,

\begin{center}
  El número 3 es primo y compuesto.
\end{center}

\noindent es una proposición falsa.






\subsection{Reglas lógicas equivalentes con dos proposiciones}

Con dos proposiciones $p$ y $q$ y cualquier conjunto de conectores, solo se
pueden construir 16 proposiciones esencialmente distintas. Esto se debe a
que solo hay 16 tablas de verdad distintas, tal y como mostramos en la
tabla~\ref{tbl:tablas-dos-prop}.

\begin{table}[h]%
  \caption{Tablas para dos proposiciones}
  \label{tbl:tablas-dos-prop}%
  \centering
  $$
    \begin{array}{|c|c||c|c|c|c|c|c|c|}
      \hline
      p & q & 0 & p & q & \neg p & \neg q & p \land q & p \lor q \\
      \hline
      \hline
      0 & 0 & 0 & 0 & 0 & 1 & 1 & 0 & 0 \\
      0 & 1 & 0 & 0 & 1 & 1 & 0 & 0 & 1 \\
      1 & 0 & 0 & 1 & 0 & 0 & 1 & 0 & 1 \\
      1 & 1 & 0 & 1 & 1 & 0 & 0 & 1 & 1 \\
      \hline
    \end{array}
  $$
\end{table}

Si bien es cierto que se puede generar una expresión sintácticamente
correcta tan grande como se desee, pues para ello basta combinar esas dos
proposiciones empleando los conectores y los paréntesis necesarios, no cabe
la menor duda de que esta expresión escrita debe tener una de las tablas de
verdad contenidas en la tabla~\ref{tbl:tablas-dos-prop}.

De esta forma se entiende que se pueden escribir muchas proposiciones, pero
necesariamente deben ser equivalentes a otras proposiciones que tienen una
escritura más corta. Con el fin de disponer de expresiones más cortas,
conviene mostrar las siguientes equivalencias, que son presentadas como
leyes lógicas.

A continuación, se presentan las principales leyes que se cumplen al
combinar dos proposiciones.





\subsubsection{Leyes de identidad}

\begin{minipage}[t]{.45\textwidth}
  \centering
  \begin{align*}
    p \lor 0 &\iff p \\
    p \land 1 &\iff p \\
    1 \rightarrow p &\iff p \\
  \end{align*}
\end{minipage}
\begin{minipage}[t]{.45\textwidth}
  \centering
  \begin{align*}
    p \land 0 &\iff \contrad \\
    p \lor 1 &\iff \tautol \\
  \end{align*}
\end{minipage}




\subsubsection{Leyes conmutativas}

El orden de las proposiciones no varía el valor.

\begin{align*}
  p \lor q &\iff q \lor p \\
  p \land q &\iff q \land p \\
  p \leftrightarrow q &\iff q \leftrightarrow p \\
\end{align*}

\noindent Como particularidad, advierta que no se cumple para el
condicional:

$$ p \to q \not\iff q \to p $$




\subsubsection{Leyes de De Morgan}

La negación de una disyunción es la conjunción de negaciones y, por otro
lado, la negación de una conjunción es la disyunción de conjunciones.

\begin{align*}
  \neg(p \lor q) &\iff \neg p \land \neg q \\
  \neg(p \land q) &\iff \neg p \lor \neg q \\
\end{align*}




\subsubsection{Leyes del condicional}

\begin{align*}
  p \rightarrow q &\iff \neg p \lor q \\
  p \rightarrow q &\iff \neg(p \land \neg q) \\
  p \rightarrow q &\iff (p \land q) \leftrightarrow p \\
\end{align*}

De estas cuatro leyes del condicional, la más destacada es aquella que
establece que $p \to q$ es equivalente a $\neg p \lor q$. Esta equivalencia
es fundamental en el estudio de la lógica proposicional, ya que permite
transformar implicaciones en una forma disyuntiva más manejable. Su utilidad
radica en simplificar la interpretación y manipulación de proposiciones
complejas en razonamientos formales.

La segunda no sería más que la aplicación de una de las leyes de De Morgan a
la primera, con lo que es fácil de recordar.




\subsubsection{Ley del bicondicional}

$$ p \leftrightarrow q \iff (p \rightarrow q) \land (q \rightarrow p) $$

Si se cumplen las dos posibles proposiciones condicionales entre dos
proposiciones $p$ y $q$, entonces $p$ y $q$ son equivalentes.

Esta ley se utiliza a menudo en las demostraciones en matemáticas para
demostrar que dos supuestos son equivalentes. Se demuestra que si el
supuesto primero es crierto, entonces el segundo también lo es. También, que
si el segundo es cierto, también lo es el primero. Por eso se dice, cuando
aparece el bicondicional, que una de las proposiciones es condición
suficiente y necesaria para la otra, y viceversa.





\subsubsection{Ley de contradicción}

$$ \neg p \rightarrow (q \land \neg q) \iff p $$

Es bastante más común en los textos en español que la llamen Ley de
Reducción al Absurdo.

Esta ley se usa con frecuencia en demostraciones en matemáticas. Para
demostrar que un enunciado es ceierto, se niega dicho enunciado y se
demuestra que de tal negación se deduce una proposición y su negación, lo
cual es una contradicción. Esta contradicción se ha producido por asumir que
el enunciado es falso, con lo que no queda otra más que este sea verdadero.

En realidad, existe una forma más elegante de explicarlo, pero para ello
necesitamos una ley que vemos un poco más adelante. Cuando lleguemos,
daremos una explicación más fácil de entender sobre las demostraciones por
contradicción.

Ejercicio. $\sqrt{2}$ no es un número racional.

% TKTK Poner en la teoría qué es una fracción irreducible.

Vamos a abordarlo por contradicción. Suponemos que $\sqrt{2}$ es racional.
Por tanto, existen $m, n \in \zset$ para los que $m/n = \sqrt{2}$. Podemos
suponer, además, sin pérdida de generalidad, que es una fracción
irreducible, es decir, $m$ y $n$, en sus respectivas descomposiciones en
factores primos, no tienen ningún factor en común, cosa que equivale a decir
que $m$ y $n$ son primos relativos, que es lo mismo que decir que su máximo
común divisor es 1.

Ahora, si elevamos al cuadrado en ambas partes de la ecuación, tenemos

\begin{align*}
  \frac{m}{n} &= \sqrt{2} \\
  m           &= n \sqrt{2} \\
  m^2
    &= \left(n \sqrt{2}\right)^2 = n^2 \left(\sqrt{2}\right)^2 = 2n^2 \\
\end{align*}

Por tanto, $2 \mid m^2$. Como sabemos por TKTK, si $2 \mid m^2$, entonces
$2 \mid m$. Por tanto, existe un $k \in \zset$ tal que $m = 2k$. Así,
pues,

\begin{align*}
  m     &= 2k \\
  m^2   &= (2k)^2 \\
  2n^2  &= 4k^2 \\
  n^2   &= 2k^2
\end{align*}

\noindent Advierta que hemos sustituido en una de estas transformaciones el
valor obtenido antes, es decir, $m^2 = 2n^2$.

De este resultado, $n^2 = 2k^2$, se deduce que también se da $2 \mid n^2$.
Entonces, al igual que antes, se tiene que $2 \mid n$. Hemos llegado
entonces a la conclusión de que el 2 es un divisor común de $m$ y $n$, es
decir, $2 \mid \{m, n\}$, lo cual se contradice con que ambos números sean
primos relativos.

Por tanto, la premisa de la que partimos será falsa. Esta era que $\sqrt{2}$
era un número racional, con lo que hemos demostrado que no lo es.

Ejercicio. Demuestre que existen infinitos números primos.

Se va a abordar por el método de contradicción.

Suponemos que existe una cantidad finita $n \in \nset$ con $n \geq 1$ de
números primos. Ese límite inferior es fácil demostrarlo puesto que sabemos
que, por ejemplo, el 2 es un número primo. Nos centraremos en los números
naturales mayores que 1, que son los únicos que pueden ser primos. Los
números primos son, en orden ascendente, $p_1 = 2$, $p_2 = 3$, $p_3 = 5$,
$p_4 = 7$, \ldots. Al último lo designamos por $p_n$.

Considere ahora al número que es la multiplicación de todos los primos.

$$ N = p_1 \, p_2 \, p_3 \, \cdots \, p_n $$

$N$ tiene que ser un entero positivo, ya que todos los primos son enteros
mayores que 1. Así que $N+1$ será un entero mayor que 1. Por el Teorema
Fundamental de la Aritmética, se tiene que $N+1$, puesto que ha de ser
necesariamente compuesto al no estar en esa lista de números primos, ha de
tener un factor primo $p$. Ese $p$ tiene que estar en la lista de factores
de $N$, ya que estos son todos los números primos. Por tanto, se da que $p
\mid N$.

Por otro lado, antes hemos dicho que $p$ es un factor de $N+1$, o lo que es
lo mismo, $p \mid (N+1)$. Entonces, por una proposición de la teoría de
números que yo suelo llamar Principio Dos de Tres, tenemos que $p$ será un
divisor de la resta de estos, es decir,

$$ p \mid ((N+1) - N) $$

\noindent o, lo que es lo mismo, $p \mid 1$. Pero ningún número primo es un
factor de 1. Por tanto, la hipótesis, es decir, que $p$ se encontraba en la
lista de los números primos, será falsa. Así, pues, $p$ será un ``nuevo''
número primo que no habíamos tenido en cuenta.

Esto mismo se puede hacer continuamente y así ir obteniendo sistemáticamente
nuevos números primos. Por tanto, la cantidad de números primos es
ilimitada.





\subsubsection{Leyes de transposición}

\begin{align*}
  p \to q &\iff \neg q \to \neg p \\
  p \leftrightarrow q &\iff \neg p \leftrightarrow \neg q
\end{align*}

Advierta que la primera de estas expresa que un condicional es equivalente a
su contrarrecíproco. Esta regla se emplea en muchas demostraciones
matemáticas. Por ejemplo, gracias a esta puede entender mejor por qué
funcionan las demostraciones por contradicción, que vimos un poco antes.
TKTK.

Si tiene una proposición condicional, eso indica, tal y como hemos visto,
que, si es cierto el antecedente, entonces también será cierto el
consecuente. A veces, resulta complicado demostrar algo con ese
razonamiento. La primera ley de transposición nos permite usar otro
razonamiento alternativo para el mismo fin. Simplemente, tenemos que
demostrar que es falso el consecuente y, entonces, gracias a esta regla, se
dará que automáticamente el antecedente también será falso.

La demostración de por qué funciona esta regla, al igual que sucede con
todas, está en su tabla de verdad.

Ejemplo. Demuestre que el límite de una sucesión, si existe, es único.

Lo primero que haremos será recordar la definición de límite de una sucesión
de números reales, por si no la recuerda.

Una sucesión de números reales $\{x_n\}$ converge a un $l \in \rset$
cuando para todo $\varepsilon \in \rset$ siendo $\varepsilon > 0$
existe un $n_\varepsilon \in \nset$ (que depende de $\varepsilon$) tal
que para todo número $n \in \nset$ que sea $n > n_\varepsilon$ se
cumple $|l - x_n| < \varepsilon$.

Vamos a demostrarlo por contradicción. Suponemos que la sucesión $\{x_n\}$
tiene dos límites: $r$ y $s$.

De entre todos los valores que puede tomar el $\varepsilon$ en la definición
de límite, vamos a tomar el que cumpla $2 \varepsilon = |r - s|$. Advierta
que $r$ y $s$ son parámetros, más que variables, con lo que ese
$\varepsilon$ tendrá un valor determinado para un problema con datos
concretos.

Por un lado, si usamos un hecho que conocemos de los números reales llamado
desigualdad triangular, podemos ver que

$$ 2 \varepsilon = |r - s| = |r - x_n + x_n - s| \leq |r - x_n| + |x_n - s|
= |r - x_n| + |s - x_n| $$

Por otro lado, por la definición de límite de una sucesión de números
reales, se tiene que para todo $n \in \nset$ tal que $n >
n_\varepsilon$, se tienen

\begin{align*}
  |r - x_n| &< \varepsilon \\
  |s - x_n| &< \varepsilon \\
\end{align*}

\noindent y, si sumamos ambas desigualdades, tenemos

$$ |r - x_n| + |s - x_n| < 2\varepsilon $$

\noindent Cosa que se contradice con lo anterior, es decir, llegamos a la
desigualdad $2\varepsilon < 2\varepsilon$, que evidentemente es falsa.
Aplicando entonces la regla de contradicción, se tiene que el límite de una
sucesión de números reales, en caso de existir, será siempre único.

Por cierto, aunque la desigualdad triangular es un hecho que conocemos de
los números reales, en realidad lo conocemos intuitivamente, sin haberlo
demostrado. Más adelante, cuando entremos de lleno en el estudio de los
conjuntos numéricos como $\nset$, $\rset$, etc., veremos de dónde sale.






\subsection{Leyes lógicas equivalentes con tres proposiciones}

Con tres proposiciones $p$, $q$ y $r$, y cualquier conjunto de operadores,
solo se pueden construir $256$ proposiciones que no sean equivalentes entre
sí. Esto se debe a que solo hay $256$ tablas de verdad distintas.

En la Tabla~\ref{tbl:op-3-proposiciones}, se intuyen las 256 tablas cuyos
valores de verdad se corresponden con las expresiones de los números del 0
al 255 en notación binaria: 00000000, 00000001, 00000010, \ldots, 11111111.

% TODO Quizás, pasarlo a array, en lugar de tabular, para que quede como los
% demás.

\begin{table}[h]
  \caption{Funciones posibles con 3 proposiciones}%
  \label{tbl:op-3-proposiciones}
  \centering
  $$
    \begin{array}{|c|c|c||c|c|c|c|c|c|}
      \hline
      p & q & r & f_0 & f_1 & f_2 & f_3 & \dots & f_{255} \\
      \hline
      \hline
      0 & 0 & 0 & 0 & 0 & 0 & 0 & \dots & 1 \\
      \hline
      0 & 0 & 1 & 0 & 0 & 0 & 0 & \dots & 1 \\
      \hline
      0 & 1 & 0 & 0 & 0 & 0 & 0 & \dots & 1 \\
      \hline
      0 & 1 & 1 & 0 & 0 & 0 & 0 & \dots & 1 \\
      \hline
      1 & 0 & 0 & 0 & 0 & 0 & 0 & \dots & 1 \\
      \hline
      1 & 0 & 1 & 0 & 0 & 0 & 0 & \dots & 1 \\
      \hline
      1 & 1 & 0 & 0 & 0 & 1 & 1 & \dots & 1 \\
      \hline
      1 & 1 & 1 & 0 & 1 & 0 & 1 & \dots & 1 \\
      \hline
    \end{array}
  $$
\end{table}

Como ya se ha indicado anteriormente, se puede generar una expresión
sintácticamente correcta tan grande como se desee, al combinar esas 3
proposiciones empleando operadores y los paréntesis necesarios. Cada
expresión escrita se corresponde con alguna de las 256 tablas de verdad
contenidas en la tabla~\ref{tbl:op-3-proposiciones}.

Las leyes lógicas siguientes nos permiten expresar operaciones con más de
dos proposiciones y una única conectiva sin paréntesis.





\subsubsection{Leyes asociativas}

\begin{align*}
  (p \lor q) \lor r &\iff p \lor (q \lor r) \\
  (p \land q) \land r &\iff p \land (q \land r) \\
  (p \leftrightarrow q) \leftrightarrow r &\iff p \leftrightarrow (q
    \leftrightarrow r) \\
\end{align*}

Cada ley asociativa establece la forma de operar con más de dos
proposiciones y una misma conectiva. Estas leyes permiten dotar de
significado a las expresiones

$$
\begin{array}{ccc}
  p \lor q \lor r & p \land q \land r & p \leftrightarrow q \leftrightarrow r
\end{array}
$$

Estas leyes permiten simplificar expresiones compuestas por tres
proposiciones y un único conector lógico, eliminando la necesidad de
paréntesis adicionales.




\subsubsection{Leyes distributivas}

\begin{minipage}[t]{.45\textwidth}
  \centering
  \begin{align*}
    p \lor (q \land r) &\iff (p \lor q) \land (p \lor r) \\
    p \land (q \lor r) &\iff (p \land q) \lor (p \land r) \\
  \end{align*}
\end{minipage}
\begin{minipage}[t]{.45\textwidth}
  \centering
  \begin{align*}
    p \to (q \lor r) &\iff (p \to q) \lor (p \to r) \\
    p \to (q \land r) &\iff (p \to q) \land (p \to r) \\
  \end{align*}
\end{minipage}






\subsection{Leyes lógicas condicionales}

Las leyes lógicas expuestas en los apartados anteriores son leyes donde se
muestra la equivalencia de dos proposiciones, y, por lo tanto, una puede ser
sustituida por la otra allí donde sea necesario.

En este apartado, se presentan nuevas tautologías compuestas por un
condicional entre dos proposiciones. Usualmente, a estas tautologías también
se las llama \emph{leyes}. Recordemos que para indicar que un condicional es
una tautología escribimos el símbolo `$\Longrightarrow$' y, al igual que con
las proposiciones equivalentes, cuando se escribe una implicación entre
proposiciones con un único símbolo `$\Longrightarrow$', las expresiones
situadas a la izquierda y derecha del símbolo constituyen las proposiciones
de la implicación aunque vayan sin paréntesis.

Con dos proposiciones $p$ y $q$, se tienen las leyes lógicas siguientes.





\subsubsection{Ley de simplificación condicional}

$$ p \land q \implies p $$




\subsubsection{Ley de ampliación disyuntiva}

$$ p \implies p \lor q $$




\subsubsection{Leyes de inferencia}

\begin{align*}
  \neg p \land (p \lor q) &\iff q \\
  p \land (\neg p \lor \neg q) &\iff \neg q \\
\end{align*}

Estas leyes de inferencia se denominan habitualmente \emph{silogismos
disyuntivos}. Aunque se puede llegar a estas fácilmente manipulando las
expresiones en base a otras reglas, como la distributiva, son evidentes a
simple vista. Por ejemplo, la primera se puede interpretar del modo
siguiente: Si $p \lor q$ es cierto y, además, se sabe que $p$ es falso,
entonces $q$ será cierto.




\subsubsection{Ley \emph{modus ponendo ponens}}

O, de forma abreviada, ley \emph{modus ponens}. Supuesto cierto el
condicional $p \to q$, si se afirma el antecedente, es decir, $p$, entonces,
será cierto también el consecuente, $q$.

$$ (p \to q) \land p \implies q $$

Ejemplo. Si llueve, entonces el suelo se moja.

Se puede descomponer en proposiciones simples unidas por un operador
condicional:

\begin{tabular}{ll}
  $p$: & Llueve. \\
  $q$: & El suelo se moja. \\
\end{tabular}

La proposición compuesta sería

$$ p \to q $$

Si, además, se da $p$, es decir, que llueve, se tiene, según la ley
\emph{modus ponens},

$$ (p \to q) \land p \implies q $$

\noindent es decir, será cierto $q$. Esto se intepretaría como: Si llueve,
el suelo se moja. Adeás, está lloviendo. Por tanto, el suelo estará mojado.





\subsubsection{Ley \emph{modus tollendo tollens}}

O, de forma abreviada, ley \emph{modus tollens}. Supuesto cierto el
condicional $p \to q$, si no se cumple el consecuente, $q$, necesariamente
no se cumple el antecedente, $p$. Simbólicamente:

$$ (p \to q) \land \neg q \implies \neg p $$

Se podría ver mejor sabiendo que $p \to q$ es equivalente a $\neg q \to \neg
p$, tal y como vimos antes. Entonces, la expresión \emph{modus tollens}
sería equivalente a

$$ (\neg q \to \neg p) \land \neg q \implies \neg p $$

\noindent que sería cierta por la regla \emph{modus ponens}, de la sección
anterior.

Siguiendo con el ejemplo anterior, imagine que ve que el suelo no está
mojado, $\neg q$. De esto puede deducir que no ha llovido, $\neg p$.

Otro ejemplo. Si la función $f(x) = {-x}^2$ tiene un máximo local en el
punto $x_0$, entonces $f'(x_0) = 0$. Resulta que comprobamos que $f'(x_0) >
0$. Por tanto, la función $f(x)$ no tiene un máximo local en $x_0$.

Con tres o más proposiciones, $p$, $q$, y $r$, se tienen varias leyes que el
lector puede encontrar entre los enunciados de los ejercicios propuestos.









