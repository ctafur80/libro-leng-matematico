

El concepto del que partimos en el estudio de la lógica es el de uno que
estudiamos en la lengua: \emph{enunciado}, que, en español, recibe muchos
otros nombres: \emph{oración}, \emph{frase}, \emph{sentencia}, etc. A veces,
en ciertos contextos, estas distintas formas de llamarlo pueden tener
sutilezas y significar cosas ligeramente distintas. En inglés, los suelen
llamar \emph{sentence} o \emph{phrase}.\footnotemark

\footnotetext{Desde un punto de vista de la abstracción, en inglés, por
``debajo'' del concepto de \emph{sentence} se encontraría el concepto de
\emph{utterance}, que englobaría a cualquier enunciado que no tendría por
qué estar de acuerdo con las reglas de la gramática de una lengua ni de
tener un significado en esta. No sé bien cómo se traduciría dicho término.
Quizás, \emph{expresión verbal} o \emph{verbalización}.}

Un enunciado es necesario que sea gramatical, para que la comunicación sea
efectiva. Esto quiere decir que su sintaxis es válida y su semántica TKTK.
Por ejemplo, en español, una lengua natural, el enunciado

\begin{quote}
  Eléctrica ordenador el máquina es una.
\end{quote}

\noindent no es sintácticamente correcto, mientras que

\begin{quote}
  El orden es una máquina eléctrica.
\end{quote}

\noindent sí lo es.

Los enunciados pueden clasificarse, según la intención de quien las
pronuncia, en varios tipos. Uno de ellos son las oraciones informativas, que
también hay quien llama declarativas, o, también, \emph{afirmaciones}
(\emph{statements}). Evidentemente, el propósito de estas es dar una
información y, de cada caso particular de estas, se podrá decir si son
\emph{verdaderas} (\emph{true}) o \emph{falsas} (\emph{false}). Entre los
otros tipos de oraciones, se tienen, por ejemplo, las interrogativas, las
exclamativas, las imperativas, etc. En estas, no tiene sentido decir si son
verdaderas o falsas.

A esa evaluación, es decir, determinar si una afirmación es verdadera o
falsa, es a lo que en lógica se conoce como su valor semántico lógico o
\emph{valor de verdad} (o, también, \emph{valor de veracidad}, \emph{valor
lógico} o \emph{valor buleano}). Independientemente de este, una afirmación
puede tener más valores semánticos, pero a la lógica solo le interesa este.

Así, una afirmación en español como

\begin{quote}
  La vaca es un animal.
\end{quote}

\noindent tiene un valor semántico verdadero, mientras que

\begin{quote}
  La piedra es un animal.
\end{quote}

\noindent es una afirmación con uno falso.

A las afirmaciones, en el estudio de la lógica también se las llama
\emph{proposiciones} (\emph{propositions}). Creo que hay quien considera que
existe una diferencia sutil entre \emph{statement} y \emph{proposition},
pero, para nosotros, serán lo mismo. Lo más común en lógica es hablar de
proposiciones, en lugar de hacerlo de afirmaciones.

En lo que respecta a los valores de verdad, verdadero y falso, estos se
suelen representar como $V$ y $F$, respectivamente, en filosofía (en ingles,
$T$ y $F$). En matemáticas y ciencias de la computación, sin embargo, se
representan como 1 y 0, respectivamente.

Las proposiciones pueden ser \emph{simples} (\emph{simple}; también llamadas
\emph{atómicas}, \emph{atomic}) o \emph{compuestas} (\emph{compound};
también llamadas \emph{moleculares}; \emph{molecular}). A las proposiciones
simples algunos libros las prefieren llamar \emph{variables lógicas}
(\emph{logical variables}) o \emph{variables buleanas} (\emph{Boolean
variables}). Las compuestas son composiciones de proposiciones usando
operadores llamados \emph{conectivas lógicas} (o, simplemente,
\emph{conectivas}; \emph{logical connectives}, \emph{connectives}). También
se las puede llamar \emph{conectores lógicos} u \emph{operadores lógicos} o,
simplemente, \emph{conectores} u \emph{operadores}.

En una proposición compuesta, se puede dar el anidamiento (o, lo que es lo
mismo, la recursión, pero eso lo veremos más adelante), y, de hecho, es lo
más usual. Es decir, una proposición compuesta puede constar de
proposiciones simples unidas mediante operadores o bien por proposiciones
compuestas y proposiciones simples, uniendo mediante operadores al igual que
antes.

\iffalse
Personalmente, la forma que prefiero usar es \emph{operador}, pues por
ejemplo se tiene la negación que no conecta ni une proposiciones, sino que
se usa sobre una sola.
\fi

Tal y como dijimos en la sección anterior, nos encontramos dentro de la
lógica formal simbólica, por lo que usamos una simbología con la que nos
comunicamos sin ambigüedad y de forma muy directa. Así, usamos letras como
si se tratara de variables para designar a una proposición. Normalmente, se
usa la regla de estilo de usar una letra minúscula en tipografía en
itálicas, como se hace normalmente con las variables en muchos de los campos
de las matemáticas. Así, por ejemplo, se podría tener la variable lógica $p$
que representaría a la proposición

\begin{quote}
  Lloverá mañana.
\end{quote}

\noindent Como se suele hacer cuando se usan variables en otros ámbitos, se
suele abusar de la lengua y decir cosas como ``la proposición $p$\ldots''.

Como puede imaginar, que se use la letra $p$ se debe a que es la letra con
la que comienza la palabra \emph{proposición}. Si se tienen varias
proposiciones, se suele seguir en orden alfabético:  $p$, $q$, etc. En otros
libros he visto que siguen la regla de designar a las proposiciones con
letras mayúsculas: $P$, $Q$, etc. Evidentemente, esta cuestión de estilo no
tiene relevancia para los resultados.

Al tratarse de una lengua (o lenguaje dirían mal traducido), la lógica
formal cuenta también con formas de combinar las proposiciones. Se tienen
así, los operadores. Por ejemplo, se tiene el operador negación, que se
suele representar por el símbolo `$\neg$', que afecta a una sola proposición
y, por tanto, se le califica de operador unario. Tiene muchas otras
representaciones, como TKTK. En nuestro caso, `$\neg$', se trata de un
operador prefijo, como en $\neg p$.

% TODO Notación en informática

Normalmente, el operador negación será el único operador unario que usemos
en nuestro estudio de la lógica.

Dentro de una expresión lógica, se conoce como \emph{literal}
(\emph{literal}) a una proposición, como $p$, o a la negación de una; por
ejemplo, $\neg p$.

Personalmente, aunque considero al operador negación un operador, no lo
considero una conectiva, al ser un operador unario.

Hay también operadores que actúan sobre más de una proposición. Se tiene,
por ejemplo, al operador conjunción, que se suele representar por `$\land$'.
Se trataría de un operador que actúa sobre dos proposiciones, pero, tal y
como veremos más adelante, en realidad puede actuar sobre un número
arbitrario de estas simultáneamente. Toma la forma de un operador infijo, es
decir, se ubica entre las dos proposiciones; por ejemplo, $p \land q$. Si
actuase, por ejemplo, sobre cuatro proposiciones, podría verse una expresión
como la siguiente: $p \land q \land r \land s$.

% TODO Hablar de la notación afija.

Los operadores básicos que usaremos serán: negación, disyunción, conjunción,
condicional y bicondicional. Existen muchos más. De hecho, veremos cuántos
pueden existir, dependiendo del número de variables lógicas sobre las que
operen. Dedicaremos una sección a estudiar cada uno de los básicos, además
de que tocaremos algo de otros menos básicos.

De todos modos, tal y como se demuestra en las asignaturas de diseño lógico
digital, bastaría con un solo operador equivalente a una puerta NAND, o,
alternativamente, uno equivalente a una puerta NOR, para poder obtener todos
los demás y, por tanto, llegar a obtener expresiones de cualquiera de las
funciones lógicas posibles. Esto no lo trataremos aquí.

Algo que también debe saber es que la lógica en realidad no se dedica a
analizar qué es verdad y qué no; esa es una tarea demasiado ambiciosa. Lo
que estudia son simplemente formas correctas de razonar. Por tanto, una
proposición puede contar con una estructura que la haga válida para la
lógica y aun así ser falsa cuando se encuentra en el mundo real.

Esto lo explica muy bien el profesor Philip Stark, de la Universidad de
Berkeley, en su asignatura de introducción a la estadística.

Lo que explica es que, en principio, el razonamiento se puede dividir en
\emph{inductivo} y \emph{deductivo}. El deductivo es lo que informalmente se
llama \emph{lógico}. Es una forma de pensar matemáticamente sobre todo tipo
de cosas. Dado un conjunto de suposiciones (también llamadas premisas), ¿qué
será cierto entonces? Por el contrario, el conocimiento inductivo trata de
generalizar desde la experiencia (dicho de otro modo, desde los datos) a
situaciones nuevas.\footnotemark ¿Cómo de fuerte es la evidencia de que una
afirmación sobre el mundo es cierta o falsa? El conocimiento inductivo es
inherentemente incierto. Por el contrario, el conocimiento deductivo es
igual de cierto que pueden serlo las matemáticas. Gran parte de la materia
de la estadística trata con el conocimiento inductivo. Se toma un especial
cuidado en extraer conclusiones fiables mediante razonamiento inductivo.
Además, el buen razonamiento inductivo requiere de razonamiento deductivo
correcto.

\footnotetext{Existe también el Principio de Inducción, que sí es riguroso.
A este último también hay quien lo llama \emph{inducción matemática}, para
distinguirlo del otro que mencionamos aquí, que es menos formal. Este lo
verá más adelante. Es uno de los puntos de partida en la formalización de
los números naturales.}

El razonamiento deductivo correcto es \emph{válido} (\emph{valid}). El
razonamiento deductivo incorrecto es \emph{falaz} (\emph{fallacious}). El
razonamiento puede ser válido aun cuando las suposiciones en las que se basa
son falsas. Si el razonamiento es válido y basado en premisas ciertas,
entonces se dice que es \emph{sólido} (\emph{sound}).

En realidad, no es tan sencillo. El término \emph{sound} se refiere a
argumentos o a sistemas formales. No a proposiciones. TKTK.



% --------------------------------------------


La lengua de las matemáticas, aun siendo una lengua formal simbólica, muchas
veces es imprecisa, por comodidad. Así, puede que vea en un texto que hablan
de un número, sin especificar de qué tipo concreto. O, si ve que aparece la
expresión $a + b$, eso puede representar, por ejemplo, una suma de números
enteros o una (suma) de matrices.

Además, para evitar que sea enorme la simbología que se usa, se da la
polisemia. Así, una expresión como $a \leq b$ puede representar la relación
de orden ``menor o igual que'' que se da en los conjuntos de números, o, si
estamos en el álgebra abstracta, puede representar a una relación de orden
cualquiera, en general.

Debido a esto, en parte sí que hay que estar pendiente del contexto, aunque
se trate de una lengua formal.

% TODO Donde sí que no hay que estar pendiente es en Lean.





\iffalse

Las proposiciones son lo que constituye los \emph{argumentos}
(\emph{arguments}) en la lógica. Estos no son más que TKTK. En matemáticas,
un argumento podría ser, por ejemplo, una demostración de un teorema. Esta
no es más que una proposición compuesta con varios niveles de anidamiento.

Todo esto se explica en \emph{An Introduction to Formal Logic} de Peter
Smith. En el capítulo 7, titulado \emph{Propositions and forms}. Aunque
seguramente sea demasiado para una asignatura de este tipo. No se necesita
tener una definición rigurosa del concepto de \emph{proposición}.

\fi



Además de las proposiciones, en el capítulo siguiente se verá una
generalización de estas: los predicados, y, más adelante, una generalización
de estos mismos, las relaciones. De momento, basta con que sepa que en
matemáticas existen muchos enunciados que no se pueden representar por medio
de proposiciones.





