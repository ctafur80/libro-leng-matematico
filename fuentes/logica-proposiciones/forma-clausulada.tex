


Dada una proposición compesta por un conjunto de proposiciones simples $p$,
$q$, $r$, \ldots, se trata de encontrar una proposición equivalente a la
primera que esté escrita únicamente como conjunción ($\land$) de
proposiciones disyuntivas ($\lor$) de literales. También, se admite que se
tenga una sola proposición TKTK.

Tal y como hemos explicado antes, a lo que nos referimos con \emph{literal}
es a una proposición simple, por ejemplo, $p$, o su negación, $\neg p$.

Un ejemplo de proposición en forma clausulada sería:

$$ (p \lor \neg q) \land (q \lor r) \land (p \lor \neg r) $$

Para una proposición dada, una proposición que sea equivalente a esta y esté
formada por conjunciones de disyunciones de literales se dice que es la
\emph{forma clausulada} de la primera. También se la llama su \emph{forma
normal conjuntiva}.

Dentro de la expresión en forma clausulada, a las expresiones que hay entre
cada conjunción se las conoce como \emph{cláusulas lógicas}.

Disponer de la forma clausulada de una proposición facilita saber si esta es
verdadera, ya que es condición suficiente para que la proposición sea falsa
que alguna de las cláusulas lógicas sea falsa, en su forma clausulada.

Ejemplo. La primera ley del condicional,

$$ p \to q \iff \neg p \lor q $$

\noindent establece la forma clausulada de un condicional. La forma
clausulada de $p \to q$ está formada unr una única cláusula: $\neg p \lor
q$. No aparece la conjunción porque se trataría de una expresión en forma
normal conjuntiva con una única cláusula.

La segunda ley del condicional,

$$ p \to q \iff \neg(p \land \neg q) $$

\noindent no presenta una forma clausulada con dos cláusulas puesto que
existe una negación que afecta a la conjunción.

% ¿Se puede siempre obtener la forma clausulada de cualquier proposición
% lógica que nos den?

Ejemplo. La ley del bicondicional establede que $p \leftrightarrow q$ se
puede expresar como $(p \to q) \land (q \to p)$.

Al aplicar la primera ley del condicional a cada una de las expresiones
entre paréntesis, se obtiene la forma clausulada de un bicondicional, que en
este caso sería $(\neg p \lor q) \land (\neg q \lor p)$; compuesta por dos
cláusulas.





\subsubsection{Pasos para obtener la forma clausulada}

A continuación, establacemos los pasos recomendados para extraer la forma
clausulada de una proposición.

Paso 1. Sustitución de los conectores bicondicionales. Se transforma cada
bicondicional en una conjunción de bicondicionales. Concretamente, se
sustituye $p \leftrightarrow q$ por $(p \to q) \land (q \to p)$.

Paso 2. Sustitución de los conectores condicionales. Se utiliza la primera
ley del condicional. Concretamente, se sustituye $p \to q$ por $\neg p \lor
q$.

Paso 3. Sustitución de los conectores que actúan sobre una proposición
conjunción o disyunción. Se utiliza la ley de De Morgan que corresponda para
transformar cada negación de una disyunción o conjunción de literales. Es
decir,

\begin{center}
\begin{tabular}{ll}
  Se sustituye $\neg(p \land q)$ por $\neg p \lor \neg q$. \\
  Se sustituye $\neg(p \lor q)$ por $\neg p \land \neg q$. \\
\end{tabular}
\end{center}

Paso 4. Utilización de las leyes distributivas, asociativas y conmutativas
para generar las cláusulas y, por tanto, la forma clausulada.

Ejercicio. Determine la forma clausulada de la proposición

$$ p \to (p \land q) $$

\noindent Hagamos algunas transformaciones:

$$
\begin{array}{L T}
  p \to (p \land q)
    &  \\

  \neg p \lor (p \land q)
    & primera ley del condicional \\

  (\neg p \lor p) \land (\neg p \lor q)
    & propiedad distributiva \\

  \tautol \land (\neg p \lor q)
    & ley del tercio excluso \\

  \neg p \lor q
    & ley de identidad \\
\end{array}
$$




Si se fija, la forma clausulada es la misma que la de $p \to q$. Por lo
tanto se tendrá que

$$ p \to (p \land q) \iff p \to q $$

% TODO Algo que explica luego es que, si dos proposiciones tienen una misma
% forma clausulada, entonces esas dos proposiciones son equivalentes. Esto
% lo que viene a decir también es que, para una proposición, su forma
% clausulada es única.

Ejemplo. Comprobación de una tautología mediante la forma clausulada.

Construyamos paso a paso la forma clausulada de la proposición

$$ [(p \to q) \land (q \to r)] \to (p \to r) $$

Será como sigue:

\begin{align*}
  & [(p \to q) \land (q \to r)] \to (p \to r) \\
  & \neg[(p \to q) \land (q \to r)] \lor (p \to r) \\
  & \neg[(\neg p \lor q) \land (\neg q \lor r)] \lor (\neg p \lor r) \\
  & [\neg(\neg p \lor q) \land \neg(\neg q \lor r)] \lor (\neg p \lor r) \\
  & [(\neg\neg p \lor \neg q) \lor (\neg\neg q \lor \neg r)] \lor (\neg p
    \lor r) \\
  & [\{(p \lor \neg q) \lor q\} \land \{(p \land \neg q) \lor \neg r\}] \lor
    (\neg p \lor r) \\
  & [\{(p \lor q) \land (\neg q \lor q)\} \land \{(p \lor \neg r) \land
    (\neg q \lor \neg r)\}] \lor (\neg p \lor r) \\
  & [\{(p \lor q) \land \tautol\} \land \{(p \lor \neg r) \land (\neg q \lor
    \neg r)\}] \lor (\neg p \lor r) \\
  & [(p \lor q) \land (p \land \neg r) \land (\neg q \lor \neg r)] \lor
    (\neg p \lor r) \\
  & [(p \lor q) \lor (\neg p \lor r)] \land [(p \lor \neg r) \lor (\neg p
    \lor r)] \land [(\neg q \lor \neg r) \lor (\neg p \lor r)] \\
  & [p \lor q \lor \neg p \lor r] \land [p \lor \neg r \lor \neg p \lor r]
    \land [\neg q \lor \neg r \lor \neg p \lor r] \\
  & [(p \lor \neg p) \lor q \lor r] \land [(p \lor \neg p) \lor (r \lor \neg
    r)] \land [(r \lor \neg r) \lor \neg p \lor \neg q] \\
  & [\tautol \lor q \lor r] \land [\tautol \lor \tautol] \land [\tautol \lor
    \neg p \lor \neg q] \\
  & \tautol \lor \tautol \lor \tautol \\
  & \tautol \\
\end{align*}

\iffalse
% TODO Terminar. No cabe
$$
\begin{array}{L T}
  [(p \to q) \land (q \to r)] \to (p \to r) \\

  \neg[(p \to q) \land (q \to r)] \lor (p \to r)
    & primera ley del condicional \\

  \neg[(\neg p \lor q) \land (\neg q \lor r)] \lor (\neg p \lor r)
    & primera ley del condicional \\

  [\neg(\neg p \lor q) \land \neg(\neg q \lor r)] \lor (\neg p \lor r)
    & \\

  [(\neg\neg p \lor \neg q) \lor (\neg\neg q \lor \neg r)] \lor (\neg p
    \lor r)
    &  \\

  [\{(p \lor \neg q) \lor q\} \land \{(p \land \neg q) \lor \neg r\}] \lor
    (\neg p \lor r)
    &  \\

  [\{(p \lor q) \land (\neg q \lor q)\} \land \{(p \lor \neg r) \land
    (\neg q \lor \neg r)\}] \lor (\neg p \lor r)
    &  \\

  [\{(p \lor q) \land \tautol\} \land \{(p \lor \neg r) \land (\neg q \lor
    \neg r)\}] \lor (\neg p \lor r)
    &  \\

  [(p \lor q) \land (p \land \neg r) \land (\neg q \lor \neg r)] \lor
    (\neg p \lor r)
    &  \\

  [(p \lor q) \lor (\neg p \lor r)] \land [(p \lor \neg r) \lor (\neg p
    \lor r)] \land [(\neg q \lor \neg r) \lor (\neg p \lor r)]
    &  \\

  [p \lor q \lor \neg p \lor r] \land [p \lor \neg r \lor \neg p \lor r]
    \land [\neg q \lor \neg r \lor \neg p \lor r]
    &  \\

  [(p \lor \neg p) \lor q \lor r] \land [(p \lor \neg p) \lor (r \lor \neg
    r)] \land [(r \lor \neg r) \lor \neg p \lor \neg q]
    &  \\

  [\tautol \lor q \lor r] \land [\tautol \lor \tautol] \land [\tautol \lor
    \neg p \lor \neg q]
    &  \\

  \tautol \lor \tautol \lor \tautol
    & \\

  \tautol
    & \\
\end{array}
$$
\fi

Ejemplo. Tabla de verdad mediante la forma clausulada. La forma clausulada
de la proposición

$$ [(p \to q) \land (\neg p \to r)] \to (q \to r) $$

\noindent que designaremos de forma abreviada como $f$, es la proposición
$\neg q \lor \neg p \lor r$. Veamos cómo se deduce.

Paso 1. Quitar condicionales.

\begin{align*}
  & \neg[(p \to q) \land (\neg p \to r)] \lor (q \to r) \\
  & \neg[\neg(p \lor q) \land (\neg\neg p \lor r)] \lor (\neg q \lor r) \\
  & \neg[\neg(p \lor q) \land (p \lor r)] \lor (\neg q \lor r) \\
\end{align*}

Paso 2. Quitar negaciones de proposiciones compuestas, haciendo uso de las
Leyes de De Morgan.

\begin{align*}
  & [\neg(\neg p \lor q) \lor \neg(p \lor r)] \lor (\neg q \lor r) \\
  & [(\neg\neg p \lor \neg q) \lor (\neg p \land \neg r)] \lor (\neg q \lor
    r) \\
  & [(p \land \neg q) \lor (\neg p \land \neg r)] \lor (\neg q \lor r) \\
\end{align*}

Paso 3. Aplicar las leyes distributiva, asociativa y conmutativa.

\begin{align*}
  & [(p \land \neg q) \lor (\neg p \land \neg r)] \lor (\neg q \lor r) \\
  & \{[(p \land \neg q) \lor \neg p] \land [(p \land \neg q) \lor \neg r]\}
    \lor (\neg q \lor r) \\
  & \{[(p \lor \neg p) \land (\neg q \lor \neg p)] \land [(p \lor \neg r)
    \land (\neg q \lor \neg r)]\} \lor (\neg q \lor r) \\
  & \{[1 \land (\neg q \lor \neg p)] \land [(p \lor \neg r) \land (\neg q
    \lor \neg r)]\} \lor (\neg q \lor r) \\
  & [(\neg q \lor \neg p) \land (p \lor \neg r) \land (\neg q \lor \neg r)]
    \lor (\neg q \lor r) \\
  & [(\neg q \lor \neg p) \lor (\neg q \lor r)] \land [(p \lor \neg r) \lor
    (\neg q \lor r)] \land [(\neg q \lor \neg r) \lor (\neg q \lor r)] \\
  & (\neg q \lor \neg p \lor \neg q \lor r) \land (p \lor \neg r \lor \neg q
    \lor r) \land (\neg q \lor \neg r \lor \neg q \lor r) \\
  & (\neg q \lor \neg p \lor r) \land (p \lor \neg q \lor \neg r \lor r)
    \land (\neg q \lor \neg r \lor r) \\
  & (\neg q \lor \neg p \lor r) \land (p \lor \neg q \lor 1) \land (\neg q
    \lor 1) \\
  & \neg q \lor \neg p \lor r \\
\end{align*}

Se puede comprobar que las tablas de verdad que se obtienen son iguales, en
cualquiera de las expresiones: la original, tabla~\ref{tbl:ej-clausulada-1},
y la clausulada, tabla~\ref{tbl:ej-clausulada-2}. Se trata, por tanto, de
proposiciones equivalentes. Evidentemente, es más fácil construir la tabla
de verdad de la expresión en forma clausulada.

\begin{table}%
  \caption{Tabla de la proposición $[(p \to q) \land (\neg p \to r)] \to (q
    \to r)$}
  \label{tbl:ej-clausulada-1}
  \centering
  $$
    \begin{array}{|c|c|c||c|c|c|c|c|c|}
      \hline
      p & q & r & \neg p & p \to q & \neg p \to r & q \to r
        & (p \to q) \to (\neg p \to r) & f \\
      \hline
      \hline
      0 & 0 & 0   & 1 & 1 & 0 & 1 & 0 & 1 \\
      0 & 0 & 1   & 1 & 1 & 1 & 1 & 1 & 1 \\
      0 & 1 & 0   & 1 & 1 & 0 & 0 & 0 & 1 \\
      0 & 1 & 1   & 1 & 1 & 1 & 1 & 1 & 1 \\
      1 & 0 & 0   & 0 & 0 & 1 & 1 & 0 & 1 \\
      1 & 0 & 1   & 0 & 0 & 1 & 1 & 0 & 1 \\
      1 & 1 & 0   & 0 & 1 & 1 & 0 & 1 & 0 \\
      1 & 1 & 1   & 0 & 1 & 1 & 1 & 1 & 1 \\
      \hline
    \end{array}
  $$
\end{table}

\begin{table}%
  \caption{Tabla de la proposición $\neg p \lor \neg q \lor r$}
  \label{tbl:ej-clausulada-2}
  \centering
  $$
    \begin{array}{|c|c|c||c|c|c|}
      \hline
      p & q & r & \neg p & \neg q & \neg p \lor \neg q \lor r \\
      \hline
      \hline
      0 & 0 & 0   & 1 & 1 & 1 \\
      0 & 0 & 1   & 1 & 1 & 1 \\
      0 & 1 & 0   & 1 & 0 & 1 \\
      0 & 1 & 1   & 1 & 0 & 1 \\
      1 & 0 & 0   & 0 & 1 & 1 \\
      1 & 0 & 1   & 0 & 1 & 1 \\
      1 & 1 & 0   & 0 & 0 & 0 \\
      1 & 1 & 1   & 0 & 0 & 1 \\
      \hline
    \end{array}
  $$
\end{table}













