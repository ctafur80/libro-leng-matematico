



Un \semph{sistema formal} (\emph{formal system}) es una estructura abstracta
compuesta por símbolos, reglas y procedimientos definidos con precisión,
utilizada para estudiar la inferencia deductiva, la demostración de teoremas
y la consistencia de argumentos. Estos sistemas buscan eliminar ambigüedades
al operar con elementos puramente sintácticos (formales), independientemente
de su significado semántico. Sus componentes básicos suelen incluir:

\begin{description}
  \item[Un alfabeto] Es decir, un conjunto de símbolos primitivos como
    pueden ser letras, operadores lógicos o signos matemáticos, entre otros.
  \item[Reglas de formación] Normas para construir \emph{fórmulas bien
    formadas}, es decir, expresiones válidas. Se podría decir que las reglas
    de formación constituyen la sintaxis, ya que un sistema formal es
    básicamente una lengua.
  \item[Axiomas] Proposiciones iniciales que se aceptan sin demostración.
  \item[Reglas de inferencia] Procedimientos para derivar proposiciones
    nuevas a partir de axiomas o de proposiciones previas. En los sistemas
    formales de las matemáticas a esas proposiciones se las suele llamar
    \emph{teoremas}.
\end{description}

Ejemplos de sistemas formales

\begin{description}
  \item[Lógica proposicional] -

    \begin{itemize}
      \item Alfabeto. Letras proposicionales ($p$, $q$, $r$; también las
        llaman \emph{variables lógicas}), operadores lógicos ($\neg$,
        $\land$, $\lor$, $\rightarrow$), paréntesis.

      \item Reglas. Fórmulas como $p \rightarrow q$ son válidas.

      \item Axiomas. Por ejemplo, $p \rightarrow (q \rightarrow p)$.

      \item Regla de inferencia. Por ejemplo, la regla \emph{modus ponens}:
        si $p \rightarrow q$ y $p$, entonces $q$.
    \end{itemize}

  \item[Aritmética de Peano] -

    \begin{itemize}
      \item Alfabeto. Símbolos numéricos (0, $S$, $+$, $\times$), variables.
      \item Axiomas. Postulados como ``0 es un número natural'' y ``Si $n$
        es natural, $S(n)$ también lo es''.
      \item Objetivo. Formalizar las propiedades de los números naturales.
    \end{itemize}

  \item[Geometría euclidiana] -

    \begin{itemize}
      \item Axiomas. Los cinco postulados de Euclides (ej: ``Por dos puntos
        pasa una única recta'').
      \item Reglas. Deducción de teoremas geométricos mediante razonamiento
        deductivo.
    \end{itemize}

  \item[Teoría de conjuntos ZFC] -

    \begin{itemize}
      \item Alfabeto. Símbolos de pertenencia ($\in$), igualdad,
        cuantificadores.
      \item Axiomas. Como el axioma de extensionalidad o el axioma de
        elección.
      \item Propósito. Fundamentar las matemáticas desde la teoría de
        conjuntos.
    \end{itemize}

  \item[Cálculo lambda] -

    \begin{itemize}
      \item Alfabeto. Variables, símbolos de abstracción ($\lambda$),
        aplicación.
      \item Reglas. Reducción de expresiones para estudiar funciones y
        computación.
    \end{itemize}
\end{description}

Teoremas como los de \semph{Gödel} (incompletitud) muestran que ciertos
sistemas formales no pueden demostrar todas sus verdades ni garantizar su
propia consistencia, lo que ha generado debates filosóficos sobre los
límites del conocimiento formalizado.







