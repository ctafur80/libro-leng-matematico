


El primero al que se le ocurrió crear el campo del conocimiento que se
conoce como \emph{lógica} (\emph{logic}) fue a Aristóteles, el famoso
filósofo de la Antigua Grecia. Pero, aunque antes de esto la lógica no
existiese como disciplina, sí se encontraba oculta en todas y cada una de
las lenguas de la humanidad.

La lógica no es más que las reglas de razonamiento que se encuentran en
todas las lenguas naturales, aunque sus hablantes no seamos conscientes de
ello. En gran medida, tiene bastante relación con la lingüística, pero no
son lo mismo. Tanto la lingüística como la lógica se encontraban ``dentro''
de las lenguas naturales antes de que sus hablantes se diesen cuenta de
ello. Los estudiosos de estos campos lo único que hacen es establecer una
teoría que explique ciertos comportamientos, a la vista del uso real de las
distintas lenguas naturales.

Independientemente de la curiosidad intelectual que pudo conducir al
nacimiento del corpus teórico de la lógica, esta tiene también una razón
práctica: llegar a sistematizar el razonamiento. Gracias a la lógica,
podemos obtener conocimiento a partir de cierto conocimiento previo, en
situaciones en las que, sin el uso de un método sistemático, sería bastante
complicado llegar a ciertas conclusiones.

Llegado un punto en el estudio de la lógica, se prefirió pasar a usar una
herramienta que nos auxiliara en la práctica de esta. Esa herramienta son
las lenguas formales, en contraposición a las lenguas
naturales.\footnotemark Esto se debe a que las lenguas naturales cuentan con
bastantes carencias, desde el punto de vista de la eficiencia. Por ejemplo,
suelen tener redundancia. Otro problema del que adolecen es que, para hallar
el significado de una expresión, se suele requerir de información contextual
que no aparece de forma explícita en esta.

\footnotetext{Quizás le ``chirríe'' que diga cosas como \emph{lengua
formal}, en lugar de \emph{lenguaje formal}. TKTK.}

% TODO lengua idioma

En una lengua formal, se solucionan estos problemas. Principalmente, el
segundo de estos, es decir, a cada expresión le corresponde, sin ambigüedad,
un único significado.

Lo que se hace en la lógica, entonces, es transformar una expresión de una
lengua informal (natural) a una que consideremos equivalente en esa lengua
formal, y, una vez con esta nueva expresión, realizar las manipulaciones que
estén permitidas en la lógica para esas expresiones. A esas manipulaciones
permitidas se las suele conocer como \emph{leyes de inferencia lógica}, o,
más brevemente, \emph{leyes} o \emph{reglas}.

De hecho, esas lenguas formales que emplea la lógica suelen ser simbólicas.
Como si se tratase de notación matemática o formulación química. De hecho,
estas son lenguas formales que emplean la lógica. La filosofía cuenta
también con su lengua formal para la lógica. Es algo distinta a la de las
matemáticas. Pero, en el fondo, ambas usan la misma lógica y, por tanto,
tienen muchos puntos comunes.

A la lógica formal que emplee una lengua formal simbólica se la suele llamar
\emph{lógica formal simbólica} o, simplemente, \emph{lógica simbólica}, pues
se sobrentiende que, si es simbólica, será también formal.

Las matemáticas, al igual que otras disciplinas de la ciencia o la
tecnología, cuenta con una lengua propia que hace uso de la lógica formal
simbólica.

Un ejemplo de aplicación de la lógica en el lenguaje de las matemáticas
sería una demostración de un teorema. Esta sería un argumento lógico. En
lógica, un \emph{argumento}\footnotemark no es más que TKTK.

\footnotetext{El término en inglés puede dar lugar a confusión por la
polisemia. El término \emph{argument} suele usarse más como equivalente a
\emph{discusión} en español. No obstante, en lógica, así como en las
disciplinas que hacen uso de esta, lo normal es que equivalga al
\emph{argumento} del español.}




