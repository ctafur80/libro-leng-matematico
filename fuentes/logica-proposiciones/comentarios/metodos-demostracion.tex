


% TODO No sé por qué usa letras mayúsculas ahora.

El conocimiento matemático se justifica empleando alguno de los dos métodos
de demostración: un sistema lógico deductivo y un sistema lógico inductivo.

El método deductivo consiste \emph{grosso modo} en la formación de un
enunciado verdadero $C$ partiendo de otro $H$ que, dentro del marco de una
teoría, es también verdadero. En lenguaje coloquial, $H$ es la hipótesis o
antecedente y $C$ la conclusión o consecuente.

Se tienen varios métodos.

\begin{description}
  \item[Deducción directa] Este método utiliza las leyes transitivas, o
    silogismo hipotético. Para demostrar que el antecedente es condición
    suficiente para asegurar la verdad del consecuente, se busca una
    condición intermedia tal que el antecedente sea condición suficiente de
    esta, y que a su vez esta sea condición suficiente del consecuente. Se
    basa pues en la implicación:

    $$ (P \to R) \land (R \to Q) \Longrightarrow P \to Q $$

    En muchos casos, la búsqueda de esta condición intermedia requiere
    utilizar algunas leyes como las leyes \emph{modus ponens} o la de
    \emph{modus tollens}. Como ya hemos visto, mediante  la primera de
    estas, si sabemos que el condicional $R \to H$ es verdadero, basta
    demostrar que $R$ es verdadero para deducir que $H$ es verdadero. La
    otra, \emph{modus tollens}, nos asegura que si sabemos que el
    condicional $R \to H$ es verdadero, basta demostrar que $H$ es falso
    para deducir que $R$ es falso.

  \item[Negación del consecuente] Este método utiliza la primera ley de
    transposición:

    $$ P \to Q \iff \neg Q \to \neg P $$

    Para demostrar que el antecedente es condición suficiente para que se
    verifique el consecuente, se niega el consecuente y de esta negación se
    deduce la negación del antecedente. Véase una demostración por negación
    del consecuente en el ejemplo TKTK.

  \item[Reducción al absurdo] Este método utiliza la ley de reducción al
    absurdo. Se supone verdadera la negación de lo que se quiere demostrar,
    y de esta negación se llega a una contradicción. Puede encontrar una
    demostración de este tipo en el ejercicio TKTK.

    Aunque en español se le conoce normalmente como \emph{reducción al
    absurdo}, e incluso hay quien se pone intelectual y lo dice en latín,
    \emph{reductio ad absurdum}, yo prefiero llamarlo \emph{por
    contradicción}. En inglés se suele ver más el equivalente a esta última,
    es decir, \emph{by contradiction}.
\end{description}

El método inductivo es diferente, y lo comentaremos en el capítulo siguiente
y, en el capítulo~\ref{ch:naturales}, se verá en mayor profundidad.






