




El conocimiento matemático se presenta empleando fórmulas bien formadas que
son valoradas sin ambigüedad.

% TODO Quizás, fórmulas bien formadas.

El primer elemento básico es la \emph{definición}. La forma habitual de
definir algún elemento matemático es describirlo directamente por extensión,
o indicando la propiedad o propiedades específicas. Concretamente, una
definición es una fórmula bien formada verdadera.

Ejemplo. Base de un espacio vectorial. Sea $(V, +, \cdot)$ un espacio
vectorial. Una base de este es un conjunto de vectores del mismo que forman
un sistema de generadores linealmente independientes.

En matemáticas, muchos conceptos no se definen de un modo directo, sino que
se hace a través de unas relaciones mutuas que se formulan en un sistema de
axiomas apropiado.

Como acabamos de mencionar en la axiomática de PM, un teorema es una
sentencia bien formada que es cierta, es decir, una tautología. Este
concepto puede extenderse a cualquier sistema lógico, y en definitiva a
cualquier lengua. Así pues, los teoremas son tautologías deducibles a partir
de otros teoremas, de definiciones o de axiomas en el marco de una teoría.
La secueucia de sentencias verdaderas necesarias para deducir un teorema se
denomina \emph{demostración} del teorema.

La base del conocimiento matemático está constituida por definiciones y
teoremas, en el sentido anterior. Los teoremas aparecen en matemáticas bajo
distintas denominaciones: lema, proposición, teorema o corolario. Aun siendo
estas denominaciones subjetivas y no excluyentes, una posible clasificación
sería:

\begin{description}
  \item[Teorema] Un enunciado con mucha utilidad tanto práctica como teórica
    y de uso en numerosas deducciones de nuevos teoremas. En el desarrollo
    de un tema o una teoría, el término \emph{teorema} se reserva para los
    resultados de mayor relevancia.

  \item[Proposición] Un enunciado con utilidad práctica en numerosas
    deducciones de otros nuevos teoremas o proposiciones y en general, de
    menor relevancia que un teorema en el marco de una teoría.

  \item[Lema] Un resultado intermedio en el proceso de una demostración de
    un teorema o de una proposición. En muchos casos, una demostración puede
    ser muy extensa y contener bloques de deducciones que pueden ser
    separados en lemas, facilitando el posterior proceso de comprensión de
    la demostración.

\item[Corolario] Un enunciado que se deduce con relativa facilidad del
    enunciado de un teorema. En muchos casos, los corolarios muestran
    distintas actuaciones prácticas de un teorema, y estos suelen ser de
    gran utilidad.
\end{description}

Estas distinciones son a veces arbitrarias. Por ejemplo, hay lemas, como el
Lema de Zorn, que su importancia no se corresponde con el atributo de lema,
pero ya se conoce universalmente de esta manera.

Finalmente, existen afirmaciones matemáticas que se creen verdaderas pero
que no han sido demostradas. Se deuominan \emph{conjeturas} o
\emph{hipótesis}, como la conjetura de Goldlbach, que dice: ``Todo uúmero
par mayor que 2 puede escribirse como smna de dos números primos''. Ya que
se trata de una conjetura, en la actualidad sigue sin ser demostrada.

En general, los teoremas, proposiciones, lemas y corolarios son de dos
tipos:

\begin{description}
  \item[De caracterización] Son teoremas del tipo $P \iff Q$. Al establecer
    la equivalencia de dos proposiciones, podemos sustituirlas entre sí a
    nuestro antojo allá donde aparezcan.

  \item[De condición suficiente y necesaria] Son teoremas del tipo $P
    \Longrightarrow Q$. Ya explicamos antes qué quieren decir condición
    suficiente y condición necesaria, pero preferimos recordarlo. En el caso
    de que se quieran emplear para comprobar la verdad de $P$, entonces se
    dice que las propiedades de $Q$ son condiciones necesarias. Si se
    emplean para asegurar la verdad de $Q$, entonces se dice que la
    propiedades de $P$ son condiciones suficientes.
\end{description}






