



Otra forma de introducir la lógica proposicional es mediante un
\semph{sistema axiomático}. Se establece un alfabeto de símbolos, una lista
de reglas sintácticas de formación (partículas conectivas y paréntesis), una
lista de sentencias verdaderas (axiomas) y una lista de reglas de
transformación (reglas de deducción).

% ¿Cómo hemos introducido antes la lógica proposicional, entonces?

Mostramos un sistema axiomático en particular simplemente como ejemplo, pero
no es algo que deba aprender en profundidad.

En el sistema axiomático \emph{Principia Mathematica} (PM), se dota a los
elementos del alfabeto (proposiciones) de un \emph{valor semántico} (0, 1) y
se combinan estos elementos, haciendo un uso correcto de las reglas
sintácticas de formación (únicamente $\neg$, $\lor$ y paréntesis) para
construir fórmulas bien formadas (proposiciones sintácticamente correctas),
que son valoradas sin ambigüedad. El resto de los conectores usuales se
definen mediante las leyes siguientes:

\begin{align*}
  p \land q           &\iff \neg(\neg p \lor \neg q) \\
  p \to q             &\iff \neg p \lor q \\
  p \leftrightarrow q &\iff (p \to q) \land (q \to p) \\
\end{align*}

Un \semph{axioma} es una fórmula bien formada que se considera verdadera, es
decir, se trata de una tautología, sin tener que deducirlo. También se dice
que es una tautología primaria no deducible.

Un \semph{teorema} es una fórmula bien formada que es cierta. Por tanto, se
trata también de una tautología, solo que, al contrario que con los axiomas,
en los teoremas se ha de demostrar que son ciertos. Son tautologías
deducibles a partir de otros teoremas y axiomas.

Se podría definir la \semph{demostración} (\emph{proof}) de un teorema como
la secuencia de sentencias verdaderas necesarias para deducirlo.

Los axiomas de PM son:

\begin{enumerate}[label=\textbf{A\arabic*}., leftmargin=1.5cm]
  \item $p \lor p \to p$.
  \item $p \to (p \lor q)$.
  \item $(p \lor q) \to (q \lor p)$.
  \item $(p \to q) \to \left[(r \lor p) \to (r \lor q)\right]$.
\end{enumerate}

Las reglas de transformación de PM son:

\begin{description}
  \item[Regla de sustitución] El resultado de reemplazar un elemento
    alfabético en un teorema por una fórmula bien formada es un teorema.

  \item[Regla de separación] Si $S$ y $R$ son fórmulas bien formadas, y $S$
    y $S \to R$ son teoremas, entonces $R$ es un teorema.
\end{description}











