


Primero, vamos a ver los básicos y luego iremos subiendo en complejidad.

Deberá memorizar todas las tablas de verdad de los operadores básicos que se
presentan aquí. No es complicado.

Una vez que las tenga memorizadas, para aplicarlas en los ejercicios
conviene fijarse en los resultados que aparecen con menor frecuencia en la
tabla. Esto le ahorrará trabajo.





\subsection{Negación}

Dada la proposición

\begin{center}
  $p$: El 4 es un número par.
\end{center}

\noindent la negación de esta proposición es:

\begin{center}
  $\neg p$: El 4 no es un número par.
\end{center}

\noindent y se puede designar de varias maneras: $\neg p$, ${\sim p}$, ``no
$p$'', $p'$. Se trata de un operador que actúa sobre una única proposición,
es decir, es un operador unario, aunque nada impide que esta sea compuesta.

Concretamente, en este ejemplo se tiene que $p = 1$ (verdad), mientras que
$\neg p = 0$ (falso).

En general, la negación de una proposición $p$ es otra proposición $\neg p$
que es cierta si $p$ es falsa, y falsa si $p$ es cierta. La
tabla~\ref{tbl:operador-negacion} indica los valores de $\neg p$ en función
del valor de $p$, es decir, es la tabla de verdad de este operador.

\begin{table}[h]%
  \caption{Tabla de verdad del operador negación}
  \label{tbl:operador-negacion}%
  \centering
  $$
    \begin{array}{|c||c|}
      \hline
      p & \neg p \\
      \hline
      \hline
      0 & 1 \\
      1 & 0 \\
      \hline
    \end{array}
  $$
\end{table}

En realidad, las lenguas naturales pueden dar lugar a cierta confusión, y el
operador de negación no está a salvo de esta. Para asegurarse de que obtiene
realmente una negación de una proposición dada, lo que puede hacer es
anteponer, a su enunciado, ``No es verdad que\ldots''. Esa sería la forma
infalible de obtener su negación. A partir de esta, podrá crear si lo desea
enunciados alternativos que signifiquen lo mismo. Para el ejemplo anterior
se tendría

\begin{center}
  $\neg p$: No es verdad que el 4 es un número par.
\end{center}

\noindent que veríamos fácilmente que se puede convertir en

\begin{center}
  $\neg p$: El 4 no es un número par.
\end{center}





\subsection{Disyunción}

La disyunción es una operación sobre dos o más proposiciones, de ahí que
también se la pueda calificar de \emph{conector} o \emph{conectiva}
(\emph{conective}). Las demás que aparecerán en esta sección también lo
serán. De momento, nos limitaremos a la que une dos proposiciones, pero
luego veremos esta misma operación para más.

Dadas las proposiciones

\begin{center}
\begin{tabular}{ll}
  $p$: & El 4 es un número par. \\
  $q$: & El 4 es un número impar.
\end{tabular}
\end{center}

\noindent la proposición disyunción de $p$ y $q$, expresada como ``\(p
\lor q\)'', es

\begin{center}
  $p \lor q$: El 4 es un número par o el 4 es un número impar.
\end{center}

\noindent que se podría decir de una forma más breve y elegante como

\begin{center}
  $p \lor q$: El 4 es un número par o impar.
\end{center}

\noindent Se representa en notación simbólica como $p \lor q$, $p + q$, $p
\cup q$, etc.

Concretamente, en este ejemplo, ``el 4 es un número par o impar'', se tiene
que $p = 1$ y $q = 0$, y la disyunción es la proposición compuesta $p \lor q
= 1$.

En general, la proposición disyunción $p \lor q$ es verdadera si al menos
una de las dos proposiciones es verdadera. La
tabla~\ref{tbl:operador-disyuncion} representa esta operación. Como puede
comprobar en esa tabla, la proposición $p \lor q$ es falsa únicamente si $p$
y $q$ son falsas.

\begin{table}[h]%
  \caption{Tabla de verdad del operador disyunción}
  \label{tbl:operador-disyuncion}%
  \centering
  $$
    \begin{array}{|c|c||c|}
      \hline
      p & q & p \lor q \\
      \hline
      \hline
      0 & 0 & 0 \\
      0 & 1 & 1 \\
      1 & 0 & 1 \\
      1 & 1 & 1 \\
      \hline
    \end{array}
  $$
\end{table}

En español, lengua natural, la disyunción \emph{o} puede aparecer en dos
variantes pero representadas por el mismo símbolo. Veámoslo mediante dos
ejemplos. En el enunciado

\begin{quote}
  El medicamento está indicado para el dolor de cabeza o la fiebre.
\end{quote}

\noindent se indica que se debe tomar el medicamento si se cumple al menos
uno de los dos requisitos (tener dolor de cabeza o tener fiebre), pudiendo
darse que se tengan ambos síntomas simultáneamente. Se trataría de una
disyunción inclusiva. Sin embargo, en

\begin{quote}
  Compraré el regalo hoy o mañana.
\end{quote}

\noindent parece que el \emph{o} es excluyente ya que, si compro el regalo
hoy, no lo compraré mañana.

El significado de operador lógico disyunción, `$\lor$' es como en el primero
de estos casos, es decir, se trata de lo que suelen llamar una \emph{o}
inclusiva. La del segundo ejemplo sería una \emph{o} exclusiva. También
veremos más adelante las de este último tipo, pero tienen menos importancia
que las anteriores.

En las lenguas naturales, al contrario que en la lógica, la variedad de
disyunción no será siempre la misma y la única forma de saber de cuál se
trata será deducirlo de la información contextual. Este sería un ejemplo de
la ambigüedad que existe en las lenguas naturales.





\subsection{Conjunción}

Dadas las proposiciones

\begin{center}
\begin{tabular}{l l}
  $p$: el 4 es un número par. \\
  $q$: el 9 es un número impar.
\end{tabular}
\end{center}

\noindent la proposición conjunción de $p$ y $q$, que se suele designar por
``$p \land q$'', es:

\begin{center}
  $p \land q$: El 4 es un número par y el 9 es un número impar.
\end{center}

\noindent y se representa con alguna de las expresiones $p \land q$, $p
\cdot q$, $p \cap q$, etc.

En este caso, $p = 1$, $q = 1$ y $p \land q = 1$.

En la tabla~\ref{tbl:operador-conjuncion} se muestra su comportamiento. La
proposición $p \land q$ es verdadera solo si $p$ y $q$ son verdaderas y es
falsa si al menos una de las dos proposiciones es falsa.

\begin{table}[h]%
  \caption{Tabla de verdad del operador conjunción}
  \label{tbl:operador-conjuncion}%
  \centering
  $$
    \begin{array}{|c|c||c|}
      \hline
      p & q & p \land q \\
      \hline
      \hline
      0 & 0 & 0 \\
      0 & 1 & 0 \\
      1 & 0 & 0 \\
      1 & 1 & 1 \\
      \hline
    \end{array}
  $$
\end{table}






\subsection{Condicional}

Dadas las proposiciones

\begin{center}
\begin{tabular}{l l}
  $p$: El 8 es un número par. \\
  $q$: El 8 es suma de dos números iguales.
\end{tabular}
\end{center}

\noindent la proposición condicional ``si $p$, entonces $q$'' es

\begin{center}
  $p \to q$: Si el 8 es un número par, entonces es suma de dos
  números iguales.
\end{center}

\noindent y se representa con alguna de las notaciones siguientes: $p \to
q$, $p \Rightarrow q$, $p \implies q$, etc. 

En este caso, $p = 1$, $q = 1$ y $p \to q = 1$.

La tabla~\ref{tbl:operador-condicional} muestra el comportamiento de este
operador. La proposición $p \to q$ es falsa únicamente si $p$ es verdadera y
$q$ es falsa.

\begin{table}[h]%
  \caption{Tabla de verdad del operador condicional}
  \label{tbl:operador-condicional}%
  \centering
  $$
    \begin{array}{|c|c||c|}
      \hline
      p & q & p \to q \\
      \hline
      \hline
      0 & 0 & 1 \\
      0 & 1 & 1 \\
      1 & 0 & 0 \\
      1 & 1 & 1 \\
      \hline
    \end{array}
  $$
\end{table}

De la proposición $p \to q$, se suele decir que $p$ es el \emph{antecedente}
y $q$ el \emph{consecuente}.

Además, si la primera proposición es falsa, entonces la proposición
condicional es verdadera. Esto suele indicarse coloquialmente diciendo que
de un antecedente falso se deduce cualquier cosa, o que una proposición
falsa implica cualquier otra. Se puede usar esto en muchas demostraciones,
tal y como verá. Puede parecer algo extraño, pero es lo que más sentido
tiene. En muchos momentos puede que le parezca un tecnicismo. Así, por
ejemplo, se pueden afirmar cosas como que TKTK.

Otra terminología que verá muy a menudo en matemáticas es la de ``condición
necesaria'' y ``condición suficiente'', que aparecen cuando se tiene un
condicional. Si se tienen dos proposiciones $p$ y $q$ y una proposición
compuesta $p \to q$, se tendrá que $p$ es una \emph{condición suficiente}
para que se dé $q$ y, por otro lado, que $q$ es una \emph{condición
necesaria} para que se dé $p$.

A este respecto, creo que es fácil comprender por qué se da la condición de
suficiencia, pero quizás no comprenda la otra, la de necesariedad. Puede
pensarlo fijándose en cómo varían los valores de $p$ en función de los de
$q$ en la tabla de verdad del operador condicional. En cualquier caso, si no
lo comprende, no se preocupe. Después veremos una herramienta que le servirá
para comprenderlo de forma muy sencilla. Cuando vea esa herramienta,
comprenderá por qué dije que la lógica permite sistematizar el razonamiento.
Es como si le permitiera razonar en ``piloto automático''.






\subsection{Bicondicional}

Dadas las proposiciones

\begin{center}
\begin{tabular}{l l}
  $p$: El 8 es un número par. \\
  $q$: El 8 es divisible por 2.
\end{tabular}
\end{center}

\noindent la proposición bicondicional ``$p$ si y solo si $q$'' es

\begin{center}
  $p \leftrightarrow q$: El 8 es un número par si y solo si también es
  divisible por 2.
\end{center}

\noindent y se puede representar simbólicamente de diversas formas: $p
\longleftrightarrow q$, $p \leftrightarrow q$, $p \Leftrightarrow q$, $p
\iff q$, etc.

En este ejemplo, $p = 1$, $q = 1$ y $p \leftrightarrow q = 1$.

\begin{table}[h]%
  \caption{Tabla de verdad del operador bicondicional}
  \label{tbl:operador-bicondicional}%
  \centering
  $$
    \begin{array}{|c|c||c|}
      \hline
      p & q & p \leftrightarrow q \\
      \hline
      \hline
      0 & 0 & 1 \\
      0 & 1 & 0 \\
      1 & 0 & 0 \\
      1 & 1 & 1 \\
      \hline
    \end{array}
  $$
\end{table}

Su comportamiento se muestra en la tabla~\ref{tbl:operador-bicondicional}.
La proposición $p \leftrightarrow q$ es verdadera solo si $p$ y $q$ toman el
mismo valor. A la vista de esta tabla, es evidente que el bicondicional
sería algo así como una igualdad, solo que en la lógica no existe la
igualdad, en principio TKTK. Si se tiene una proposición con un
bicondicional y es, en general, verdadera, las proposiciones que la
constituyen se dice que son \emph{equivalentes}.

Cuando se sabe que una proposición con un bicondicional es verdadera, tan
solo hay que estudiar si alguna de las proposiciones es verdadera, para
concluir que la otra también es verdadera. O que es falsa para demostrar que
la otra también lo es. Esto es algo que se puede usar para hacer
demostraciones, y, de hecho, se usa bastante; por ejemplo, en combinatoria.
Es decir, esto nos permite, en muchos casos, demostrar algo indirectamente
simplemente demostrando otra cosa que nos parezca más sencilla.

Otras formas frecuentes de expresar esta equivalencia entre proposiciones en
la literatura matemática son: ``$p$ si y solo si $q$'', que se resume en la
expresión ``$p$ ssi $q$'' (``$p$ iff $q$'').

Ejemplo. Dentro del contexto matemático, podemos encontrar proposiciones con
los operadores anteriores.

\begin{center}
  La función $f(x) = 1/x$ no está definida para $x = 0$.
\end{center}

Se trata de una proposición de negación verdadera, $\neg p$, donde la
proposición $p$ es:

\begin{center}
\begin{tabular}{ll}
  $p$: & La función $f(x) = 1/x$ está definida para $x = 0$.
\end{tabular}
\end{center}

\noindent que es falsa.

El punto $(1,1)$ está contenido en la región del plano $x^2 + y^2 \leq 4$.
Se puede ver como una proposición de disyunción, $p \lor q$, donde

\begin{center}
\begin{tabular}{ll}
  $p$: & El punto \((1,1)\) está contenido en la región del plano \(x^2 +
    y^2 < 4\). \\
  $q$: & El punto $(1,1)$ está contenido en la región del plano $x^2 + y^2 =
    4$. \\
\end{tabular}
\end{center}

\noindent Se tiene que $p = 1$, $q = 0$ y $p \lor q = 1$.

La función $f(x) = x^2$ es continua en $[0,1]$ y derivable en $(0,1)$. Se
puede ver como una proposición de conjunción verdadera ($p \land q$), donde
las proposiciones

\begin{center}
\begin{tabular}{ll}
  $p$: & La función $f(x) = x^2$ es continua en $[0,1]$. \\
  $q$: & La función $f(x) = x^2$ es derivable en $(0,1)$.
\end{tabular}
\end{center}

\noindent son ambas verdaderas.






\subsection{Notación `$\rightarrow$' contra `$\Rightarrow$'}%
\label{subsec:notacion-flecha}

Veamos una particularidad sobre la notación. En contexto matemático,
usualmente solo se escriben proposiciones que sean verdaderas. Es decir,
cuando aparece en un texto de matemáticas una proposición como, por ejemplo,
$p \rightarrow q$ o $p \longrightarrow q$, se entiende que se está afirmando
que esa proposición es verdadera en ese caso. Por nuestra parte, vamos a ser
más rigurosos y emplearemos notaciones distintas para cuando se desea
expresar esto último o si simplemente deseamos expresar alguna idea sobre
esa proposición condicional. Concretamente, con algo como

\begin{center}
  $p \to q$
\end{center}

\noindent o, si lo prefiere, $p \longrightarrow q$, se expresará simplemente
 esa proposición, sin afirmar si es verdadera o si no lo es. Por el
 contrario, si escribimos

\begin{center}
  $p \implies q$
\end{center}

\noindent o, si lo prefiere, $p \Rightarrow q$, estaremos afirmando que
dicha proposición condicional es verdadera. Haremos lo mismo con la
simbología del bicondicional: `$\leftrightarrow$' y `$\longleftrightarrow$'
contra `$\Rightarrow$' y `$\Longleftrightarrow$'.

Ejemplo 1.4. Dentro del contexto matemático, podemos encontrar proposiciones
con conectores condicionales como:

\begin{center}
  Al ser $f(x) = 3x^3 + 2x^2 + x$ una función derivable en $\mathbb{R}$,
  entonces $f(x)$ es continua en todo $\mathbb{R}$.
\end{center}

Se trata de una proposición condicional verdadera, $p \rightarrow q$, donde
la proposición

\begin{center}
\begin{tabular}{ll}
  $p$: & La función $f(x) = 3x^3 + 2x^2 + x$ es derivable en $\mathbb{R}$.
\end{tabular}
\end{center}

\noindent es verdadera y la proposición

\begin{center}
\begin{tabular}{ll}
  $q$: & La función $f(x) = 3x^3 + 2x^2 + x$ es continua en $\mathbb{R}$.
\end{tabular}
\end{center}

\noindent también es verdadera.

En este caso decimos que la derivabilidad de la función $f(x) = 3x^3 + 2x^2
+ x$ en $ \mathbb{R} $ implica la continuidad de esta en todo $\mathbb{R}$.

Veamos otro ejemplo. Tenemos

\begin{center}
  La dimensión de $\mathbb{R}^2$ es dos si y solo si el conjunto $\{(1, 0),
  (0, 1)\}$ constituye una base de $\mathbb{R}^2$.
\end{center}

\noindent Se trata de una proposición bicondicional verdadera, $p
\leftrightarrow q$, donde la proposición

\begin{center}
\begin{tabular}{ll}
  $p$: & La dimensión de $\mathbb{R}^2$ es dos.
\end{tabular}
\end{center}

\noindent es verdadera y la proposición

\begin{center}
\begin{tabular}{ll}
  $q$: & El conjunto $\{(1, 0), (0, 1)\}$ es una base de $\mathbb{R}^2$.
\end{tabular}
\end{center}

\noindent también es verdadera.

Si nos fijamos en la proposición condicional $p \to q$, podemos tener otras
proposiciones condicionales que tomarán ciertas designaciones respecto a
esta:

\begin{center}
\begin{tabular}{l|l}
  Proposición & Designación respecto a $p \to q$ \\
  \hline
  $q \rightarrow p$ & su condicional recíproco \\
  $\neg p \rightarrow \neg q$ & su condicional contrario \\
  $\neg q \rightarrow \neg p$ & su condicional contrarrecíproco \\
\end{tabular}
\end{center}

\noindent De estos, el más importante es el contrarrecíproco, y se usa en
muchas demostraciones. También hay quien lo llama \emph{contrapositivo}
(\emph{contrapositive}). Se volverá a ver en este capítulo.





\subsection{Operadores que actúan sobre una proposición}

¿Cuántos operadores que actúen sobre una única proposición se pueden
definir? Hay tantos como tablas de verdad distintas se puedan construir con
una única proposición $p$. Véanse en la tabla~\ref{tbl:operadores-1-prop}
todas estas tablas y los correspondientes operadores unarios que hemos
representado con los símbolos $C_0$, $C_1$, $C_2$ y $C_3$, que se
corresponden con las expresiones de los números del 0 al 3 en notación
binaria: 00, 01, 10 y 11.

\begin{table}[h]%
  \caption{Operadores sobre una sola proposición}
  \label{tbl:operadores-1-prop}%
  \centering
  $$
    \begin{array}{|c||c|c|c|c|}
      \hline
      p & C_0 p & C_1 p & C_2 p & C_3 p \\
      \hline
      \hline
      0 & 0 & 0 & 1 & 1 \\
      1 & 0 & 1 & 0 & 1 \\
      \hline
    \end{array}
  $$
\end{table}

El operador $C_1$ es el conector identidad, es decir, $C_1 p \iff p$,
mientras que la $C_2$ es la negación, es decir, $C_2 p \iff \neg p$.




\subsection{Operadores que actúan sobre dos proposiciones}

¿Cuántas operaciones que actúen sobre dos proposiciones se pueden definir?
Si se escriben todas las posibles tablas de verdad para dos proposiciones
$p$ y $q$, se comprueba que hay 16 tablas distintas, que presentamos en la
tabla~\ref{tbl:operadores-2-prop}. Por tanto, se pueden definir 16
operadores distintos, uno por cada una de las tablas, y los representamos
por esos símbolos: $C_0, C_1, \ldots, C_{15}$, que se corresponden con los
números del 0 al 15 en notación binaria: $0000, 0001, 0010, \ldots, 1111$.

\begin{table}[h]%
  \caption{Operadores sobre dos proposiciones}
  \label{tbl:operadores-2-prop}%
  \centering
  $$
    \begin{array}{|c|c||c|c|c|c|c|c|c|c|}
    \hline
    p & q & p C_0 q & p C_1 q & p C_2 q & p C_3 q & p C_4 q & p C_5 q & p
      C_6 q & p C_7 q \\
    \hline
    \hline
    0 & 0 & 0 & 0 & 0 & 0 & 0 & 0 & 0 & 0 \\
    0 & 1 & 0 & 0 & 0 & 0 & 1 & 1 & 1 & 1 \\
    1 & 0 & 0 & 0 & 1 & 1 & 0 & 0 & 1 & 1 \\
    1 & 1 & 0 & 1 & 0 & 1 & 0 & 1 & 0 & 1 \\
    \hline
    p & q & p C_8 q & p C_9 q & p C_{10} q & p C_{11} q & p C_{12} q & p
      C_{13} q & p C_{14} q & p C_{15} q \\
    \hline
    \hline
    0 & 0 & 1 & 1 & 1 & 1 & 1 & 1 & 1 & 1 \\
    0 & 1 & 0 & 0 & 0 & 0 & 1 & 1 & 1 & 1 \\
    1 & 0 & 0 & 0 & 1 & 1 & 0 & 0 & 1 & 1 \\
    1 & 1 & 0 & 1 & 0 & 1 & 0 & 1 & 0 & 1 \\
    \hline
    \end{array}
  $$
\end{table}

Observemos que a algunos de estos operadores ya los conocemos. Por ejemplo,
$C_1$ es el operador conjunción, $p \land q$, $C_7$ el disyunción, $p \lor
q$, $C_9$ el bicondicional, $p \longleftrightarrow q$, y $C_{13}$ el
condicional, $p \longrightarrow q$.

Ejemplo. La disyunción excluyente. En la tabla anterior, al operador $C_6$
se le denomina \emph{disyunción excluyente}, y sería como la otra variedad
de la disyunción \emph{o} en español que se mencionó antes. Si recuerda, se
usaba en la proposición

\begin{quote}
  Compraré el regalo hoy o mañana.
\end{quote}

\noindent Se trataría de una proposición compuesta por las proposiciones
simples siguientes:

\begin{tabular}{ll}
  $p$: & Compraré el regalo hoy. \\
  $q$: & Compraré el regalo mañana. \\
\end{tabular}

Se suele denotar por el símbolo `$\oplus$'. En este caso, se tendría que la
proposición $p \oplus q$ es cierta.




