



Dada una relación homogénea $\rrel \subseteq U \times U$, definimos las
propiedades siguientes:

1. Propiedad. Reflexiva. Cuando

\[ \forall x \in U. \ x \rrel x \]

\noindent Dicho de otra forma, si

\[ \{(x, x) \st x \in U\} \subseteq \rrel \]

2. Propiedad. Simétrica. Cuando

$$ \forall x, y \in U. \ \text{Si} \ x \rrel y, \ \text{entonces} \ y \rrel
x $$

\noindent Se puede definir también como

\[ \rrel^{-1} \subseteq \rrel \]

La justificación de este hecho se podría deducir de la definición anterior.
Además de esa expresión, sería evidente que para cualquier $(x, y) \in
\rrel$ se tiene que $(y, x) \in \rrel^{-1}$. Así, uniendo ambas, se tiene

\[ \rrel^{-1} \subseteq \rrel \]

% TODO

No lo tengo del todo claro. Quizás esto requeriría que se diera la propiedad
transitiva. TKTK. ¿No sería al revés? $\rrel \subseteq \rrel^{-1}$.

3. Propiedad. Antisimétrica. Cuando

\[ \forall x, y \in U \ \text{Si} \ x \rrel y \ \text{e} \ y \rrel x, \
\text{entonces} \ x = y \]

\noindent Dicho de otro modo,

\[ \rrel^{1} \cap \rrel \subseteq \{(x, x) \st x \in U\} \]

4. Propiedad. Transitiva. Cuando

\[ \forall x, y, z \in U \ \text{se cumple que, si} \ x \rrel y \ \text{e} \
y \rrel z, \ \text{entonces} \ x \rrel z \]

\noindent También se puede definir del modo siguiente

\[ \rrel \circ \rrel \subseteq \rrel \]

Advierta que, aunque diga ``cuando'' o ``si'', para estas definiciones, en
realidad sería ``si y solo si''. En estos casos tampoco hay que ser tan
exacto en el lenguaje, pues sería un engorro TKTK. Se sobrentiende que en
las definiciones se está usando en realidad una bicondicional, aunque no se
haga realmente.

En realidad, se podría haber definido sobre relaciones en el sentido amplio,
es decir, no solo sobre las homogéneas, pero tampoco nos interesa en esta
asignatura. Ya que nos centramos en las homogéneas, vamos a definir más
adelante otros términos centrándonos en estas.

Observaciones. La relación del ejemplo TKTK no es reflexiva. Para que una
relación en $\rset$ sea reflexiva, la representación del grafo de la
relación debe contener a la diagonal, $y = x$.

La relación del ejemplo TKTK es simétrica, pero no la relación del ejemplo
TKTK. Para que una relación en $\rrel$ sea simétrica, la representación del
grafo de la relación debe ser simétrica respecto a la recta diagonal del
primer cuadrante.












