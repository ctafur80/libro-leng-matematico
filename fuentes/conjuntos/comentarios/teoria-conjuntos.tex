



La teoría de conjuntos actual, que fue desarrollada en su inicio por Georg
Cantor en el siglo XIX, constituye el fundamento de las matemáticas. El
propósito de Cantor era tratar cuestiones relacionadas con el infinito, y su
método allanaba dificultades. Para Cantor, un conjunto es una reunión de
objetos determinados y bien diferenciados de nuestra intuición o nuestro
pensamiento, formando una totalidad. Cantor trataba una colección o conjunto
de objetos como un todo, aceptando implícitamente lo siguiente:

\begin{enumerate}
  \item Un conjunto es una colección de elementos que cumplen cierta
    propiedad. Por tanto, queda definido por dicha propiedad. Es decir, tal
    y como lo hemos definido aquí por comprensión, mediante un predicado.
  \item Un conjunto es una sola entidad matemática, de modo que puede a su
    vez ser contenido por otro conjunto.
  \item Dos conjuntos que tengan los mismos elementos son iguales. Un
    conjunto está determinado por sus elementos.
\end{enumerate}

Esta teoría tuvo éxito, pero necesitó ser precisada por otros matemáticos
como Gottlob Frege, Bertrand Russell, E.~Zermelo, A.~Skolem y A.~Fraenkel.
Después de varios intentos de axiomatización, teoría de Frege, teoría de
Russell-Whitehead (PM) y otras, se destacan dos sistemas axiomáticos de la
teoría de conjuntos: la teoría de conjuntos de Zermelo-Fraenkel (ZF),
desarrollada por Zermelo, Skolem y Fraenkel, y la teoría de conjuntos de Von
Newman-Gödel, desarrollada por John von Newman, Bernays y Kurt Gödel.

Tal y como explicamos al comienzo del capítulo, el concepto de
\emph{conjunto} se encuentra a un nivel tan elemental que no es posible dar
una definición precisa del mismo. La utilización de palabras como
\emph{colección}, \emph{familia}, \emph{reunión}, etc. en un intento de
definirlo, no hacen nada más que emplear el objeto a definir dentro de la
definición, puesto que esas palabras son sinónimos de la palabra
\emph{conjunto}.

Es claro que el lenguaje natural es necesario para describir los objetos
matemáticos y que este posee cierto nivel de ambigüedad, pero las
definiciones matemáticas deben quedar exentas de ambigüedad aunque se
formulen con un lenguaje natural.

En la teoría ``ingenua'' de conjuntos, se admite el uso de esas palabras, y
se acepta la existencia de un universo de objetos, sin importar la
naturaleza de los objetos. A partir de ese universo se construyen los
conjuntos como entidad matemática. Un elemento posterior es introducir la
relación de pertenencia de elementos a conjuntos, $\in$. Al definir
\emph{conjunto} a partir de una propiedad determinada que deben cumplir sus
elementos, se producen ciertas paradojas como la de Bertrand Russell, y
aparecen ``conjuntos enormes'' que producen cierto desasosiego intuitivo y
lógico.

Dificultades como estas introducen la necesidad de axiomatizar y formalizar
la teoría de conjuntos para poder obtener resultados profundos. Se renuncia
a una definición intuitiva de \emph{conjunto}, y se establecen una serie de
principios (axiomas) que describen el comportamiento de dicho concepto.
Cualquier resultado obtenido debe ser consecuencia de tales principios.

A continuación exponernos una de las axiomáticas de conjuntos más utilizadas
con el espíritu de que el lector se dé cuenta de la dificultad que tiene el
formalizar una teoría. No se trata de que memorice los axiomas; ni siquiera,
de que comprenda los enunciados de los mismos. Simplemente queremos que vea
que establecer un lenguaje sin ambigüedad precisa un esfuerzo enorme, y que
incluso solo comprenderlo requiere de una sólida formación matemática.

La teoría de conjuntos de ZF establece el concepto de \emph{conjunto} como
elemento primitivo, al igual que la relación de pertenencia. Dispone de los
axiomas que mostramos a continuación. No tiene que aprendérselos; se
muestran simplemente con el propósito de TKTK.

\begin{enumerate}
  \item Axioma de extensión. Dos conjuntos $A$ y $B$ son iguales si
    contienen los mismos elementos. Es decir, 

    \[ \forall x.\ (x \in A \leftrightarrow x \in B) \implies A = B \]

  \item Axioma del conjunto vacío. Existe un conjunto sin elementos. Es
    decir,

    \[ \exists \emptyset \ \forall x. \ x \notin \emptyset \]

  \item Axioma de pares. Dados dos conjuntos cualesquiera $A$ y $B$, existe
    otro conjunto cuyos elementos son únicamente $A$ y $B$, $\{A, B\}$. Es
    decir,

    \[ \forall A, B \ \exists C \ \forall x. \ [x \in C \leftrightarrow (x =
    A \lor x = B)] \]

  \item Axioma de la unión. Dado cualquier conjunto de conjuntos, $C$,
    existe un conjunto, que denominamos ``unión de $C$'' y denotamos por
    $\cup C$, que contiene a todos los elementos de cada conjunto de $C$. Es
    decir,

    \[ \forall C \ \exists {\cup C} \ \forall x. \ [x \in {\cup C}
    \leftrightarrow \exists A. \ (A \in C \land x \in A)] \]

  \item Axioma del conjunto potencia. Para cualquier conjunto $A$, existe
    otro conjunto, llamado ``conjunto potencia de $A$'' y que denotamos por
    $\powset(A)$, que contiene todos los subconjuntos de $A$. Es decir,

    \[ \forall A \ \exists \powset(A) \ \forall B.\ [B \in \powset(A)
    \leftrightarrow \forall x.\ (x \in B \implies x \in A)] \]

  \item Axioma de especificación. Sea $\phi(t)$ una fórmula de un lenguaje
    de primer orden que contenga una variable libre $t$. Entonces, para
    cualquier conjunto $A$ existe un conjunto $B$ cuyos elementos son
    aquellos elementos $x$ de $A$ que cumplen $\phi(x)$. Es decir,

    \[ \forall A \ \exists B \ \forall x.\ [x \in B \leftrightarrow (x \in A
    \land \phi(x))] \]

  \item Axioma de sustitución. Si $\phi(x, y)$ es una sentencia tal que para
    cualquier elemento $x$ de un conjunto $A$ existe el conjunto $B = \{y
    \st \phi(x, y)\}$, entonces existe una función $f: A \to B$ tal que
    $f(A) = B$.

  \item Axioma de infinitud. Existe un conjunto $A$ tal que $\emptyset \in
    A$ y tal que si $x \in A$, entonces $x \cup \{x\} \in A$. Es decir,

    \[ \exists A.\ [\emptyset \in A \land (\forall x \in A \ x \cup \{x\}
    \in A)] \]

  \item Axioma de regularidad. Para todo conjunto no vacío $A$, existe un
    conjunto $B$ que es elemento de $A$ tal que $A \cap B = \emptyset$. Es
    decir,

    \[ \forall A \neq \emptyset \ \exists B.\ (B \in A \land \forall x[x \in
    B \implies x \notin A]) \]
\end{enumerate}




\subsubsection{Paradojas}

Finalmente, señalamos algunas de las paradojas que hemos citado y que
motivaron el establecimiento de estos axiomas como la teoría de conjuntos de
ZF.

% TODO Este no lo entiendo. No sé si se explica mejor en el capítulo de los
% números naturales.

\paragraph{Paradoja de Cantor} Sea $C$ la colección de todos los conjuntos
posibles. Por un lado, si $C$ es un conjunto, se cumple que $C \in
\powset(C)$. Por el otro, como cualquier subconjunto $A \subseteq C$ también
es un conjunto, $A$ será un elemento de $C$, es decir, $A \in C$. Al darse
esto para todos los subconjuntos de $C$, es decir, para $\powset(C)$,
resulta que $\powset(C) \subseteq C$. Esto es una contradicción, tal y como
se verá en el capítulo~\ref{ch:naturales}.

Obsérvese que también se deduce que $C \in C$, que está en contradicción con
una de las reglas básicas de las que hemos partido. Por tanto, el concepto
de ``conjunto de todos los conjuntos'' conduce a una paradoja.

\paragraph{Paradoja de Russell} Sea $M$ la colección de todos los conjuntos
que no son elementos de sí mismos. es decir:

\[ M = \{X \st X \notin X\} \]

Si $M$ fuera un conjunto, la pregunta que se plantea es: ¿Es $M$ elemento de
sí mismo? Si $M$ es elemento de $M$, entonces $M \in M$ por definición de
$M$. Por otro lado, si $M$ no es elemento ele $M$, entonces $M \in M$, por
definición de $M$. En ambos casos llegamos a una contradicción.

La paradoja de Russell es análoga a una paradoja más popular que se denomina
paradoja del barbero que más o menos dice así: En un pueblo, hay un único
barbero que afeita a todos los que no se afeitan a sí mismos. ¿Quién afeita
al barbero?




