

En el ejemplo TKTK vimos el Principio de Inducción. Este proporciona un
método para establecer que un predicado $P_n$ en el que interviene una
variable $n \in \nset$ es verdadero para todo $n$. Es decir, si se quiere
demostrar que la proposición $\forall n \in \nset.\ P_n$ es verdadera,
basta comprobar los dos puntos siguientes:

\begin{itemize}
  \item Para $n = 0$, la proposición $P_0$ es verdadera.

  \item Para todo $n$, si la proposición $P_n$ es verdadera, entonces la
    proposición $P_{n+1}$ es verdadera.
\end{itemize}

La utilización de este principio permite también construir una sucesión de
elementos de un conjunto $A$ cuando se dispone de una manera para formar el
término $a_n$ en función de términos anteriores. Este tipo de sucesiones se
denominan \emph{sucesiones recurrentes}.

\begin{example}[Algunas Sucesiones]
  Una sucesión recurrente famosa es la \emph{sucesión de Fibonacci}:

  \[ 0, 1, 1, 2, 3, 5, 8, 13, 21, \dots \]

  \noindent En esta, cada término es la suma de los dos anteriores. Es
  evidente que para que esta definición conduzca a una única sucesión, deben
  conocerse los dos primeros términos, que en este caso son $0$ y $1$.

  Otro caso particular de sucesión recurrente es la \emph{progresión
  aritmética} de diferencia $d$. Su definición recurrente es

  \[ x_n = x_{n-1} + d \]

  De la definición se deduce directamente que

  \[ x_n = x_1 + d(n-1) \]

  % TODO ¿Cómo se deduce eso directamente? Creo que yo lo he demostrado
  % alguna vez por el principio de inducción. De hecho, tiene sentido que
  % sea ese el método. Hacer la demostración aquí.

  Para determinar la sucesión hay que conocer el primer término, $x_1$.

  Otro, la \emph{progresión geométrica} de razón $r$. Su definición
  recurrente es

  \[ x_n = r x_{n-1} \]

  De la definición se deduce directamente que

  \[ x_n = x_1 r^{n-1} \]

  Al igual que con la aritmética, para determinar la sucesión hay que
  conocer el primer término, $x_1$.
\end{example}

Sea $a \in \nset$. En ocasiones hay que demostrar que una determinada
propiedad $P_n$, que no es verdadera para todo $n \in \nset$, sí lo es si $n
\geq a$. En este caso, se cambia el primer punto en la demostración por
inducción, teniendo que comprobar que la proposición $P_a$ es verdadera y,
además,

\[ \forall n \in \nset \text{ tal que } n \geq a.\ (P_n \implies P_{n+1}) \]

\noindent para concluir que la proposición $P_n$ es verdadera para $n \geq
a$.

Lo anterior se aplica a menudo cuando se demuestran predicados donde la
variable se restringe a $\nset^*$, porque por ejemplo la
proposición $P_0$ no tenga sentido. Se empieza pues probando que $P$
es verdadera para $n = 1$.

Ocurre a veces que, para establecer un predicado con variable en $\nset$, el
suponer que $P_n$ es cierto no basta para demostrar la validez de $P_{n+1}$.
En cambio, sí se demuestra la validez de $P_{n+1}$ si se supone cierta $P_k$
para todo $k \leq n$. La conclusión es la misma. En este caso la inducción
se denomina \textbf{inducción completa}:

\begin{itemize}
  \item La proposición $P_0$ es verdadera.
  \item $\forall n \in \nset, \text{ si } P_k \text{ es verdadera para todo
    } k \text{ tal que } 0 \leq k \leq n, \text{ entonces } P_{n+1} \text{
      es verdadera}$.
\end{itemize}

Entonces la proposición $P_n$ es verdadera para $n \in \nset$.













