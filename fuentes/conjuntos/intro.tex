



La que considero que es la mejor referencia para introducirse en la teoría
de conjuntos es la primera parte de \cite{open-logic}.

Hoy en día, prácticamente todos los conceptos matemáticos se definen
formalmente en términos conjuntistas. Por ejemplo, las propiedades de los
números naturales o las de los reales se deducen dentro de un marco de
teoría de conjuntos. Las relaciones de orden y de equivalencia, que forman
parte de la teoría de conjuntos, son ubicuas en todos los campos de las
matemáticas.

Aquí, introducimos los conjuntos de una manera intuitiva, es decir, no
mediante un sistema axiomático, que sería lo correcto pero poco práctico.
Dentro de las operaciones que existen sobre conjuntos, nos centramos en la
unión, la intersección, la diferencia y el complementario.

Tal y como veremos en este capítulo, existe un nexo entre la lógica y la
teoría de conjuntos. La lógica de proposiciones no es lo suficiente
expresiva para el conocimiento matemático. Así, existen expresiones como

\begin{quote}
  Los números naturales son pares o son impares.
\end{quote}

\noindent que no podríamos expresar con el grado de detalle que desearíamos.
Esta expresión en concreto no está referida a un objeto en particular, sino
a toda una colección de objetos; el conjunto de los números naturales.

Desearíamos poder expresarla mediante una proposición universal, que se
cumpla para todos los números naturales. Para lograr esto, se introduce un
sistema lógico conocido como \emph{lógica de predicados} (o \emph{lógica de
primer orden}), en el que expresiones como la anterior puedan ser
representadas de forma simbólica con el grado de detalle deseado. Al igual
que en la lógica proposicional, se tratan expresiones de las cuales tan solo
interesa su valor de verdad, sin importar otros significados que puedan
tener en un lenguaje natural.

La lógica proposicional y la de predicados tienen mucho en común. Tal y como
he mencionado, esta última es como una generalización o abstracción de la
anterior, con el fin de tener un lenguaje más rico o expresivo que nos sirva
para las matemáticas. También se trata de una lógica formal simbólica, al
descender de esta.




