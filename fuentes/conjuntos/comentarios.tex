



\iffalse Tal y como hemos explicado, no vamos a tratar la teoría de
conjuntos mediante un sistema axiomático formal, que es como se trata de
forma rigurosa. Esto se deja para cursos más avanzados dedicados a esa
materia en particular. Aun así, el comienzo del estudio de la teoría de
conjuntos sirvió en su momento para establacer una base sólida de ciertas
áreas de las matemáticas en las que se había avanzado mucho, como el cálculo
infinitesimal. Se presentarán aquí algunas sutilezas del concepto de
conjunto y, en las secciones de comentarios, al final de cada capítulo, se
darán ciertaas nociones sobre el estudio profundo de la teoría de conjuntos.
A este respecto, de momento, basta con que sea consciente de que hay ciertas
situaciones que debemos evitar, a la hora de definir un conjunto. Por
ejemplo, una entidad matemática no puede ser simultáneamente un conjunto y
un elemento de ese mismo conjunto. Ojo, esto no impide que una entidad
matemática pueda ser un conjunto y un elemento de \emph{otro} conjunto.

Si se diera la situación que hemos prohibido, aplicando la lógica se podría
llegar a ciertas paradojas, tal y como demostró Bertrand Russell. Al hacer
esa prohibición, se tendrá que el conjunto de todos los conjuntos no existe.
Podríamos considerar al conjunto de todos los conjuntos como una especie de
agrupación pero no lo podríamos calificar de conjunto.
\fi


