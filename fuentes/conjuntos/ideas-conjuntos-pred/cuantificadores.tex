


Volvamos al predicado $P_x$: ``El número elegido es par'', donde el universo
del predicado es $\nset$. El valor de verdad de $P_x$ varía en relación al
valor que tome $x$. Sin embargo, las expresiones

\begin{quote}
  Todos los números naturales son pares. \\
  Existe algún número natural par.
\end{quote}

\noindent tienen un valor falso en el primer caso y verdadero en el segundo.
Hemos efectuado el proceso de cuantificar de alguna manera los elementos que
satisfacen la propiedad del predicado.

En una expresión, pueden aparecer implícita o explícitamente algún grupo de
palabras orientativas de la cantidad de elementos que satisfacen la
propiedad del predicado, tales como: ``para cualquier'', ``para cada'',
``todo'', ``para todo'', ``cada'', ``cualesquiera que sean'', etc., o cosas
como ``para algún'', ``existe'', ``existe al menos un'', etc.

En realidad, no existe una fórmula mágica para saber si una expresión en
lengua natural se debe interpretar refiriéndose a todos y cada uno de los
elementos de un conjunto. Puede que se refieran a ``un elemento arbitrario''
del conjunto, o incluso que esto mismo ni se mencione de forma explícita y
deba interpretarlo. TKTK. Un poco más adelante retomaremos esta cuestión.

En la lógica de primer orden, existen símbolos que representan estas ideas.
Estos reciben el nombre de \emph{cuantificadores} (\emph{quantifiers}).

Por muy extensa que sea la forma de expresar los cuantificadores en una
lengua natural, solo tenemos dos cuantificadores, como vemos a continuación.





\subsubsection{Cuantificador universal}

Sea $C$ un conjunto y $P_x$ un predicado sobre $C$. Consideremos el
subconjunto donde se cumple $P_x$:

$$ C_P = \{x \in C \st P_x\} $$

\noindent o, tal y como dijimos, la extensión del predicado $P$ sobre $C$.
Si para cada $x \in C$ se satisface $P_x$, escribiremos

$$ \forall x \in C.\ P_x $$

\noindent que se lee ``Para todo $x$ de $C$, $P_x$'', o, si lo prefiere,
``Cualquiera que sea el elemento $x$ de $C$, $x$ satisface $P_x$''. El
símbolo `$\forall$' se denomina \emph{cuantificador universal}
(\emph{universal quantifier}) y transforma un predicado en una proposición
con un valor verdadero o falso.

En algunos textos, en expresiones de este tipo ponen a la parte del
cuantificador entre paréntesis, al ser algo de menor relevancia que el
predicado. Por ejemplo, se tendría

$$ (\forall x \in C) \ P_x $$

\noindent pero personalmente no me gusta, ya que me parece una notación algo
sobrecargada.

Cuando se sobrentienda el conjunto del que se toman los valores de $x$, $C$
en el caso anterior, se puede poner simplemente

$$ \forall x. \ P_x $$

Observe que la proposición $\forall x \in C.\ P_x$ es equivalente a la
proposición $C_P = C$, es decir, la extensión del predicado es igual al
universo del predicado.

El cuantificador universal es una generalización de la conjunción ($\land$)
en el sentido siguiente. Supongamos que $C$ sea un conjunto finito; por
ejemplo, $C = \{1, 2, 3\}$. Entonces, la proposición $\forall x \in C.\ P_x$
es equivalente a la proposición $P_1 \land P_2 \land P_3$.





\subsubsection{Cuantificador existencial}

Sería el cuantificador dual al universal.

Sea $C$ un conjunto y $P_x$ un predicado sobre $C$. Consideremos el
subconjunto donde se cumple $P_x$:

$$ C_P = \{x \in C \st P_x\} $$

\noindent Si existe un elemento $x \in C$ que satisface $P_x$, escribiremos

$$ \exists x \in C.\ P_x $$

\noindent que se lee, ``Existe al menos un elemento $x$ de $C$ que satisface
$P_x$''. El símbolo `$\exists$' se denomina \emph{cuantificador existencial}
(\emph{existential quantifier}), y transforma un predicado en una
proposición.

También se podría expresar lo anterior como

$$ (\exists x \in C) \ P_x $$

Cuando se sobrentienda el conjunto del que se toman los valores de $x$, $C$
en el caso anterior, se puede poner simplemente

$$ \exists x. \ P_x $$

Observe que la proposición $\exists x \in C.\ P_x$ es equivalente a la
proposición $C_P \neq \emptyset$, es decir, la extensión del predicado no es
el conjunto vacío.

El cuantificador existencial es una generalización de la disyunción ($\lor$)
en el sentido siguiente. Supongamos que $C$ sea un conjunto finito. por
ejemplo $C = \{1, 2, 3\}$. Entonces la proposición $\exists x \in C.\ P_x$
es equivalente a la proposición $P_1 \lor P_2 \lor P_3$.

La variable empleada en la sintaxis de un predicado con cuantificadores no
tiene ninguna importancia. Tan solo lo tiene el universo de esa variable,
pues la proposición $\forall x \in C.\ P_x$ es equivalente a la proposición
$\forall y \in C.\ P_y$. Análogamente, la proposición $\exists x \in C.\
P_x$ es equivalente a la proposición $\exists z \in C.\ P_z$.

% TODO Hablar de dummy variable.

Los cuantificadores son un complemento muy útil para las expresiones por
comprensión de conjuntos.

\begin{example}[Cuantificadores]
  Veamos algunos ejemplos del uso de los cuantificadores.

  El conjunto de los números pares, $\{0, 2, 4, 6, \ldots\}$, denotado por
  $2\nset$, se escribe con más precisión como

  \[ \{x \in \nset \st \exists k \in \nset. \ x = 2k\} \]

  \noindent A veces, se omite la escritura del cuantificador y se da esta
  otra:

  \[ \{x \in \nset \st x = 2k, \ k \in \nset\} \]

  \noindent o incluso

  \[ \{2k \st k \in \nset\} \]

  Las proposiciones

  \[ \forall x \in \rset. \ x^2 - 1 = (x + 1)(x - 1) \]

  \noindent y

  \[ \exists x \in \rset. \ x + 5 = 3 \]

  \noindent son ambas verdaderas. La primera es una identidad que se cumple
  para todos los elementos de $\rset$, mientras que la segunda plantea una
  ecuación que tiene al menos una solución. Así, por ejemplo,

  \begin{align*}
    \forall x \in \rset. \ ax + b = 0 &\iff a = b = 0 \\
    \forall x \in \rset. \ ax + b = 0 &\iff (a \neq 0) \lor (a = b = 0)
  \end{align*}
\end{example}

Definición. Equivalencia de Predicados. Dos predicados $P_x$ y $Q_x$ son
equivalentes sobre un universo $C$ cuando determinan el mismo subconjunto
del mismo, es decir, $C_P = C_Q$. Se podría expresar como

$$ C_P = C_Q \iff \forall x \in C.\ P_x \leftrightarrow Q_x $$

Observación. La forma de escribir matemáticas ha ido variando a lo largo de
los años. Si hace unos años lo usual era escribir los enunciados de los
resultados con el máximo de símbolos posibles, la tendencia actual es la de
no usar tanta simbología, volviendo a usar en parte la prosa. Así, rara vez
se utilizan los símbolos de los cuantificadores, salvo en los temas de
lógica o de conjuntos. Sin embargo hay un uso implícito, o explícito pero
sin símbolos, de ellos. Expresiones como ``Si una función real de variable
real es derivable en un punto, entonces la función es continua en ese
punto'' y ``Un número primo es impar'', que aparentemente son predicados sin
cuantificar, desde el punto de vista matemático son dos enunciados que van
cuantificados y significan: ``Toda función real de variable real derivable
en un punto es continua en ese punto'', que es una proposición verdadera, y
``Todo número primo es impar'', que es una proposición falsa pues el número
2 es primo y no es impar.




\subsubsection{Relación entre los cuantificadores universal y existencial}

Aquí supondremos que el universo de la variable $x$ es el conjunto $C$ y nos
ahorraremos su escritura. Buscamos la negación de las proposiciones

\begin{align*}
  & \forall x. \ P_x \\
  & \exists x. \ P_x \\
\end{align*}

\noindent La primera de estas es equivalente a la proposición

$$ C_P = \{x \in C \st P_x\} = C $$

Por tanto, negando esta proposición nos encontramos, por un lado, con que
$\neg(\forall x. \ P_x)$ es equivalente a $C_P \neq C$. A su vez, esto
último es lo mismo que decir que existe al menos un $x \in C$ que no
satisface $P_x$, que en notación simbólica expresaríamos como $\exists x.\
\neg P_x$.

Análogamente, la proposición $\exists x. \ P_x$ es equivalente a la
proposición $C_P = \{x \in C \st P_x\} \neq \emptyset$. Si hacemos la
negación de este predicado, se tendría $\neg(\exists x. \ P_x)$ o $C_P =
\emptyset$. Esto se puede interpretar como que ningún elemento de $C$
satisface $P_x$, o equivalentemente, que todo elemento de $C$ satisface la
negación de $P_x$, que en notación simbólica se expresaría como $\forall x.\
\neg P_x$.

En definitiva, se cumplen las equivalencias siguientes:

\begin{align*}
  \neg(\forall x.\ P_x) &\iff \exists x.\ \neg P_x \\
  \neg(\exists x.\ P_x) &\iff \forall x.\ \neg P_x \\
\end{align*}

Estas equivalencias se usan en la práctica para manipular expresiones
lógicas de predicados y llegar a obtener conclusiones de una forma más
sistemática que si tuviésemos que razonarlo TKTK.

Concretamente, nos permiten resolver por la ``vía rápida'' muchos de los
problemas que se nos presenten. Así, si tenemos que demostrar que no es
cierto que se cumpla una propiedad $P_x$ para todos los elementos de un
conjunto $C$, es decir, $\neg(\forall x \in C.\ P_x)$, gracias a la primera
de estas equivalencias podemos comprobar que es cierto simplemente
demostrando que existe un elemento para el que no se cumple, que no es más
que demostrar que es cierto su predicado equivalente $\exists x \in C.\ \neg
P_x$. En este tipo de demostraciones se dice que se ha hecho uso de un
\emph{contraejemplo} (\emph{counterexample}), y se ven con bastante
frecuencia en las matemáticas. Evidentemente, se trata de demostraciones muy
cómodas.

\begin{example}
  La proposición $\forall x \in \rset. \ x \leq x^2$ es falsa. Es decir,
  deseamos demostrar

  \[ \neg(\forall x \in \rset. \ x \leq x^2) \]

  \noindent Como sabemos, esto equivale a

  \[ \exists x \in \rset. \ \neg(x \leq x^2) \]

  \noindent o, lo que es lo mismo,

  \[ \exists x \in \rset. \ x \not\leq x^2 \]

  Esto no es más que una demostración mediante un contraejemplo. Llamaremos
  $x_0$ al parámetro TKTK. Si $x_0 = 1/2$, se obtiene que $x_0^2 = 1/4$ y,
  por tanto, se tiene que $x_0 \not\leq x_0^2$.
\end{example}




