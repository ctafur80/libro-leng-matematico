



Algunas expresiones sencillas en contexto matemático no pueden ser
expresadas con la lógica de proposiciones. Por ejemplo,

\begin{quote}
  El número natural elegido es un número par. \\
  Un número múltiplo de 10 es un número par. \\
  Si una función es derivable en un punto, entonces la función es continua
    en ese punto. \\
  Un número primo mayor que 2 es impar. \\
\end{quote}

Analicemos la primera expresión, ``El número natural elegido es un número
par''. Podemos decir que, si bien vemos que describe la propiedad ``Ser
número par'', en realidad queda indefinida ya que no se indica el número al
que se aplica esta. Por eso, esta expresión es cierta o falsa dependiendo
del número que se elija.

\begin{deffinition}[Predicado Simple de Un Argumento]
  Dado un conjunto $C$, un \semph{predicado} (\emph{predicate}) simple de un
  argumento sobre $C$ es una proposición que tendrá un valor de verdad en
  función del valor que tome ese argumento, que no es más que una variable
  $x \in C$.
\end{deffinition}

Al conjunto de valores que puede tomar el argumento se le conoce como
\emph{universo del predicado}. Sería $C$ en la terminología de la definición
anterior.

La notación que se suele seguir es usar una letra mayúscula para la
proposición y poner el argumento sobre el que actúa como subíndice. Por
ejemplo, $P_x$. Existen otras notaciones; por ejemplo, la notación de
funciones, como $P(x)$.

Así, el enunciado

\begin{quote}
  El número natural elegido es un número par.
\end{quote}

\noindent se podría expresar como un predicado simple de un argumento $P_x$,
donde $P$ indicaría ``Ser un número par'' y $x$ el número natural elegido.
Para cada valor de $x$ se tendrá una proposición, y, como es evidente, esta
tomará un valor de verdad. Sería algo análogo al paso de la aritmética al
álgebra. Un salto de nivel ganando abstracción.\footnotemark

\footnotetext{Quizás le interese saber que Isaac Newton solía llamar al
álgebra la \emph{aritmética universal}.}

En este caso, si el universo del predicado es $C = \{1, 2, 3, 5. 6, 7\}$,
entonces para $x = 1$ se tendrá la proposición $P_1$, que es falsa. Para $x
= 2$ se tiene $P_2$; verdadera.

Se puede observar también el subconjunto del universo del predicado cuyos
elementos hacen que se tenga una proposición verdadera. Para el predicado
$P_x$ sobre un universo $C$, lo designaremos por $C_P$ y se puede expresar
del modo siguiente:

$$ C_P = \{x \in C \st P_x\} $$

\noindent El símbolo `$:$' se leería como ``tal que''. Evidentemente,
cualquiera que sea el predicado $P_x$, se tendrá que $C_P$ es un subconjunto
(no propio, necesariamente) del universo del predicado, $C$.

En el ejemplo anterior, se tiene que $C_P = \{2, 6\}$. En otro caso, podría
darse como particularidad que $C_P = C$, pues seguiría siendo un subconjunto
suyo.

En lugar de cosas como $C_P$, se puede expresar un conjunto de este tipo
simplemente como un conjunto cualquiera; por ejemplo, $A$. Se tendría que

$$ A = \{x \in C \st P_x\} $$

\noindent En este caso, se dice que el conjunto $A$ está definido o
expresado \semph{por comprensión} (también,
\emph{caracterización}\footnotemark). Se tendría que la propiedad $P$ es una
\emph{propiedad característica} del conjunto $A$ en $C$, y al conjunto $A$
se le llama \emph{extensión del predicado}.

\footnotetext{Existen otras formas de llamar a esto. En inglés, por ejemplo,
también la llaman \emph{set-builder notation}. TKTK.}

Esta es la otra forma de expresar o definir un conjunto, distinta a la forma
por extensión. Tal y como veremos, al tratar de ``construir'' conjuntos por
medio de predicados, pueden aparecer paradojas, como mencionamos antes.

Los conjuntos de los apartados 2, 3, 4 y 5 del
ejemplo~\ref{ejemplo:conjuntos_01} están definidos por comprensión; así, en
el punto 2 se tiene que $B = \{x \in \zset \st x^3 - 3x + 2 = 0\}$ o, en el
5, que $E = \{x \in \zset \st x \ \text{es par}\}$. También, el conjunto $B
= \{1, 2\}$ coincide con el conjunto $A$ del apartado 1.

A su vez, un conjunto puede estar determinado por distintos predicados. Por
ejemplo, si $C = \{x \in \zset \st 0 < x < 3\}$, entonces $B = C$. Se dice
que los predicados ``$x^2 - 3x + 2$'' y ``$0 < x < 3$'' son equivalentes
sobre el universo $\zset$. Es decir, dos predicados se consideran
\emph{equivalentes} sobre un universo $C$ cuando determinan un mismo
subconjunto de dicho universo, $C$.

También, al revés, se tiene que cualquier subconjunto $A$ de $C$, es decir,
$A \subseteq C$, definido por extensión, puede determinarse mediante el
predicado $x \in A$.

Sea $C$ un conjunto cualquiera. Consideramos sobre este el predicado $x
\notin C$. Se tendría un conjunto $A$ que podríamos expresar por comprensión
del modo siguiente:

$$ A = \{x \in C \st x \notin C\} $$

\noindent Por lo que dijimos, se cumple que $A \subseteq C$, pero se trata
de un subconjunto algo especial. Es un (sub)conjunto sin elementos. A este
se le llama \semph{conjunto vacío} (\emph{empty set}) y se denota con el
símbolo `$\emptyset$'. Alternativamente, se podría haber expresado por
comprensión haciendo uso de cualquier otra contradicción distinta a $x
\notin C$. En cualquier caso, lo importante es que esa contradicción conduce
a que el conjunto no tenga ningún elemento.

Es fácil demostrar que el conjunto vacío es un subconjunto de cualquier
conjunto. Simplemente hay que aplicar la definición de \emph{subconjunto}.
Supongamos un conjunto $C$ arbitrario. Dado un elemento arbitrario $x \in
\emptyset$, se da que $x \in C$. Esta proposición es verdadera;
concretamente, vacuamente cierta, ya que nunca se da el antecedente, es
decir, nunca se da $x \in \emptyset$.

No hay que confundir las expresiones conjuntistas ``$\emptyset$'' y
``$\{\emptyset\}$''. La primera de estas es el conjunto vacío, mientras que
la segunda denota al conjunto unitario cuyo único elemento es el conjunto
vacío. Puede ver también que se trata de conjuntos con distinto número de
elementos: 0 el primero y 1 el segundo.





\subsubsection{Lógica de predicados}

Los predicados que hemos visto hasta ahora son simples, ya que se
concretizan en proposiciones simples. Se tienen también los predicados
compuestos, que son los que lo hacen sobre proposiciones compuestas.

Sea $C$ un conjunto sobre el que están definidos diversos predicados: $P_x$,
$Q_x$, $R_x$, etc. Cada vez que damos un valor a $x$, por ejemplo, $x = c$
siendo $c \in C$, obtenemos las proposiciones $P_c$, $Q_c$, etc. a las que
se les puede aplicar todo el cálculo de proposiciones del capítulo anterior.
Por tanto, tienen sentido en $C$ los predicados

\[ \neg P_x, \ P_x \lor Q_x, \ P_x \to Q_x, \ P_x \land Q_x, \ \ldots \]

\noindent Estos predicados determinan diferentes subconjuntos de $C$
formados por los elementos de $C$ donde son ciertos los nuevos predicados.

La lógica de predicados también recibe el nombre de \emph{lógica de primer
orden}.

\begin{example}
  Si $C$ es un conjunto y $P_x$ un predicado sobre $C$, entonces

  \[ \emptyset = \{x \in C \st P_x \land \neg P_x\} \]

  Si recuerda de la lógica de proposiciones, ese predicado compuesto es una
  contradicción, $\contrad$, y, tal y como dijimos antes, este conduce a
  definir al conjunto vacío. También es fácil ver que

  \[ \emptyset = \{x \in \zset \st (x^2 = 9) \land (x \ \text{es par})\} \]

  \noindent ya que tenemos también otra contradicción, o que

  \[ \{x \in \zset \st (x \ \text{es par}) \land (x \ \text{es múltiplo de
  3})\} = \{x \in \zset \st x \ \text{es múltiplo de 6}\} \]
\end{example}



\subsection{Los números naturales}

A continuación, asumimos la existencia del conjunto de los números
naturales, que solemos representar por $\nset$. En el
capítulo~\ref{ch:naturales} se hará un estudio más completo de la
fundamentación de estos. Nos interesa, en primer lugar, que la definición
axiomática de $\nset$ asegura la existencia de conjuntos ``infinitos''.
Además, que estos presentan un hecho, muy útil en la práctica, conocido como
el Principio de Inducción.

Los números naturales. Aunque intuitivamente se conocen los números
naturales,

\[ \nset = \{0, 1, 2, 3, 4, 5, 6, 7, \dots\} \]

\noindent como los números que utilizamos para contar, y este proceso nos es
familiar desde nuestros comienzos en la educación primaria, resulta que la
existencia del conjunto de los números naturales se asegura mediante los
axiomas de Peano, que presentamos aquí de manera informal.

\begin{enumerate}[label=\textbf{A\arabic*.}, leftmargin=1.5cm]
  \item El elemento 0 es un número natural.
  \item Todo número natural $n$ tiene un único elemento sucesor que es
    también un número natural.
  \item 0 no es el sucesor de ningún número natural.
  \item Dos números naturales cuyos sucesores son iguales son iguales.
  \item Si un subconjunto de números naturales contiene al 0 y a los
    sucesores de cada uno de sus elementos, entonces contiene a todos los
    números naturales.
\end{enumerate}

De aquí se pueden extraer algunas conclusiones. Por ejemplo, A1 permite
asegurar que el conjunto de los números naturales es un conjunto no vacío.
Hablar de sucesor en A2 refleja precisamente la idea de contar. A3 indica
que hay un primer elemento en dicho conjunto. A2, A3 y A4, en forma
conjunta, aseguran que al ir contando nunca volvernos a un mismo elemento.
A5 es el axioma utilizado en las demostraciones por inducción. Es la
formulación conjuntista del principio siguiente:

% TODO Meterlo en un entorno.

Principio de Inducción. Si $P$ es una propiedad definida sobre $\nset$ tal
que:

\begin{enumerate}
  \item El número 0 satisface la propiedad $P$, es decir, $P_0$ es
    verdadera.
  \item Si $n$ satisface la propiedad $P$, entonces el sucesor de $n$
    también la satisface.
\end{enumerate}

\noindent Entonces todo número natural satisface la propiedad $P$.

En efecto, si consideramos el subconjunto $M$ de los elementos de $\nset$
(es decir, $M \subseteq \nset$) que satisfacen la propiedad $P$ y
comprobamos que $M$ contiene al 0 y a los sucesores de cada demento,
aplicando entonces el Principio de Inducción (o, si lo prefiere, el A5 de
Peano) se obtiene que $\nset \subseteq M$. Entonces, como se cumplen $M
\subseteq \nset$ y $\nset \subseteq M$, se tiene que $M = \nset$, por la
propiedad de la doble inclusión de conjuntos, que vimos antes.

\begin{example}[Demostración por Inducción]
  Demuéstrese para todo número natural la igualdad

  \[ \frac{0}{2^0} + \frac{1}{2^1} +  \frac{2}{2^2} +  \frac{3}{2^3} +
  \cdots + \frac{n}{2^n} = 2 - \frac{n + 2}{2^n} \]

  La igualdad es verdadera para $n = 0$, pues

  \[ \frac{0}{2^0} = 0 = 2 - \frac{0 + 2}{2^0} \]

  Supongamos que la igualdad es cierta para $n$, esto es,

  \[ \frac{0}{2^0} + \frac{1}{2^1} +  \frac{2}{2^2} +  \frac{3}{2^3} +
  \cdots + \frac{n}{2^n} = 2 - \frac{n + 2}{2^n} \]

  \noindent y comprobemos que es cierta para el sucesor de $n$, es decir,
  para $n + 1$. En consecuencia hay que comprobar que

  \[ \frac{0}{2^0} + \frac{1}{2^1} +  \frac{2}{2^2} +  \frac{3}{2^3} +
  \cdots + \frac{n}{2^n} + \frac{n+1}{2^{n+1}} = 2 - \frac{(n+1) +
  2}{2^{n+1}} \]

  En efecto:

  \begin{align*}
    \frac{0}{2^0} + \frac{1}{2^1} + \frac{2}{2^2} + \cdots + \frac{n}{2^n} +
      \frac{n+1}{2^{n+1}}
    &= \left( \frac{0}{2^0} + \frac{1}{2^1} +  \frac{2}{2^2} +  \frac{3}{2^3}
      + \cdots + \frac{n}{2^n} \right) + \frac{n+1}{2^{n+1}} \\
    &= \left( 2 - \frac{n+2}{2^n} \right) + \frac{n+1}{2^{n+1}} \\
    &= 2 - \frac{2n + 4 - n - 1}{2^{n-1}} = 2 - \frac{n + 3}{ 2^{n+1}} \\
    &=  2 - \frac{(n+1) + 2}{2^{n+1}} \\
  \end{align*}
\end{example}

Las demostraciones que se hacen echando mano del Principio de Inducción son
muy comunes en matemáticas. El principal problema con el que cuentan es que
no nos van indicando, sino que debemos saber qué es lo que deseamos
demostrar. TKTK.

El A5 de Peano también se utiliza para definir términos donde intervienen
los números naturales donde se define el objeto que depende de un número
natural en función de objetos que dependen de términos anteriores. Se trata
de definiciones que hacen uso de la autorreferencia, cosa que está prohibida
en otros ámbitos\footnotemark{} pero, en matemáticas, bajo ciertas
circunstancias, sí se permite.

\footnotetext{Recuerde que se dijo en la definición ``ingenua'' de
\emph{conjunto}.}

A las definiciones que emplean la autorreferencia se las conoce como
definiciones \emph{recurrentes} o \emph{por recurrencia}. Aunque se ve
menos, también se las podría llamar, perfectamente, definiciones
\emph{autorreferentes}\dots{} o \emph{inductivas}, ya que se basan en el
Principio de Inducción.

Un ejemplo de definición recurrente es la del factorial de un número natural
$n$. Para cualquier número natural $n \in \nset$, se define $(n + 1)!$ en
función de $n!$ mediante

$$ (n + 1)! = n! \, (n + 1) $$

\noindent y se lee ``factorial de $n + 1$''. Es evidente que hay que conocer
el valor de $0!$ para poder determinar todos los demás, pues, de lo
contrario, estaría sin definir ya que habría toda una gama de factoriales.
Se define $0! = 1$, cosa que hace que se trate de una función concreta. En
resumen, la definición recurrente de factorial de $n$ es

\begin{equation*}
  (n+1)! =
  \begin{cases}
    1             & \text{si}\ n + 1 = 0 \\
    (n)!\,(n+1)   & \text{si}\ n + 1 > 0
  \end{cases}
\end{equation*}

\noindent o, si lo prefiere,

\begin{equation*}
  n! =
  \begin{cases}
    1           & \text{si}\ n = 0 \\
    (n-1)!\,n   & \text{si}\ n > 0
  \end{cases}
\end{equation*}

\noindent que me parece más elegante. Como puede ver, se está usando un
factorial dentro de la definición del factorial.

De esta definición se obtiene fácilmente su forma no recursiva:

% TODO ¿A la forma no recursiva se la conoce también como forma explícita?

\[ n! = n \cdot (n-1) \cdot (n-2) \cdot (n-3) \cdot \dots \cdot 3 \cdot 2
\cdot 1 \]

Si $n'$ designa el sucesor de $n$, los cinco axiomas de Peano permiten
pensar en $\nset$ como en el conjunto

\[ \{0, 0', (0')', ((0')')', \dots\} \]

Existe cierta controversia sobre la inclusión del número 0 en el conjunto de
los números naturales, $\nset$, pues a veces se excluye de este conjunto. No
existe un consenso generalizado y, de hecho, para las distintas áreas de las
matemáticas, hay razones tanto para incluirlo como para excluirlo. Así, en
la lógica y la teoría de conjuntos parece evidente que ha de darse que $0
\in \nset$, pero esto no sucede en otras áreas. En cualquier caso, esto es
simplemente una cuestión de terminología y no afecta a la validez de los
resultados que se obtengan. En caso de excluirlo, el Principio de Inducción
se enunciaría de tal forma que comience con el 1, en lugar de con 0.

En la terminología que seguiremos en este texto, sí consideramos que $0 \in
\nset$. Si deseamos designar a $\nset$ excluyendo el 0, usaremos la notación
$\nset^*$, es decir,

\[ \nset^* = \{1, 2, 3, 4, 5, 6, 7, \dots\} \]

Definición. Conjuntos finitos y conjuntos infinitos. Los conjuntos pueden
ser finitos o infinitos. Intuitivamente, un conjunto es finito si contando
los diferentes elementos del conjunto, el proceso de contar se termina. En
caso contrario, el conjunto es infinito. En capítulos posteriores se verá
una definición más precisa de estos dos conceptos. En cualquier caso, los
conjuntos $A = \{1, 2\}$, $B = \{a, e, i, o, u\}$ y $C = \{2, i, \text{museo
del Prado}\}$ son conjuntos finitos. Los axiomas A2, A3 y A4 de Peano
permiten asegurar que el proceso de contar los elementos del conjunto
$\nset$ no se acaba nunca. Es decir, $\nset$ es un conjunto infinito. Esto
se tratará de forma más rigurosa en el capítulo~\ref{ch:naturales}.



