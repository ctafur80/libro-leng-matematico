








\section{Forma clausulada de proposiciones}

Dada una proposición compuesta por un conjunto de proposiciones simples \(p,
q, r, \dots\), se busca encontrar una proposición equivalente escrita
únicamente como una conjunción (\(\land\)) de disyunciones (\(\lor\)). En
estas disyunciones solo pueden aparecer las proposiciones simples o sus
negaciones. Por ejemplo:

\[ (p \lor \neg q) \land (\neg p \lor r). \]

Esta forma se denomina \textbf{forma clausulada de la proposición} o
\emph{forma normal conjuntiva}, y cada una de las disyunciones se llama
\textbf{cláusula lógica}.

Tener la forma clausulada de una proposición facilita verificar su validez,
ya que solo es necesario comprobar que todas las cláusulas son verdaderas.

\subsection{Ejemplo 1.17: Forma clausulada de un condicional}

La primera ley del condicional:

\[ p \to q \iff \neg p \lor q \]

\noindent establece que la forma clausulada de un condicional \(p \to q\)
está formada por una única cláusula:

\[ \neg p \lor q. \]

\subsection{Pasos para obtener la forma clausulada}

Para transformar una proposición a su forma clausulada, se siguen estos
pasos:

\begin{enumerate}
    \item \textbf{Eliminar conectores bicondicionales
      (\(\leftrightarrow\)):} Se sustituye cada bicondicional \(p
      \leftrightarrow q\) por:

    \[ (p \to q) \land (q \to p). \]

    \item \textbf{Eliminar conectores condicionales (\(\to\)):} Se utiliza
    la equivalencia:

    \[ p \to q \iff \neg p \lor q. \]

    \item \textbf{Aplicar las leyes de De Morgan:} Transformar negaciones
    que actúan sobre conjunciones o disyunciones:

    \begin{align*}
        \neg (p \land q) &\iff (\neg p \lor \neg q), \\
        \neg (p \lor q) &\iff (\neg p \land \neg q).
    \end{align*}

    \item \textbf{Distribuir conjunciones sobre disyunciones:} Usar las
    leyes distributivas:

    \begin{align*}
        p \lor (q \land r) &\iff (p \lor q) \land (p \lor r), \\
        p \land (q \lor r) &\iff (p \land q) \lor (p \land r).
    \end{align*}
\end{enumerate}

\subsection{Ejemplo 1.18: Forma clausulada de un bicondicional}

La ley del bicondicional establece que:

\[ p \leftrightarrow q \iff (p \to q) \land (q \to p). \]

Aplicando la equivalencia del condicional (\(p \to q \iff \neg p \lor q\)),
la forma clausulada del bicondicional \(p \leftrightarrow q\) es:

\[ (\neg p \lor q) \land (\neg q \lor p). \]



\subsection{Ejemplo 1.19: Forma clausulada de una proposición compleja}

Dada la proposición:

\[ p \to (p \land q), \]

transformemos paso a paso para obtener su forma clausulada:

\begin{enumerate}
    \item \textbf{Eliminar el condicional:} Usamos la equivalencia \(p \to
      (p \land q) \iff \neg p \lor (p \land q)\):

    \[ \neg p \lor (p \land q). \]

    \item \textbf{Distribuir la disyunción:} Aplicamos la ley distributiva:

    \[ (\neg p \lor p) \land (\neg p \lor q). \]

    \item \textbf{Simplificar:} Usamos la ley del tercero excluido (\(\neg p
    \lor p \iff 1\)):

    \[ 1 \land (\neg p \lor q). \]

    \item \textbf{Eliminar redundancias:} Finalmente, obtenemos:

    \[ \neg p \lor q. \]
\end{enumerate}

La forma clausulada de \(p \to (p \land q)\) es:

\[ \neg p \lor q. \]


\subsection{Ejemplo 1.20: Comprobación de una tautología mediante forma clausulada}

Consideremos la proposición:

\[ [(p \to q) \land (q \to r)] \to (p \to r). \]

Transformémosla a su forma clausulada paso a paso:
\begin{enumerate}
    \item \textbf{Eliminar conectores condicionales:}

    \[ [(p \to q) \land (q \to r)] \to (p \to r) \iff \neg [(p \to q) \land
    (q \to r)] \lor (p \to r). \]

    \item Sustituyendo \(p \to q \iff \neg p \lor q\), obtenemos:

    \[ \neg [(\neg p \lor q) \land (\neg q \lor r)] \lor (\neg p \lor r). \]

    \item Aplicar la ley de De Morgan a \(\neg [(\neg p \lor q) \land (\neg
      q \lor r)]\):

    \[ [(\neg \neg p \land \neg q) \lor (\neg \neg q \land \neg r)] \lor
    (\neg p \lor r). \]

    \item Simplificar las dobles negaciones:

    \[ [(p \land \neg q) \lor (q \land \neg r)] \lor (\neg p \lor r). \]

    \item Distribuir la disyunción:

    \[ (p \lor \neg p \lor r) \land (\neg q \lor \neg p \lor r) \land (q
    \lor \neg p \lor r). \]

    \item Simplificar usando las leyes de identidad y el tercero excluido
      (\(p \lor \neg p \iff 1\)):

    \[ 1 \land (\neg q \lor \neg p \lor r) \land (q \lor \neg p \lor r). \]

    \item Finalmente, eliminando redundancias:

    \[ (\neg q \lor \neg p \lor r) \land (q \lor \neg p \lor r). \]
\end{enumerate}

Dado que la forma clausulada es válida en todas las combinaciones posibles
de \(p, q, r\), la proposición original es una tautología.








