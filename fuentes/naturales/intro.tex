


Hemos supuesto a lo largo del texto que el lector conoce, al menos
iutuitivamente, los números naturales, los enteros, los racionales y los
reales. Conoce cómo se suman, se multiplican e incluso sabe reconocer cuándo
un número es mayor que otro.

En este capítulo, vamos a fundamentar todas estas propiedades sobre los
números naturales y enteros. Es decir, el objeto del capítulo es justificar
resultados familiares y conocidos. En los ejemplos 2.5 y 3.8, hemos
introducido los conjuntos de números naturales, $\nset$ y enteros, $\zset$.
El primer conjunto se ha introducido mediante los axiomas de Peano, mientras
que los enteros se han construido partiendo de los naturales como conjunto
cociente de una determinada relación de equivalencia. Para facilitar la
lectura del capítulo, repetiremos estas construcciones.

La idea básica de los números naturales es que sirven para contar los
elementos de los conjuntos finitos y que dos conjuntos tienen el mismo
número de elementos cuando existe una biyección entre ellos. Retomaremos,
pues, el concepto de \emph{cardinal}, introducido en la
sección~\ref{sec:equipotencia-conjuntos}, centrándonos en los cardinales
finitos y en los cardinales numerables, ambos conceptos íntimamente
relacionados con la noción de número natural.

Los números naturales no forman un grupo respecto de la suma. La ecuación $b
+ x = a$ no tiene solución en $\nset$ si $a < b$. Construimos el conjunto de
los números enteros $\zset$, donde esta ecuación tendrá siempre solución.
Este conjunto será una extensión del conjunto de los números naturales en el
sentido de que identificaremos $\nset$ con un subconjunto de $\zset$,
conservando las operaciones y el orden.

Estudiaremos los conceptos de máximo común divisor y mínimo común múltiplo
vía los ideales de $\zset$, que aportan un método sencillo y natural para
introducirlos.

















