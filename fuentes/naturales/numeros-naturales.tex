



Tal y como se explicó anteriormente, el lector seguramente conoce
intuitivamente los números naturales como los números que se utilizan para
contar:

$$ \nset = \{0, 1, 2, 3, 4 \dots\} $$

\noindent El cero, 0, a veces se opta por incluirlo y otras por no hacerlo,
pero tampoco es relevante en realidad. En nuestro caso, sí lo incluiremos, y
usaremos la notación

$$ \nset^* = \{1, 2, 3, 4 \dots\} $$

\noindent para cuando deseemos excluirlo.

Volvemos a introducir los axiomas de Peano, como hicimos antes, solo que
ahora se dará una justificación más profunda de por qué estos constituyen al
conjunto de los números naturales.

\begin{enumerate}[label=\textbf{A\arabic*.}, leftmargin=1.5cm]
  \item El elemento 0 (\emph{cero}) es un número natural.
  \item Todo número natural $n$ tiene un único elemento sucesor, $s(n)$, que
    es también un número natural.
  \item 0 no es el sucesor de ningún número natural.
  \item Dos números naturales cuyos sucesores son iguales son iguales.
  \item Si un conjunto de números naturales contiene al 0 y a los sucesores
    de cada uno de sus elementos, entonces contiene a todos los números
    naturales.
\end{enumerate}

Por el segundo axioma, existe una aplicación $s: \nset \longrightarrow
\nset$ que asigna a cada número natural su sucesor. Además, el cuarto axioma
garantiza que esta aplicación $s$ es inyectiva. Si no lo ve de forma
directa, puede transformarlo mediante la regla del contrarrecíproco de la
lógica, con lo que se tendría que, si dos números naturales son distintos,
sus respectivos sucesores son distintos.

% Lo siguiente sería la razón de que se incluya el A5.

El tercer axioma asegura que $0 \notin \text{Im}(s)$ y el quinto axioma
permite asegurar que el número 0 es el único elemento sin antecesor en este
conjunto, es decir, en $\nset$. Se podría demostrar pensando que, en caso
contrario, existe $a \in \nset$, siendo $a \neq 0$, tal que $a \notin
\text{Im}(s)$. En este caso, el conjunto $A = \nset \setminus \{a\}$, es un
conjunto de naturales que satisface las hipótesis del quinto axioma y, por
tanto, $\nset \subseteq A$, lo cual es una contradicción.

En resumen:

% ¿Esto no sería lo mismo que acabamos de decir?

Propiedad. Todo número natural no nulo es sucesor de algún número natural.

Se podría expresar también en forma simbólica, del modo siguiente:

$$ \forall n \in \nset \setminus \{0\} \ \exists k \in \nset. \ s(k) = n $$

\noindent Advierta que esto no es lo mismo que el axioma 2.

Los axiomas de Peano permiten también ver que los elementos que se van
generando son distintos. En concreto:

Propiedad. Para todo $n \in \nset$, $n \neq s(n)$.

% En realidad, esto no impide la circularidad. Es decir, aunque se cumpla
% que, para todo $n$, se tiene que $n \neq s(n)$, esto no impide que se
% pueda tener $k = s(s(k))$ para algún $k$.

Esta propiedad es consecuencia de los axiomas. El más relevante a este
respecto es el 4. Supongamos el conjunto

$$ A = \{ n \in \nset \st n \neq s(n) \} $$

\noindent que, evidentemente, es un subconjunto de $\nset$, es decir, $A
\subseteq \nset$.

Por el axioma 3, como 0 no es sucesor de ningún número natural y $s(0)$ sí
lo es, no pueden ser iguales, por lo que $0 \neq s(0)$, o, lo que es lo
mismo, $0 \in A$.

Supongamos que para un número natural arbitrario $n$ se cumple $n \in A$.
Supongamos que $s(n) \notin A$, con lo que se tendrá que $s(n) = s(s(n))$.
Como consecuencia, por el cuarto axioma se cumple entonces que $n = s(n)$,
es decir, $n \notin A$, pero esto es una contradicción con la hipótesis de
partida, con lo que se ha de cumplir tambíen necesariamente que $s(n) \neq
s(s(n))$.

Tal y como hemos demostrado, se trata de una propiedad que se define
inductivamente (o recursivamente, si prefiere llamarlo así). Se ha visto que
se cumple para el 0 y, además, si se cumple para un $n$ cualquiera, también
se cumplirá para su sucesor, $s(n)$.

Además, como el conjunto $A$ contiene al 0 y a los sucesores de todos sus
elementos, se tendrá, por el quinto axioma, que $\nset \subseteq A$.
Juntando esto con que $A \subseteq \nset$, tal y como dijimos al definirlo,
tenemos que $A = \nset$.

Por cierto, aún no hemos demostrado que no exista circularidad en esta
construcción de los números naturales. Es decir, nada impide que se tenga,
para algún $n \in \nset$, $n = s(s(n))$. TKTK.

Como ya habíamos observado en el ejemplo 2.5, los cinco axiomas de Peano
permiten pensar en $\nset$ como en el conjunto:

$$ \nset = \left\{ 0, s(0), s(s(0)), s(s(s(0))), \dots \right\} $$

De esta manera, cero, uno, dos, tres, etc., son las denominaciones de cero,
sucesor de cero, sucesor del sucesor de cero, sucesor del sucesor del
sucesor de cero, etc., y 0, 1, 2, 3, etc., son las notaciones utilizadas
para \(0, s(0), s(s(0)), s(s(s(0)))\), etc.





\subsection{Suma}

La suma de números naturales se define por recurrencia utilizando el axioma
5.

Definición. Suma. Dados $m, n \in \nset$, se define por recurrencia sobre
$n$ la suma $m + n$ mediante:

\begin{enumerate}
  \item $m + 0 = m$ para todo $m \in \nset$.
  \item $m + s(n) = s(m + n)$ para todo $m, n \in \nset$.
\end{enumerate}

De la definición anterior se obtiene que $m + 1 = s(m)$ para todo $m \in
\nset$, pues

$$ m + 1 = m + s(0) = s(m + 0) = s(m) $$

También se cumple que $1 + m = s(m)$ para todo $m \in \nset$, pues
procediendo por inducción sobre $m$ se tiene que la propiedad es cierta para
$m = 0$ pues

\begin{align*}
  1 + 0 &= s(0) \\
  1 &= 1 \\
\end{align*}

Ahora, supongamos que la propiedad es cierta para $m$, esto es, $1 + m =
s(m)$ y veamos que es cierta para $s(m)$, esto es, $1 + s(m) = s(s(m))$. En
efecto, se tiene

$$ s(s(m)) = s(1 + m) = 1 + s(m) $$

\noindent La primera igualdad se justifica por la hipótesis de inducción y,
la segunda, por la definición recursiva de suma.

A partir de ahora, utilizaremos indistintamente las notaciones $s(m)$ o $m +
1$, o $1 + m$.

Por otro lado, dados $m, n \in \nset$, si $m + n = 0$, entonces $m = n = 0$.

Veamos por qué. Si $n \neq 0$, existe $r \in \nset$ tal que $n = s(r)$ y,
por tanto,

$$ 0 = m + n = m + s(r) = s(m + r) $$

\noindent en contradicción con el axioma $A3$. En consecuencia, $n = 0$.
Entonces, sustituyendo en la expresión original, se tiene $m + 0 = 0$, de lo
que se deduce, del caso base de la definición recursiva de suma, que también
$m = 0$.

Las propiedades básicas de esta operación están resumidas en la proposición
siguiente.

Proposición. Propiedades de la Suma. La suma de números naturales es una
operación interna en $\nset$ que satisface, cualesquiera que sean $m, n$ y
$p \in \nset$, las propiedades siguientes:

\begin{enumerate}
  \item Existencia del elemento neutro. Es decir, $m + 0 = 0 + m = m$.
  \item Asociativa. Es decir, $(m + n) + p = m + (n + p)$.
  \item Conmutativa. Es decir, $m + n = n + m$.
  \item Cancelativa. Es decir, Si $m + p = n + p$, entonces $m = n$.
\end{enumerate}

Demostración. Las cuatro propiedades se demuestran por inducción.

1. Propiedad del elemento neutro. Que $m + 0 = m$ es evidente al ser el caso
base en la definición recursiva de la suma, con lo que solo habría que
demostrar que $0 + m = m$. Por inducción sobre $m$ se tiene que la propiedad
es cierta para $m = 0$ pues $0 + 0 = 0$.

Supongamos que la propiedad es cierta para $m$, esto es, $0 + m = m$, y
veamos que es cierta para $s(m)$, esto es, $0 + s(m) = s(m)$. En efecto:

$$ 0 + s(m) = s(0 + m) = s(m) $$

\noindent La segunda igualdad se justifica por la hipótesis de inducción.

2. Propiedad asociativa. Se procede por inducción sobre $p$.

Por un lado, la propiedad es cierta para $p = 0$, pues

$$ (m + n) + 0 = m + n = m + (n + 0) $$

\noindent La primera igualdad por la definición de suma y, la segunda, por
una propiedad de esta.

Supongamos que la propiedad es cierta para $p$, esto es, $(m + n) + p = m +
(n + p)$ y veamos que es cierta para $s(p)$, esto es, $(m + n) + s(p) = m +
(n + s(p))$. En efecto:

\begin{align*}
  (m + n) + s(p)
    &= s((m + n) + p) \\
    &= s(m + (n + p)) \\
    &= m + s(n + p) \\
    &= m + (n + s(p)) \\
\end{align*}

3. Propiedad conmutativa. Procedemos por inducción sobre $n$.

Tal y como se demostró en la propiedad del elemento neutro, se cumple para
$n = 0$; esto es, $m + 0 = 0 + m$.

Supongamos que la propiedad es cierta para $n$, esto es, $m + n = n + m$, y
veamos que es cierta para $s(n)$, esto es, $m + s(n) = s(n) + m$. En efecto:

\begin{align*}
  m + s(n)
    &= s(m + n) \\
    &= s(n + m) \\
    &= n + s(m) \\
    &= n + (1 + m) \\
    &= (n + 1) + m \\
    &= s(n) + m \\
\end{align*}

\noindent Advierta que hemos usado la hipótesis de inducción, $m + n = n +
m$, y la propiedad asociativa, $n + (1 + m) = (n + 1) + m$, que se justificó
en el apartado anterior.

4. Propiedad cancelativa. Procedemos por inducción sobre $p$.

La propiedad es cierta para $p = 0$ por cumplirse la propiedad del elemento
neutro. Es decir, se tiene que, si $m + 0 = n + 0$, entonces $m = n$.

La hipótesis de inducción será que, si $m + p = n + p$, entonces $m = n$.
Ahora, veamos si se cumple la meta de inducción, que sería que de $m + s(p)
= n + s(p)$ se deduce que $m = n$. Tenemos

\begin{align*}
  m + s(p)  &= n + s(p) \\
  s(m + p)  &= s(n + p) \\
  m + p     &= n + p \\
  m         &= n \\
\end{align*}

\noindent Advierta que de $s(m + p) = s(n + p)$ se deduce, por el axioma 4,
que $m + p = n + p$. El último paso se justifica por la hipótesis de
inducción.






\subsection{Producto}

El producto de números naturales se define por recurrencia utilizando el
axioma $A5$.

Definición. Producto. Se define por recurrencia sobre $n$ el producto, que
designaremos por $m \cdot n$ o $mn$, de los números naturales $m$ y $n$
mediante:

\begin{enumerate}
  \item $m \cdot 0 = 0$ para todo $m \in \nset$.
  \item $m \cdot s(n) = (m \cdot n) + m$ para todo $m, n \in \nset$.
\end{enumerate}

Nótese que el apartado 2 en esta definición se escribe también como:

$$ m(n + 1) = (mn) + m \ \text{para todo} \ m, n \in \nset $$

\noindent que seguramente sea más fácil de recordar.

Se obtiene que $0 \cdot m = 0$ para todo $m \in \nset$. Se demostraría
también por inducción. Para el caso base, $m = 0$, se tiene $0 \cdot 0 = 0$,
por el punto 1 de la definición de producto.

Supongamos que la propiedad es cierta para $m$, esto es, $0 \cdot m = 0$ y
veamos que es cierta para $s(m)$, esto es, $0 \cdot s(m) = 0$. En efecto:

$$ 0 \cdot s(m) = (0 \cdot m) + 0 = 0 + 0 = 0 $$

\noindent La segunda igualdad, se justifica por la hipótesis de inducción,
y, la última, del elemento neutro de la suma.

Debido al hecho de que $m \cdot 0 = 0 \cdot m = 0$, se dice que 0 es
absorbente para el producto.

Conociendo este último hecho, podemos demostrar que $m \cdot 1 = m$ para
todo $m \in \nset$, pues

$$ m \cdot 1 = m \cdot s(0) = (m \cdot 0) + m = 0 + m = m $$

\noindent Entre otras, se ha hecho uso del punto 1 en la definición de
producto y del elemento neutro de la suma.

Resumimos las propiedades básicas del producto en la proposición siguiente:

Proposición. Propiedades del Producto. El producto de números naturales es
una operación interna en $\nset$ que satisface, cualesquiera que sean $m, n$
y $p \in \nset$, las propiedades siguientes:

\begin{enumerate}
  \item Existencia del elemento neutro. Es decir, $m \cdot 1 = 1 \cdot m = m$.
  \item Distributiva respecto a la suma. Es decir,

    \begin{align*}
      m(n + p) &= mn + mp \\
      (n + p)m &= nm + pm
    \end{align*}
  \item Asociativa. Es decir, $(mn)p = m(np)$.
  \item Conmutativa. Es decir, $mn = nm$.
  \item Cancelativa. Es decir, Si $mp = np$ y $p \neq 0$, entonces $m = n$.
\end{enumerate}

Demostración. Las cinco propiedades se demuestran por inducción y son
análogas a las demostraciones de las propiedades de la suma. Demostraremos
la primera, la segunda y la última.

1. Sólo hay que demostrar que $1 \cdot m = m$ para todo $m \in \nset$, pues
la otra ya se demostró. Procedemos por inducción sobre $m$.

La propiedad es cierta para $m = 0$ pues $1 \cdot 0 = 0$, por el punto 1 en
la definición de producto.

Supongamos que la propiedad es cierta para $m$, esto es, $1 \cdot m = m$ y
veamos que es cierta para $s(m)$, esto es, $1 \cdot s(m) = s(m)$. En efecto:

$$ 1 \cdot s(m) = (1 \cdot m) + 1 = m + 1 = s(m) $$

\noindent En otras palabras, 1 es el elemento neutro del producto.

2. Se procede por inducción sobre $p$. Sólo demostraremos la primera
propiedad distributiva.

La propiedad es cierta para $p = 0$ pues $m(n + 0) = mn = mn + 0 = mn + m
\cdot 0$.

Supongamos que la propiedad es cierta para $p$, esto es, $m(n + p) = mn +
mp$ y veamos que es cierta para $s(p)$, esto es, $m(n + s(p)) = mn + ms(p)$.
En efecto:

$$ m(n + s(p)) = m \cdot s(n + p) = m(n + p) + m = (mn + mp) + m = mn + (mp
+ m) = mn + ms(p) $$

\noindent En la penúltima igualdad, se hace uso de la propiedad asociativa
de la suma.

Habría que demostrarla también en el otro sentido. En cualquier
caso, si demostrásemos antes la propiedad conmutativa del producto, bastaría
con demostrar una de estas dos.

3. Se demuestra por inducción sobre $p$.

4. Se demuestra por inducción sobre $n$.

5. Procedemos por inducción sobre $n$.

La propiedad es cierta para $n = 0$. Hay que demostrar que, si $mp = 0 \cdot
p = 0$ y $p \neq 0$, entonces $m = 0$. Si $p \neq 0$, entonces existe $q \in
\nset$ tal que $s(q) = p$. De $mp = 0$, sustituyendo se obtiene que $ms(q) =
0$, esto es, $mq + m = 0$. De una observación de la definición de suma, se
deduce que $mq = m = 0$.

Supongamos que la propiedad es cierta para $n$, esto es, que para todo $m,
p$, de $mp = np$ y $p \neq 0$ se deduce que $m = n$. Veamos que de $mp =
s(n) \cdot p$ también se deduce que $m = s(n)$. En efecto. Observemos en
primer lugar que $m \neq 0$, pues, si $m = 0$, entonces $s(n) \cdot p = mp =
0$ y, por tanto, de (i) se deduce que $s(n) = 0$, lo que contradice el
axioma 3. En consecuencia $m = s(r) = r + 1$ para un cierto $r \in \nset$.

Sustituyendo en la igualdad $mp = s(n) \cdot p$, se obtiene $(r + 1)p = (n +
1)p$, esto es $rp + p = np + p$. Por la propiedad cancelativa de la suma se
obtiene que $rp = np$ y, por la hipótesis de inducción, se deduce que $s(r)
= s(n)$, es decir, $m = s(n)$.

Observación. De las propiedades cancelativa y conmutativa del producto se
deduce que, si $m, p \in \nset$ y $mp = 0$, entonces $m = 0$ o $p = 0$.

Una vez definido el producto, se define la potenciación de números naturales
en forma recurrente.

Definición. Potencia $n$-ésima. Se define la potencia $n$-ésima de $a$,
$a^n$, mediante

\begin{enumerate}
  \item $0^n = 0$ para todo $n \in \nset^*$.
  \item $a^0 = 1$ para todo $a \in \nset^*$.
  \item $a^{n+1} = a^n \cdot a$ para todo $a \in \nset^*$ y $n \in \nset$.
\end{enumerate}

\noindent Otra forma de presentarlo:

$$
  a^n =
  \begin{cases}
    0 & \text{para} \ a = 0 \ \text{y} \ n \in \nset^* \\
    1 & \text{para} \ a \in \nset^* \ \text{y} \ n = 0 \\
    a^{n-1} \cdot a & \text{para} \ a \in \nset^* \ \text{y} \ n \in \nset^*
  \end{cases}
$$

Observaciones. 1) Si $n \in \nset^*$, es fácil ver que $a^n = \overbrace{a
\cdot a \cdot \dots \cdot a}^{n \ \text{veces}}$.

2) Hemos dejado sin definir el valor de $0^0$, pues no hay un tratamiento
único al respecto y depende del contexto en el que se maneje. En algunas
áreas de las matemáticas, se suele considerar que existe mientras que, en
otras, se evita su aparición.

En muchos contextos, donde no intervienen argumentos de continuidad,
interpretar $0^0$ como $1$ simplifica fórmulas y elimina a veces el tener
que estudiar el caso $0$ como caso especial. Es habitual, por tanto, usar la
convención $0^0 = 1$ en teoría de conjuntos o en álgebra. Por ejemplo, en la
teoría de polinomios o series de potencias, las notaciones se simplifican
notablemente si una constante $a$ se escribe como $ax^0$ para un $x$
arbitrario. Por ejemplo, la expresión del binomio de Newton

$$ (1 + x)^n = \sum_{i=0}^{n} \binom{n}{i} x^i $$

\noindent no es válida para $x = 0$ salvo que $0^0$ se sustituya por $1$. O
la regla de derivación de $x^n$, $(x^n)' = nx^{n-1}$, no es válida para $n =
1$ y $x = 0$ salvo que a $0^0$ se le dé el valor $1$.

Por otro lado, $0^0$ debe fijarse como una indeterminación cuando se obtiene
como expresión algebraica en el cálculo de límites: cuando $f, g \in
\mathcal{F}(\rset, \rset)$ con $f(x) > 0$ y $\lim_{x \to a} f(x) = \lim_{x
\to a} g(x) = 0$, el límite de la función $f(x)^{g(x)}$ cuando $x$ tiende a
$a$ es indeterminado en el sentido de que, dependiendo de las funciones $f$
y $g$, el resultado puede ser cualquier número mayor o igual a $0$,
$\infty$, o incluso el límite puede no existir.

De la propia definición, se obtienen por inducción las siguientes
propiedades de las potencias:

Para todo $(a, m, n) \in \nset^* \times \nset \times \nset$,

$$ a^m \cdot a^n = a^{m+n} $$

Para todo $(a, m, n) \in \nset^* \times \nset \times \nset$,

$$ (a^n)^m = a^{nm} $$

Para todo $(a, b, n) \in \nset^* \times \nset^* \times \nset$,

$$ a^n b^n = (ab)^n $$

% TODO Hacer las demostraciones.






\subsection{Ordenación}

Definición. Relación Menor o Igual Que. Dados $m, n \in \nset$, se define la
relación ``menor o igual que'', $\leq$, mediante:

$$ m \leq n \ \text{si y solo si existe} \ p \in \nset \ \text{tal que}\  m
+ p = n $$

Si $m \leq n$, se dice que ``$m$ es menor o igual que $n$''.

Si $m \leq n$ y $m \neq n$, se dice que $m$ es ``menor que'' $n$, y se
escribe $m < n$. Hay quien prefiere ``estrictamente menor que'', para
recalcar que se excluye el caso $m = n$. Observemos que dados dos elementos
$m, n \in \nset$ se tiene:

$$ m < n \ \text{si y solo si existe} \ p \in \nset^* \ \text{tal que} \ m +
p = n $$

Además, se obtiene la relación siguiente:

$$ m < n \ \text{si y solo si} \ m + 1 \leq n $$

En efecto, si $m < n$, entonces existe $p \in \nset$ tal que $m + p = n$.
Además, $p \neq 0$, pues, si $p = 0$, entonces $n = m$. Luego $p$ es el
sucesor de algún número natural $r$. En consecuencia, $m + r + 1 = n$, esto
es, $(m + 1) + r = n$. Por tanto, $m + 1 \leq n$. El recíproco es inmediato
pues $m < m + 1$.

\iffalse
Las relaciones ``mayor o igual que'', $\geq$, y ``mayor que'', o
``estrictamente mayor que'', $>$, se definen mediante:

\begin{align*}
  n \geq m, \ \text{respectivamente} \ n > m, \ \text{si y solo si} \ m \leq
    n \\
  n \geq m, \ \text{respectivamente} \ n > m, \ \text{si y solo si} \ m \leq
    n \\
\end{align*}
, \ \text{respectivamente} \ m < n $$
\fi

Proposición. La relación $\leq$ es una relación de orden total en $\nset$
compatible con la suma y producto de números naturales; es decir, para todo
$m, n, p \in \nset$, se tiene:

$$ \text{si} \ m \leq n, \ \text{entonces} \ m + p \leq n + p \ \text{y} \
mp \leq np $$

Demostración. Veamos primero que la relación $\leq$ es una relación de
orden.

Es reflexiva, pues $n \leq n$, ya que $n + 0 = n$ para todo $n \in \nset$,
por la definición de la suma.

Es antisimétrica. Si $n \leq m$ y $m \leq n$, entonces existen $p, q \in
\nset$ tales que $n + p = m$ y $m + q = n$. Al sustituir $n$ en la primera
igualdad se obtiene $(m + q) + p = m$. Aplicando a esta expresión la
propiedad asociativa de la suma, se tiene que $m + (q + p) = m + 0$ y, por
la propiedad cancelativa de la suma, se obtiene que $q + p = 0$. De la
observación 3 de la definición 5.1 se deduce que $p = q = 0$. En
consecuencia, $n = m$.

Es transitiva. Si $n \leq m$ y $m \leq r$ entonces existen $p, q \in \nset$
tales que $n + p = m$ y $m + q = r$. Al sustituir $m$ en la segunda igualdad
se obtiene $(n + p) + q = r$, esto es, $n + (p + q) = r$, y, como $p + q \in
\nset$, se tiene que $n \leq r$.

El orden $\leq$ es total. Hay que ver que para todo $m, n \in \nset$ se
cumple que $m \leq n$ o $n \leq m$. Lo demostramos por inducción sobre $n$
para cualquier $m \in \nset$.

La propiedad es cierta para $n = 0$ pues de $0 + m = m$ se deduce que $0
\leq m$.

Supongamos que la propiedad es cierta para $n$, esto es, que, para cada $m
\in \nset$, $m \leq n$ o $n \leq m$. Veamos que la propiedad es cierta para
$s(n) = n + 1$, esto es, $m \leq n + 1$ o $n + 1 \leq m$.

En efecto, si $m \leq n$, como $n \leq n + 1$, de la propiedad transitiva se
tiene $m \leq n + 1$.

Si $n \leq m$, entonces $n = m$ o $n < m$. En el primer caso, aplicando el
caso anterior o directamente, se obtiene que $m \leq n + 1$. Si $n < m$,
entonces $n + 1 \leq m$.

Finalmente el orden es compatible con las operaciones. En efecto, sean $m,
n, p \in \nset$ y supongamos, sin pérdida de generalidad, que $m \leq n$.
Sea $q \in \nset$ tal que $m + q = n$. Entonces, por un lado, $m + q + p = n
+ p$, esto es, $(m + p) + q = n + p$ y, por tanto, $m + p \leq n + p$. Por
otro lado, $(m + q)p = np$, es decir, $mp + qp = np$ y, en consecuencia, $mp
\leq np$.

% TODO Comprobar si en lo anterior, para el producto, tiene que ser $p \neq
% 0$.

Observemos que de la definición de orden que hemos dado se deduce que
$\nset$ no tiene máximo.

Finalmente estudiamos tres propiedades del orden definido en $\nset$. Son
propiedades específicas del orden de $\nset$ que no serán ciertas ni en
$\qset$ ni en $\rset$ con el orden usual. La primera de ellas es la
existencia de intervalos abiertos de $\nset$ con extremos distintos que no
tienen elementos. En concreto:

Propiedad. El intervalo abierto $(n, n + 1)_{\nset}$ es vacío para todo $n
\in \nset$.

Demostración. Recordemos que $(n, n + 1)_{\nset} = \{ p \in \nset \st n < p
< n + 1 \}$. Razonamos por reducción al absurdo. Sea $p \in \nset$ tal que
$n < p < n + 1$. De $n < p$ se obtiene que $n + 1 \leq p$ y por tanto, $n +
1 \leq p < n + 1$, es decir, $n + 1 < n + 1$, que es una contradicción.

Propiedad. El conjunto $\nset$ con la relación $\leq$ es un conjunto bien
ordenado.

Demostración. Tenemos que demostrar que todo subconjunto de $\nset$ no vacío
tiene mínimo. Por reducción al absurdo, supongamos que existe $A \subseteq
\nset$ sin elemento mínimo. Veamos que $A = \emptyset$. Sea $U$ el conjunto
de cotas inferiores de $A$:

$$ U = \{ n \in \nset \st n \leq a, \ \text{para todo} \ a \in A \} $$

Se tiene:

$U \cap A = \emptyset$, pues, si existiera $n \in U \cap A$, entonces $n$
sería una cota inferior de $A$ y al mismo tiempo un elemento de $A$. Por
tanto, sería un mínimo de $A$.

$U = \nset$. En efecto, se procede por inducción:

$0 \in U$ pues $0 \leq m$ para todo $m \in \nset$, y en particular, para
todo $m \in A$.

Supongamos que $n \in U$ y veamos que $n + 1 \in U$. En efecto, si $n \in
U$, entonces $n \leq a$ para todo $a \in A$. Además, como $n \notin A$, se
puede asegurar que $n < a$ para todo $a \in A$. Por tanto, para todo $a \in
A$ se cumple que $n + 1 \leq a$ y, en consecuencia, $n + 1 \in U$.

Propiedad. En $\nset$, todo subconjunto no vacío y acotado superiormente
tiene máximo.

Demostración. Basta observar que, si $\emptyset \neq A \subseteq \nset$ está
acotado superiormente, entonces el conjunto $U$ de las cotas superiores de
$A$ es un conjunto no vacío y, por tanto, tiene mínimo, $m = \min(U)$.

Veamos que $m \in A$. Razonando por reducción al absurdo, si $m \notin A$,
como $m$ es cota superior de $A$, tendríamos que $a < m$ para todo $a \in
A$. Podemos deducir dos cosas: $a + 1 \leq m$ para todo $a \in A$ y $m \neq
0$, \ pues $A \neq \emptyset$.

De ii) se deduce que existe un número natural $n$ tal que $m = n + 1$.

Al sustituir en i), se obtiene $a + 1 \leq n + 1$. Es decir, para todo $a
\in A$ existe $p \in \nset$ tal que $(a + 1) + p = n + 1$. De las
propiedades asociativa, conmutativa y cancelativa de la suma se obtiene que
$a + p = n$ y, por tanto, $a \leq n$ para todo $a \in A$. Por consiguiente,
$n$ es una cota superior de $A$. Pero $n < n + 1 = m$ y, por tanto, $m$ no
es el mínimo de las cotas superiores de $A$, que es una contradicción. Así
pues, $m \in A$ y, por tanto, $m$ es el máximo de $A$.
















