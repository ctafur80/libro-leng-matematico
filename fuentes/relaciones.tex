


\iffalse
En este capítulo, nos centraremos en primer lugar en las relaciones desde el
punto de vista de la teoría de conjuntos, estudiando las propiedades que se
les pueden atribuir. Tratamos, en particular, las relaciones de equivalencia
y las de orden. Las relaciones de equivalencia en un conjunto permiten
clasificar los elementos del conjunto, creando una partición del propio
conjunto. La identificación de los elementos de una misma clase genera un
nuevo conjunto, el conjunto cociente. Este concepto es de gran utilidad en
casi todas las ramas de las matemáticas. Las relaciones de orden también
aparecen en todas partes: desde la ordenación de números hasta la ordenación
de palabras para disponerlas en un diccionario (orden lexicográfico).
Estudiaremos los elementos más importantes que se definen en todo conjunto
ordenado con el ánimo de que el lector se familiarice con la manipulación de
conjnntos ordenados.

Por otro lado, y dentro del marco de las relaciones binarias, estudiaremos
las aplicaciones entre conjuntos. Son las relaciones para las que la imagen
de cada elemento del conjunto inicial es un único elemento del conjunto
final. Estudiaremos la composición de aplicaciones y los conceptos de
aplicación inyectiva, sobreyectiva y biyectiva. La noción de biyección
conduce de manera natural al concepto de cardinal.
\fi





