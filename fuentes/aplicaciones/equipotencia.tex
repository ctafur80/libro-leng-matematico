

La existencia de una biyección entre dos conjuntos $A$ y $B$ permite
emparejar cada elemento de $A$ con un único elemento de $B$, y podemos decir
de manera coloquial que, si $A$ y $B$ tienen un número finito de elementos,
entonces el conjunto $A$ tiene tantos elementos como el conjunto $B$.

Definición. Equipotencia de conjuntos. Dos conjuntos $A$ y $B$ se dicen
\emph{equipotentes} si y solo si existe una biyección entre ellos, y se
denota $A \equiv B$.

La equipotencia de conjuntos satisface las propiedades siguientes:

\begin{enumerate}
  \item Reflexiva. $A \equiv A$ puesto que la aplicación identidad es una
    biyección.

  \item Simétrica. Si $A \equiv B$, entonces existe $f \in \mathcal{F}(A,
    B)$ biyectiva. Como la aplicación inversa $f^{-1} \in \mathcal{F}(B, A)$
    es biyectiva, se deduce que $B \equiv A$.

  \item Transitiva. Si $A \equiv B$ y $B \equiv C$, entonces existen dos
    biyecciones $f \in \mathcal{F}(A, B)$ y $g \in \mathcal{F}(B, C)$.
    Entonces la aplicación $g \circ f \in \mathcal{F}(A, C)$ es una
    biyección, por tanto $A \equiv C$.
\end{enumerate}

Diremos que es una ``relación de equivalencia'' entre conjuntos. Ponemos
comillas porque en las relaciones de equivalencia definidas en la sección
TKTK, el marco de la relación es un conjunto. En este caso el marco es la
colección de todos los conjuntos que no es un conjunto.

Esta ``relación'' de equivalencia nos permite hacer una clasificación del
conjunto de todos los conjuntos en algo así como sus ``tamaños'', propiedad
que llamamos \emph{cardinalidad} del conjunto. Se tendrían entonces:

\begin{description}
  \item[Cardinal 0] es la colección de todos los conjuntos equipotentes con
    $\emptyset$, y se representa con el símbolo del número $0$.

  \item [Cardinal $n$] es la colección de todos los conjuntos que son
    equipotentes con $\{1, \ldots, n\} \subseteq \nset^*$, y se representa
    con el símbolo del número $n$.

  \item [Cardinal de $\nset$ o $\aleph_0$] es la colección de todos los
    conjuntos equipotentes con $\nset$, y se representa por el símbolo
    $\aleph_0$, leído ``aleph cero''.\footnotemark

  \item [Cardinal de $\rset$ o $\mathfrak{c}$] es la colección de todos los
    conjuntos equipotentes con $\rset$, y se representa con
    $\mathfrak{c}$.\footnotemark
\end{description}

\footnotetext{Esa letra $\aleph$, llamada \emph{aleph}, es del alfabeto
hebreo.}

\footnotetext{Advierta que esta letra no es una \emph{c}, aunque se le
parezca.}

En general, dado un conjunto $A$, llamaremos \emph{número cardinal} de $A$,
$\text{card}(A)$ o $\text{card} \, A$, a la colección de todos los conjuntos
equipotentes con el conjunto $A$. La mayoría de las veces lo llamaremos
simplemente \emph{cardinal}.

Proposición. Decimos que el conjunto $A$ tiene $n$ elementos siendo $n \in
\nset^*$ si y solo si

$$ \text{card}(A) = n $$

En conjuntos finitos, el concepto de número cardinal está intuitivamente
asociado con el recuento del número de elementos del conjunto, es decir, con
lo que comúnmente llamamos \emph{su tamaño}.

Sean un conjunto $A$ que contiene $n$ elementos y un conjunto $B$ que tiene
$m$ elementos, es decir, $\card(A) = n$ y $\card(B) = m$. Las siguientes
observaciones son intuitivas y posiblemente el lector ya las conoce. Se
estudiarán con más rigor y precisión en el capítulo~\ref{ch:naturales}.
Estas suelen explicarse por conceptos que se estudian en la combinatoria y
que no se han explicado en esta asignatura. En cualquier caso, si no tiene
una buena base de combinatoria, es posible que pueda entenderlas
intuitivamente.

\begin{itemize}
  \item Si $n < m$, entonces no existen aplicaciones sobreyectivas de $A$ a
    $B$ puesto que siempre existirá un elemento de $B$ que no estará
    relacionado con ningún elemento de $A$.

  \item Si $n \leq m$, entonces se pueden definir tantas aplicaciones
    inyectivas como variaciones \footnotemark (sin repetición) hay de $m$
    elementos tomados de $n$ en $n$, $V(m, n)$. Es decir, el número de
    aplicaciones inyectivas distintas es

    $$ \frac{m!}{(m-n)!} $$

    \noindent o, si lo prefiere, de $m(m-1) \cdots (m-n+1)$.

  \item Si $n > m$, entonces no existen aplicaciones inyectivas de $A$ a
    $B$, puesto que para definir la imagen de todos los elementos de $A$ se
    tiene que repetir alguna imagen. Esto se podría explicar por otra
    propiedad de la combinatoria, conocida como el Principio de la
    Distribución, o Del Palomar.

  \item Si $n = m$, entonces se pueden definir tantas aplicaciones
    biyectivas como permutaciones de $n$ elementos distintos hay, es decir,
    hay $n!$ biyecciones distintas de $A$ a $B$.

  \item Si $n \neq m$, entonces no existen aplicaciones biyectivas entre $A$
    y $B$, puesto que $n < m$ o $n > m$, y esto impide, respectivamente, ser
    sobreyectiva o ser inyectiva.
\end{itemize}

\footnotetext{También reciben el nombre de \emph{permutaciones parciales},
entre otros.}

Una aplicación entre los conjuntos $A$ y $B$ queda determinada al precisar
la imagen de cada elemento de $A$. Si el conjunto $A$ tiene $n$ elementos y
el conjunto $B$ tiene $m$ elementos, entonces cada aplicación es una
variación con repetición de los $m$ elementos de $B$ tomados de $n$ en $n$.
Por lo tanto, el conjunto de todas las aplicaciones de $A$ a $B$,
$\mathcal{F}(A, B)$, tiene $m^n$ aplicaciones distintas. Dicho de forma
simbólica,

$$ \card(\mathcal{F}(A, B)) = \card(B)^{\card(A)} $$

Definición.

\begin{itemize}
  \item Un conjunto $A$ es \emph{finito} si existe $n \in \nset$ tal que
    $\text{card}(A) = n$.

  \item Un conjunto $A$ es \emph{infinito} si no es un conjunto finito.

  \item Un conjunto $A$ es un \emph{conjunto numerable} si existe una
    biyección de los números naturales al conjunto, y se indica escribiendo
    $\text{card}(A) = \aleph_0$. A veces, lo califican de \emph{infinito
    numerable}.
\end{itemize}

Un error que se comete a veces es decir que un conjunto es \emph{contable},
en lugar de \emph{numerable}.

Ejemplo. Identificación de conjuntos. Sean dos conjuntos $A$ y $B$ tales que
existe una biyección $f$ entre ambos, es decir $A \equiv B$. Entonces a cada
subconjunto $A_1$ de $A$ le corresponde un subconjunto $f(A_1)$ y solo uno
de $B$, puesto que $f^{-1} \circ f(A_1) = A_1$.

En este caso, a cualquier operación de conjuntos que se realice en $A$ le
corresponde la operación análoga en $B$ con las imágenes de los elementos de
$A$. En algunos casos, operar en $B$ resulta más cómodo que en $A$ debido a
la naturaleza de los elementos del conjunto $B$. En estos casos tan solo ha
de operarse en $B$ y posteriormente aplicar la biyección $f^{-1}$.

Un ejemplo de biyección es la identificación que se produce entre los
conjuntos $\rset^2 \times \rset$ con $\rset^3$ con la biyección $f((x, y),
z) = (x, y, z)$, o en general, entre los conjuntos $\rset^n \times \rset^m$
y $\rset^{n+m}$ mediante la aplicación:

$$ f((x_1, \ldots, x_n), (x_{n+1}, \ldots, x_{n+m})) = (x_1, \ldots,
x_{n+m}). $$

Otro ejemplo es la identificación del conjunto de vectores libres del plano
o del espacio con el conjunto $\rset^2$ o $\rset^3$ mediante las coordenadas
de un vector respecto a una base.

En general, este tipo de identificaciones es muy útil si la biyección
conserva las estructuras algebraicas de los conjuntos, cuestión que excede
los contenidos de este capítulo y que se tratará en capítulos posteriores.

Ejemplo. \emph{Inmersión} de conjuntos. Dados dos conjuntos $A$ y $B$ tales
que existe una inyección $f$ entre ambos, entonces resulta que $f$ es una
biyección entre $A$ y $f(A)$, es decir $A \equiv f(A)$. Algunas veces se
identifica el conjunto $A$ con $f(A)$ y en lugar de considerar los elementos
de $A$, se consideran los de $f(A)$.

Por ejemplo, la identificación entre el conjunto $\nset^*$ y el conjunto
$\zset^+ = \{z \in \zset \st z \geq 1\}$ mediante la aplicación que al
número natural $n$ positivo le corresponde el número entero (clase de
equivalencia, véase el ejemplo 3.8) que contiene al par $(n, 0)$.

Otro ejemplo es la identificación de $\zset$ con el subconjunto de números
racionales:

$$ \left\{\frac{z}{1} \st z \in \zset\right\} \subseteq \qset $$

En general, este tipo de inmersiones es muy útil si la inyección conserva
las estructuras algebraicas de los conjuntos, cuestión que excede los
contenidos de este capítulo.























