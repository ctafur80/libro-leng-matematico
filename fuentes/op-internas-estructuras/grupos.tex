




Definición. Grupo. Sea $G$ un conjunto y $\star$ una operación interna en
$G$. Se dice que el par $(G, \star)$ es un \emph{grupo} (\emph{group}) si se
satisfacen las propiedades siguientes:

\begin{itemize}
  \item La operación $\star$ es asociativa.
  \item Existe elemento neutro de $\star$ en $G$.
  \item Para todo elemento $a \in G$, existe en $G$ el elemento simétrico de
    $a$ respecto de $\star$.
\end{itemize}

Otra forma de decir que $(G, \star)$ es un grupo es decir que ``$(G, \star)$
tiene estructura de grupo''. Otras formas de expresar esto es decir que
``$G$ es un grupo respecto de $\star$'', o incluso, si el contexto es
suficientemente claro respecto de la operación considerada, que ``$G$ es un
grupo''.

Si además la operación $\star$ es conmutativa, se dice que el grupo es
\emph{conmutativo} (\emph{commutative}) o \emph{abeliano} (\emph{abelian}).

En cuanto a la terminología, por comodidad, al hablar de las estructuras, a
veces seremos algo vagos y nos referiremos al conjunto hablando de la
estructura, sobrentendiendo que TKTK. Así, por ejemplo, a veces diremos ``el
grupo $G$'', en lugar de ``el grupo $(G, \star)$'', que es lo que se debería
decir.

Existen ciertas convenciones que se suelen seguir respecto a los operadores
internos. En realidad, esto es simplemente una cuestión de notación y
terminología, con lo que no afecta a los resultados TKTK. Es decir, con
cualquier otra terminología y notación se podría expresar lo mismo y llegar
a las mismas conclusiones TKTK. En cualquier caso, conviene conocer estas
``costumbres''. Son las siguientes.




\subsection{Notación aditiva}

Cuando la operación de un grupo se representa con el símbolo `$+$', el grupo
se llama \emph{aditivo}. También, a su elemento neutro se le llama
\emph{elemento nulo}, o \emph{cero}, y se denota por `$0$'. El elemento
simétrico de $a$ se denotará entonces por `${-a}$' y se denomina
\emph{elemento opuesto}.

Se crea así también la operación resta, que no sería más que la suma del
opuesto, denotada como $a - b$; es decir, esta expresión expresa lo mismo
que $a + ({-b})$. Advierta que el signo \emph{menos}, `$-$', lo acabamos de
usar de dos formas distintas. En la expresión `${-a}$' o `${-b}$', es un
operador unario prefijo que representa al elemento simétrico de $a$. Por
otro lado, en `$a - b$' es un operador binario infijo TKTK.

La expresión $na$ indica la suma de $n$ veces $a$, siendo $n \in \nset^*$.
La propiedad asociativa de la operación $+$ hace que $a + a + \cdots + a$
permanezca invariable cuando se reagrupan de manera arbitraria los factores,
y se escribe:

$$ na = \overbrace{a + a + \cdots + a}^{\text{$n$ veces}} $$

Al elemento neutro, tal y como hemos dicho, hay quien lo llama
\emph{elemento identidad}. En el caso de la notación aditiva, a veces lo
llaman la \emph{identidad aditiva} (\emph{additive identity}).





\subsection{Notación multiplicativa}

Cuando la operación se representa con el símbolo `$\cdot$', el grupo se dice
\emph{multiplicativo}, y a su elemento neutro se denota por `$1$' y se llama
\emph{unidad}. El elemento simétrico de $a$, que se denota por `$a'$',
se llama \emph{elemento inverso} de $a$.

En realidad, hay que tener cuidado con esta última teminología, ya que, en
muchos textos en inglés llaman \emph{inverse element} a lo que aquí llamamos
\emph{elemento simétrico}. Al elemento simétrico de un producto en inglés lo
suelen llamar \emph{reciprocal element}.

Análogamente al caso aditivo, la expresión $a^n$ indica el producto de $n$
veces $a$, siendo $n \in \nset^*$. Además, $a \cdot a \cdot \cdots \cdot a$
permanece invariable cuando se reagrupan de manera arbitraria los factores,
y se escribe:

$$ a^n = \overbrace{a \cdot a \cdot \cdots \cdot a}^{\text{$n$ veces}} $$

Las notaciones $\frac{1}{a}$ o $1/a$ por $a'$ se utilizan exclusivamente
para los números. De hecho, la notación $\frac{b}{a}$ sería confusa si la
operación no es conmutativa, ya que a priori $\frac{1}{a} b$ y $b
\frac{1}{a}$ pueden ser distintos. Así, por ejemplo, si $A$ es una matriz
cuadrada invertible de orden 2, su inversa se denota por $A'$ y nunca se
utiliza la notación $\frac{1}{A}$, ya que el producto de matrices no es
conmutativo.

Ejemplo. Veamos algunos grupos que ya conocemos. Los conjuntos $\zset$,
$\qset$, $\rset$ y $\cset$ son grupos conmutativos respecto de la suma, cosa
que demostraremos en los capítulos posteriores. Los conjuntos $\qset^*$,
$\rset^*$ y $\cset^*$, junto con la operación producto, también son grupos.
El conjunto de matrices de orden $n \times m$ junto con su suma es un grupo
conmutativo. El conjunto de matrices cuadradas invertibles de orden $n$ es
un grupo no conmutativo respecto del producto. El conjunto
$\mathcal{B}(\Omega)$ de las aplicaciones biyectivas de un conjunto $\Omega$
en sí mismo es un grupo no conmutativo respecto de la composición de
aplicaciones.

Ejercicio. Demuestre que el conjunto de las partes de un conjunto $\Omega$
junto con la diferencia simétrica de conjuntos, $(\powset(\Omega),
\triangle)$, es un grupo conmutativo.

Solución. Recordemos que si $A, B \in \powset(\Omega)$, entonces 

$$ A \triangle B = (A \setminus B) \cup (B \setminus A) = (A \cap
\overline{B}) \cup (\overline{A} \cap B) $$

Es evidente que la operación $\triangle$ es interna y conmutativa en
$\powset(\Omega)$. Veamos si es asociativa. Sean $A, B, C \in
\powset(\Omega)$. Se cumple:

\iffalse
\begin{align*}
  (A \triangle B) \triangle C
    &= [(A \cap \overline{B}) \cup (\overline{A} \cap B)] \triangle C \\

    &= \left\{[(A \cap \overline{B}) \cup (\overline{A} \cap B)] \cap
      \overline{C}\right\} \cup [\overline{(A \cap \overline{B}) \cup
      (\overline{A} \cap B)} \cap C] \\

    &= [(A \cap \overline{B} \cap \overline{C}) \cup (\overline{A} \cap B
      \cap \overline{C})] \cup \left\{[\overline{(A \cap \overline{B}) \cap \
      \overline{(\overline{A} \cap B)}] \cap C\right\} \\

    &= [(A \cap \overline{B} \cap \overline{C}) \cup (\overline{A} \cap B
      \cap \overline{C})] \cup [(\overline{A} \cup B) \cap (A \cup
      \overline{B})] \cap C \\
\end{align*}
\fi

Vamos a desarrollar de forma independiente la parte derecha de esta
expresión.

\begin{align*}
  (A \triangle B) \triangle C
    &= \left\{[(\overline{A} \cup B) \cap A] \cup [(\overline{A} \cup B)
      \cap \overline{B}]\right\} \cap C \\
    &= \left\{[(\overline{A} \cap A) \cup (B \cap A)] \cup [(\overline{A}
      \cap \overline{B}) \cup (B \cup \overline{B})]\right\} \cap C \\
    &= \left\{[\emptyset \cup (A \cap B)] \cup [(\overline{A} \cap
      \overline{B}) \cup \emptyset]\right\} \cap C \\
    &= [(A \cap B) \cup (\overline{A} \cap \overline{B})] \cap C \\
    &= (A \cap B \cap C) \cup (\overline{A} \cap \overline{B} \cap C) \\
\end{align*}

\noindent Se tiene, entonces,

$$ (A \triangle B) \triangle C = (A \cap \overline{B} \cap \overline{C})
\cup (\overline{A} \cap B \cap \overline{C}) \cup (A \cap \overline{B} \cap
C) \cup (\overline{A} \cap B \cap C) $$

Utilizando la fórmula anterior, junto con la propiedad conmutativa de
$\triangle$ y las propiedades conmutativa y asociativa de la unión y la
intersección, se deduce que

\begin{align*}
  A \triangle (B \triangle C) &= (B \triangle C) \triangle A \\
    &= (B \cap \overline{C} \cap A) \cup (\overline{B} \cap C \cap
      \overline{A}) \cup (\overline{B} \cap \overline{C} \cap A) \cup (B
      \cap C \cap A) \\
    &= (A \triangle B) \triangle C \\
\end{align*}

El elemento neutro de este grupo es el conjunto vacío, $\emptyset$, pues $A
\triangle \emptyset = \emptyset \triangle A = A$ para todo $A$, y el
elemento simétrico de un elemento arbitrario $A \in \powset(\Omega)$ es el
propio $A$, pues $A \triangle A = \emptyset$.

Ejemplo. Vamos a ver ahora casos en los que no se tienen grupos.

$(\nset, +)$ y $(\rset, \cdot)$. Sólo el $0$ tiene elemento opuesto en el
primer caso, mientras que, en el segundo, el $0$ no tiene inverso.

En $(\powset(\Omega), \cap)$ ningún elemento, salvo el elemento neutro,
tiene elemento simétrico, pues, si $A \cap B = \Omega$, se tiene
necesariamente que $A = B = \Omega$; recuerde que en dicho grupo el elemento
neutro era $\Omega$. En $(\powset(\Omega), \cup)$ sucede algo análogo. Si $A
\cup B = \emptyset$, entonces $A = B = \emptyset$; recuerde que en este
grupo el elemento neutro era $\emptyset$.

El conjunto de matrices cuadradas de orden $n$ con la multiplicación de
matrices no es un grupo pues las matrices cuyo determinante es cero no
tienen inversa.

Proposición. Propiedades en un grupo.

Sea $(G, \star)$ un grupo y unos elementos arbitrarios de este $a, b, c \in
G$. Se tiene:

\begin{enumerate}
  % TODO Quizás ampliarla para añadir en el otro sentido.
  \item Si $a \star b = a \star c$, entonces $b = c$. Se la conoce como
    \emph{propiedad cancelativa}.
  \item Existe un único $x \in G$ tal que $a \star x = b$.
  \item Si $a'$ y $b'$ son los elementos simétricos de $a$ y $b$,
    respectivamente, entonces $(a \star b)' = b' \star a'$.
\end{enumerate}

Demostración.

1. Basta operar por la izquierda con el elemento simétrico de $a$:

\begin{align*}
  a \star b &= a \star c \\
  a' \star (a \star b) &= a' \star (a \star c) \\
  (a' \star a) \star b &= (a' \star a) \star c \\
  e \star b &= e \star c \\
  b &= c \\
\end{align*}

\noindent Las manipulaciones que hemos hecho en la expresión para llegar a
este resultado se justifican por ser condiciones que han de darse para que
$(G, \star)$ sea un grupo; por ejemplo, hemos usado la asociativa.

Como es fácil de comprobar, se daría también en el otro sentido, es decir,
de $b \star a = c \star a$ se deduce también que $b = c$.

2. Se sigue la misma estrategia que en la anterior.

\begin{align*}
  b &= a \star x \\
  a' \star b &= a' \star (a \star x) = (a' \star a) \star x = e \star x = x
\end{align*}

\noindent Esto demuestra que es único, ya que es la imagen del par $(a, b')
\in G \times G$ en la aplicación que es la operación interna. Ojo, esto no
quiere decir que, dados $a, b, c, d \in G$, los elementos $x, y \in G$ tales
que $a \star x = b$ y $c \star y = d$ sean distintos necesariamente.

3. Basta observar que

$$ (b' \star a') \star (a \star b) = b' \star (a' \star a) \star b = b'
\star e \star b = b' \star (e \star b) = b' \star b = e = (a \star b)' \star
(a \star b) $$

\noindent y, aplicando la propiedad cancelativa (la del punto 1), se tiene
que $(a \star b)' = b' \star a'$. Análogamente, se tiene que

$$ (a \star b) \star (b' \star a') = e $$

\noindent pero no lo demostraremos.

Observaciones. En las tres propiedades ha de observarse que el orden en el
que se disponen los elementos es importante cuando el grupo no es
conmutativo. El inverso de $a \star b$ es $b' \star a'$, que no tiene por
qué coincidir con $a' \star b'$. También, en la segunda propiedad, se ha
obtenido el elemento $x = a' \star b$ tal que $a \star x = b$, que puede ser
diferente de $b \star a'$.

La propiedad cancelativa indica que en un grupo $(G, \star)$ la aplicación

$$
\begin{array}{llll}
  f_a:  & G & \longrightarrow & G \\
        & x & \longmapsto     & f_a(x) = a \star x \\
\end{array}
$$

\noindent es inyectiva.

Para demostrarlo, tomemos dos valores arbitrarios del conjunto imagen de la
aplicación. Serán $y_1, y_2 \in G$ tales que existen $x_1, x_2 \in G$ de tal
forma que $y_1 = f_a(x_1)$ y $y_2 = f_a(x_2)$. Entonces,

\begin{align*}
  y_1 &= y_2 \\
  f_a(x_1) &= f_a(x_2) \\
  a \star x_1 &= a \star x_2
\end{align*}

\noindent y, por la propiedad cancelativa, de esto se deduce que $x_1 =
x_2$. Si se fija, este hecho es lo mismo que afirmar que la función $f_a$ es
inyectiva.

Ejemplo. Las siguientes tablas representan operaciones internas. Las dos
primeras tablas representan dos operaciones, $\odot$ y $\star$, en el
conjunto $G = \{e, a\}$ mientras que la tercera tabla representa la
operación $\star$ en el conjunto $G' = \{e, a, b, c\}$.

$$
  \begin{array}{c|cc}
    \odot & e & a \\
    \hline
    e & e & a \\
    a & a & a \\
  \end{array}
  \quad
  \begin{array}{c|cc}
    \star & e & a \\
    \hline
    e & e & a \\
    a & a & e \\
  \end{array}
  \quad
  \begin{array}{c|cccc}
    \star & e & a & b & c \\
    \hline
    e & e & a & b & c \\
    a & a & e & c & b \\
    b & b & c & e & a \\
    c & c & b & a & e \\
  \end{array}
$$

Así, por ejemplo, el elemento que está situado en la intersección de la
línea de $b$ con la columna de $c$ en la tercera tabla es $b \star c$, y, en
este caso, $b \star c = a$. Veamos si definen o no una estructura de grupo
conmutativo.

En los tres casos, $e$ es el elemento neutro pues la fila y columna de $e$
dejan invariante la primera fila y la primera columna respectivamente.

También se observa a primera vista que las tres operaciones son conmutativas
pues las tablas son simétricas respecto de la diagonal principal (la que
baja de izquierda a derecha).

En el primer caso, el elemento $a$ no tiene simétrico, pues no existe ningún
elemento $a'$ tal que $a \odot a' = e$. Aunque sí tiene simétrico el
elemento $e$, que sería él mismo, para que se dé esta propiedad, han de
tenerlos todos los elementos del conjunto $G$, con lo que podemos afirmar
que $(G, \odot)$ no es un grupo.

En el segundo caso, el simétrico de $a$ es el propio $a$, y, el de $e$, $e$.
Entonces, todos los elementos de $G$ tienen simétrico, con lo que se cumple
la propiedad del elemento simétrico.

En el tercer ejemplo los elementos simétricos de $a$, $b$ y $c$ son
respectivamente los propios $a$, $b$ y $c$, respectivamente.

La propiedad asociativa en el segundo caso se cumple comprobando que $x
\star (y \star z) = (x \star y) \star z$ en todos los casos posibles de las
variables $x, y, z \in G$. Esto es fácil de ver para las dos primeras
tablas, pues, claramente se cumple si uno de los tres elementos es el
elemento neutro $e$ por tanto sólo hay que comprobar que $a \star (a \star
a) = (a \star a) \star a$ que se cumple pues la operación es conmutativa.

Esta propiedad es bastante más tediosa de demostrar para el tercer cuadro.
Hay que comprobar que 

\iffalse
% TODO Hacerlo más elegante
\begin{align*}
  a \star (b \star c) &= (a \star b) \star c, \quad a \star (c \star b) = (a
    \star c) \star b, \quad a \star ((a \star c)) = ((a \star a) \star c) \\

  a \star (c \star a) &= ((a \star c) \star a), \quad a \star (a \star b) =
    ((a \star a) \star b), \quad a \star (b \star c) = ((b \star b) \star c)
    \\

  b \star (c \star b) &= ((b \star c) \star b), \quad b \star (b \star a) =
    ((b \star b) \star a), \quad b \star (a \star b) = ((b \star a) \star b)
    \\

  c \star (a \star c) = ((c \star a) \star c), \quad c \star ((c \star b)) =
    ((c \star c) \star b), \quad c \star (b \star c) = ((c \star b) \star c)
    \\
\end{align*}
\fi

Todos los demás casos se deducirían de los casos anteriores, la propiedad
conmutativa y la del elemento neutro.

Al comprobar si un conjunto junto con una operación son un grupo, o
cualquiera de las estructuras que vemos más adelante, algo que nos puede
facilitar el trabajo es que este cumpla la propiedad conmutativa. Así, si la
cumple, nos ahorraríamos el trabajo a la mitad, peus TKTK.

El grupo que corresponde a la tercera de las tablas se denomina \emph{grupo
de Klein}, y tiene una representación geométrica en el que $e$ es la
identidad en el espacio tridimensional y $a$, $b$ y $c$ representan las
simetrías axiales de eje $O_x$, $O_y$ y $O_z$. La operación $\star$ es la
composición de movimientos.

Ahora, vamos a ver una subestructura de grupo. Concretamente, la de
subgrupo. La norma que se suele seguir es llamar con el mismo nombre
precedido del prefijo \emph{sub} a la subestructura que es igual que la que
la engloba, es decir, TKTK. Tal y como veremos, ese prefijo solo aporta
información contextual TKTK.

Tal y como verá en otras asignaturas posteriores de álgebra abstracta, a
veces, la subestructura más interesante no es la subestructura principal
TKTK.





\subsection{Subgrupos}

Definición. Subgrupo. Dados un grupo $(G, \star)$ y un subconjunto $H$ de
$G$, $H \subseteq G$. Se dice que $(H, \star)$ es un \emph{subgrupo} de $(G,
\star)$ si cumple las condiciones para ser un grupo.

Hay quien pone la condición de que $H$ no sea vacío, pero en realidad no es
necesario, pues, en ese caso, no cumplirá las condiciones de grupo; por
ejemplo, la del elemento neutro. TKTK. No estoy muy seguro de esto.

Vuelvo a recalcar aquí que es bastante común ser vago cuando hablamos de
estructuras algebraicas, y es bastante normal que vea expresiones como ``el
subgrupo $H$'', en lugar de ``el subgrupo $(H, +)$''. Este ahorro en la
notación se hará cuando se sobrentienda a qué estructura nos referimos en
ese caso.

Para un grupo cualquiera $(G, \star)$, se tiene, en particular, que $(H,
\star)$ es un subgrupo de $(G, \star)$ siendo $H = \{e\}$ para $e$ el
elemento neutro del grupo $(G, \star)$. Veamos por qué. Ya que $e$ es el
elemento neutro, se cumple $e \star e = e$, que, ya que es el único elemento
de $H$, la operación $\star$ es interna en dicho conjunto. Veamos si se
cumple la propiedad asociativa:

$$ (e \star e) \star e = e \star e = e \star (e \star e) $$

Las propiedades del elemento neutro y el simétrico son evidentes
directamente del hecho de que $e \star e = e$.

También, se considera que todo grupo es un subgrupo de sí mismo; es decir,
$(G, \star)$ es un subgrupo del grupo $(G, \star)$, cosa que es evidente ya
que se trata de lo mismo y no hace falta, por tanto, volver a demostrar que
cumple las condiciones para ser un grupo.

Sabiendo que $(G, \star)$ es un grupo, para un subconjunto $H$ de $G$ no
habría que estudiar todas las condiciones de grupo para ver si $(H, \star)$
es un grupo. Observemos que, si para todos los elementos de $G$ se cumple la
propiedad asociativa mediante $\star$, entonces, también se cumplirá en
particular para todos los de elementos de $H$. Gracias a esto, nos
ahorraremos la comprobación de esta propiedad cuando tengamos que averiguar
si una estructura es un subgrupo de un grupo en concreto. Sucedería lo mismo
con la propiedad conmutativa, en caso de que se tratase de un grupo
conmutativo.

Por tanto, para averiguar si un par $(H, \star)$ es un subgrupo de $(G,
\star)$, en realidad no hay por qué hacer todas las comprobaciones de grupo,
como se explica en la proposición siguiente.

Proposición. Caracterización de Subgrupo. Sea un grupo $(G, \star)$ y $H$ un
subconjunto no vacío de $G$, $H \subseteq G$. Se tiene que $(H, \star)$ es
un subgrupo de $(G, \star)$ si y solo si, para todo $a, b \in H$, $a \star
b' \in H$ siendo $b'$ el elemento simétrico de $b$ en el grupo $(G, \star)$.

% TODO Significado de caracterización.

Creo que se podría quitar la condición de que $H$ no sea vacío. Si lo es, no
se cumplirá la condición y será, por tanto, vacuamente cierto. TKTK.

Advierta que, simplemente con esa condición, estamos garantizando que la
operación $\star$ sea interna también en $H$, así como que se cumplen las
propiedades del elemento neutro y del simétrico. Por cierto, en lo que
respecta al elemento neutro, en realidad no es necesario ``buscarlo'', sino
que, si el elemento neutro de $(G, \star)$ pertenece a $H$, será este el
elemento neutro de este último. Esto es evidente porque TKTK.

Respecto a la propiedad asociativa, tal y como dijimos, se cumple siempre,
ya que se cumple en $G$.

Esto también nos permite descartar rápidamente que $(H, \star)$ sea un
subgrupo del grupo $(G, \star)$ simplemente comprobando que el elemento
neutro de este último no se encuentra en el primero. Bueno, en realidad, con
cualquiera de las otras condiciones se pueden hacer descartes también.

Demostración. Es evidente que la condición es necesaria para que $(H,
\star)$ sea un subgrupo, ya que, si no se da alguna de las condiciones de
grupo, entonces no se dará $a \star b' \in H$ para todo $a, b \in H$. Es
decir, la condición $a \star b' \in H$ lleva implícitas a las demás. Así,
por ejemplo, si no se da la condición de que la operación $\star$ sea
interna en $H$, entonces ya no se cumpliría que $a \star b' \in H$ para todo
$a, b \in H$. Lo mismo, con las demás.

Veamos que es suficiente. Supongamos que para todo $a, b \in H$ se cumple $a
\star b' \in H$. De esto se deduce que $\star$ es una operación interna en
$H$ para todo $a, b \in H$.

Como $H \neq \emptyset$, existe un elemento $a$ en $H$, $a \in H$. Además,
como $H$ es un subconjunto de $G$, $H \subseteq G$, se tiene que $a \in G$.
Como $(G, \star)$ es un grupo, sabemos entonces que existen $a', e \in G$
para los que se cumple $a \star a' = e$, siendo, en dicho grupo, $e$ el
elemento neutro y $a'$ el inverso de $a$. Pero, ojo, aún no hemos demostrado
que esos elementos pertenezcan a $H$.

Si tomamos como caso particular de $b$ al elemento $a$, es decir, $b = a$,
tenemos, por la hipótesis de la que partimos, que $a \star a' \in H$.
Advierta que, de momento, $a'$ no tiene por qué pertenecer a $H$. Pero en
$(G, \star)$ se cumple que $a \star a' = e$, con lo que, por lo anterior, se
tiene que $e \in H$, siendo $e$ el elemento neutro de $(G, \star)$\ldots{} y
también el de $(H, \star)$, ya que su comportamiento será el de elemento
neutro tanto con los elementos de $G$ como con los de $H$ ya que estos son
casos particulares de los anteriores. Debido a que $e \in H$, se cumple la
condición del elemento neutro.

Tomando ahora al elemento $e$ como valor de la variable $a$, tenemos, por la
hipótesis, que $e \star b' \in H$. Además, como por $(G, \star)$ se tiene
que $e \star b' = b'$, se tendrá entonces que $b' \in H$, cumpliéndose así
la condición del elemento simétrico.

En cuanto a la propiedad asociativa, tal y como ya dijimos, en $H$ se
``hereda'' automáticamente de que se cumpla en $G$.

\iffalse
En consecuencia, $e = a \star a' \in H$. Luego el elemento neutro es un
elemento de $H$ y se cumple ii). En consecuencia, para todo $b \in H$ se
tiene que $e \star b' = b' \in H$ y se cumple iii). Finalmente, la operación
$\star$ es interna en $H$ pues si $a, b \in H$, acabamos de ver que $b' \in
H$ y por tanto $a \star (b')' = a \star b \in H$.
\fi

La ventaja de esta caracterización es que muchas veces se puede demostrar
que un $(H, \star)$ es un grupo simplemente mediante la demostración de que
es un subgrupo de un grupo conocido.

Ejercicio. Sea $\zset[\sqrt{2}] = \{a + b\sqrt{2} \st a, b \in \zset\}$.
Demuestre que $(\zset[\sqrt{2}], +)$ es un grupo siendo $+$ la suma habitual
de números reales restringida a $\zset[\sqrt{2}]$, y siendo $b\sqrt{2}$ el
producto usual en $\rset$.

Solución. Basta ver que $(\zset[\sqrt{2}], +)$ es un subgrupo de $(\rset,
+)$ gracias a la caracterización de grupo. En primer lugar,
$\zset[\sqrt{2}]$ no es vacío, ya que contiene al elemento 0, pues

$$ 0 = 0 + 0\sqrt{2} \in \zset[\sqrt{2}] $$

Dados dos elementos arbitrarios de $\zset[\sqrt{2}]$,

\begin{align*}
  z &= a + b\sqrt{2} \\
  y &= c + d\sqrt{2} \\
\end{align*}

\noindent siendo $a, b, c, d \in \zset$, debemos comprobar que $z + ({-y})
\in \zset[\sqrt{2}]$, siendo ${-y} \in \rset$ el elemento simétrico de $y$
en $(\rset, +)$, es decir,

$$ {-y} = {-(c + d\sqrt{2})} = {-c} - d\sqrt{2} $$

\noindent Entonces,

$$ z + ({-y}) = a + b\sqrt{2} + (-c - d\sqrt{2}) = (a - c) + (b - d)\sqrt{2}
$$

\noindent Es evidente que $a - c, b - d \in \zset$, con lo que $z + ({-y})
\in \zset[\sqrt{2}]$, cosa que demuestra que $(\zset[\sqrt{2}], +)$ es un
grupo. Como ve, no hemos tenido que hacer todas las comprobaciones de la
definición de grupo.

Ejercicio. Sea $n\zset = \{kn \st k \in \zset\}$, con $n \in \nset^*$.
Demuestre que es un grupo respecto de la suma de números reales restringida
a dicho conjunto.

Haciendo uso de la caracterización de subgrupo, vamos a comprobar si
$(n\zset, +)$ es un subgrupo de $(\zset, +)$. Lo primero será comprobar que
$n\zset$ no es vacío. Es evidente que $n \in n\zset$, con lo que se cumple
esta condición.

Ahora, tomemos dos elemntos cualesquiera de $n\zset$. Serán $a$ y $b$, que
son

\begin{align*}
  a &= kn \\
  b &= hn \\
\end{align*}

\noindent El simétrico de $b$ en $(\zset, +)$ es ${-b} = {-(hn)}$.
Entonces, se tiene que

$$ a + ({-b}) = kn + [{-(hn)}] = kn + ({-h})n = [k + ({-h})]n = (k - h)n $$

\noindent y es evidente que $(k - h)n \in n\zset$, con lo que $(n\zset, +)$
será un grupo, al haber quedado demostrado que es un subgrupo de $(\zset,
+)$.

Advierta que hemos usado, en el último paso, algunas propiedades de $(\zset,
+)$ que conocemos de nuestros estudios previos de matemáticas, aunque sea de
forma intuitiva. Por ejemplo, la propiedad distributive del producto
respecto a la suma o que ${-(hn)} = ({-h})n$. Estas se demostrarán en un
capítulo posterior, cuando se definan de forma rigurosa los números enteros.

Ejercicio. Lo mismo que el anterior pero para el conjunto $2\pi\zset =
\{2\pi k \st k \in \zset\}$.

Para demostrar que $(2\pi \zset, +)$ es un grupo, se hará de forma indirecta
simplemente demostrando que es un subgrupo de $(\rset, +)$ haciendo uso de
la caracterización de grupo.

Lo primero será comprobar que $2\pi \zset$ es no vacío. Por ejemplo, se
tiene que $2\pi \in 2\pi \zset$.

Ahora, tomemos dos elementos arbitrarios de $2\pi \zset$:

\begin{align*}
  a &= 2\pi k \\
  b &= 2\pi h \\
\end{align*}

\noindent El simétrico por $(\rset, +)$ de $b$ será ${-b} = {-(2\pi h)}$.
Ahora, vemos qué nos da $a + ({-b})$:

$$ a + ({-b}) = 2\pi k + [{-(2\pi h)}] = 2\pi k + 2\pi({-h}) = 2\pi(k - h)
$$

\noindent que es un elemento de $2\pi\zset$, con lo que queda demostrado que
$(2\pi \zset, +)$ subgrupo de $(\rset, +)$, y, evidentemente, es un grupo.
Como ve, se trata de un ejercicio prácticamente igual que el anterior.





\subsection{Congruencia módulo un subgrupo}

Sea $(G,\star)$ un grupo conmutativo y sea $(H, \star)$ un subgrupo de este.
La relación $\rrel_H$ en $G$ definida para todo $a, b \in G$ por

$$ a \rrel_H b \ \text{si y solo si} \ a \star b' \in H $$

\noindent es una relación de equivalencia que se denomina \emph{congruencia
módulo $H$}. Concretamente, esta es la congruencia módulo $H$ por la
derecha. Existiría también una por la izquierda, que no sería más que
cambiando lo último por $a' \star b$.

Comprobemos que efectivamente se trata de una relación de equivalencia. Es
reflexiva, pues para todo $a \in G$ se cumple $a \star a' = e$, y, por ser
$(H, \star)$ un grupo, se tiene que $e \in H$, con lo que se cumple que $a
\rrel_H a$.

Es simétrica. Veamos una demostración. Si $a \rrel_H b$, entonces $a \star
b' \in H$. Al tratarse de un grupo $(H, \star)$, el simétrico de $a \star
b'$ debe pertenecer también a $H$. Este sería $(a \star b')'$, y, por una de
las propiedades de grupo se tendrá que $(a \star b')' = (b')' \star a' \in
H$. Al tratarse de un grupo, se cumple la propiedad asociativa y por tanto
se tiene que $(b')' = b$, con lo que se tiene $b \star a' \in H$ y, por
tanto, $b \rrel_H a$.

% TODO Creo que se debería demostrar en algún punto que $(a')' = a$, en todo
% grupo.

Se cumple la propiedad transitiva. Si $a \rrel_H b$ y $b \rrel_H c$,
entonces $a \star b' \in H$ y $b \star c' \in H$. Como la operación $\star$
es interna en $H$, resulta que

\begin{align*}
  (a \star b') \star (b \star c')
    &= a \star (b' \star b) \star c' \\
    &= a \star e \star c' \\
    &= (a \star e) \star c' \\
    &= a \star c' \\
\end{align*}

\noindent ya que en todo grupo se cumple la propiedad asociativa de sus
elementos. Por tanto, $a \rrel_H c$.


% TODO Definición de clase lateral por la derecha y por la izquierda.

Estudiemos cómo son las clases de equivalencia de $\rrel_H$.


% ----------------------------------


Primero, demostraremos que esta clase de equivalencia induce una clase
lateral por la derecha.

Dirección 1: Si $a \mathcal{R}_H b$, entonces $a \in H \star b$.

Dado que $a \star b' \in H$, podemos escribir:

$$ \exists h \in H. \ a \star b' = h $$

Multiplicando por $b$ a la derecha,

\begin{align*}
  h &= a \star b' \\
  h \star b &= (a \star b') \star b = a \star (b' \star b) = a \star e = a \\
\end{align*}

\noindent Esto muestra que $a$ pertenece a la clase lateral derecha de $b$:

$$ a \in H \star b = \{ h \star b \st h \in H \} $$

Por lo tanto, la relación de equivalencia $a \mathcal{R}_H b$ implica que
$a$ pertenece a la clase lateral derecha $H \star b$.

Dirección 2: Si $a \in H \star b$, entonces $a \rrel_H b$.

Supongamos ahora que $a \in H \star b$, es decir, existe $h \in H$ tal que:

$$ a = h \star b $$

Multiplicando por $b'$ a la derecha, tenemos que

$$ a \star b' = h \in H $$

\noindent Esto implica que la relación de equivalencia $a \rrel_H b$ se
cumple.

Conclusión. Hemos probado que:

$$ a \rrel_H b \iff a \star b' \in H \iff a \in H \star b $$

Esto demuestra que la relación de congruencia definida por $a \star b' \in
H$ induce la clase lateral derecha:

$$ H \star b = \{ h \star b \st h \in H \} $$

Comentario sobre unicidad de clases laterales.

Dado que $H$ es un subgrupo, las clases laterales derechas $H \star b$
forman una partición del grupo $G$. Cada elemento pertenece exactamente a
una clase lateral derecha, lo que confirma que la relación de congruencia $a
\star b' \in H$ clasifica los elementos de $G$ en clases laterales derechas.

Conclusión final

La clase de equivalencia de un elemento $a$ bajo la relación $a \star b' \in
H$ está dada por:

$$ [a] = H \star a = \{ h \star a \st h \in H \} $$

Esto demuestra formalmente que la relación $a \rrel_H b \iff a \star b' \in
H$ genera clases laterales por la derecha.



% ------------------------------------------



\iffalse
Por un lado,
tenemos que

Sea $a \in G$,
se tiene que la clase de equivalencia del elemento $a$ por $\rrel_H$ es

$$ [a] = a \star H = \{a \star h \st h \in H\} $$

% TODO Esta notación a \star H ...

En efecto, si $b \in [a]$, entonces el elemento $h = b \star a'$ pertenecerá
a $H$. De esto se deduce que $b = h \star a = a \star h$. Recíprocamente, si
$b = a \star h$ con $h \in H$, entonces $b \star a' = h \in H$. 

% En el libro hay un error. La relación de equivalencia congruencia módulo
% un subgrupo definida así es por la derecha y las clases de equivalencia
% serán clases de equivalencia por la derecha, tal y como pongo aquí.
\fi

La expresión de las clases de equivalencia permite deducir las propiedades
siguientes:

Propiedad. Toda clase de equivalencia de la relación $\rrel_H$ es
equipotente a $H$.

En efecto, sea $a \in G$ y $[a]$ la clase de $a$. Sea la aplicación $\phi: H
\longrightarrow [a]$ definida por $\phi(h) = a \star h$ para todo $h \in H$.
De la expresión de $[a]$ se deduce que $\phi$ es sobreyectiva. La
inyectividad de $\phi$ resulta de la propiedad cancelativa que se satisface
en todo grupo.

Propiedades. Si $\card(G)$ es finito, entonces cualquier subgrupo $H$ cumple
que $\card(H)$ es un divisor de $\card(G)$.

Supongamos que $G$ tiene $n$ elementos y sea $k$ el número de elementos de
$H$. Por la propiedad anterior, todas las clases de equivalencia tienen $k$
elmentos. Denotaremos al conjunto cociente $G/\rrel_H$ por $G/H$. Como

$$ G = \bigcup_{[a] \in G/H} [a] \ \text{y si} \ [a], [b] \in G/H.
\quad [a] \neq [b] \implies [a] \cap [b] = \emptyset $$

\noindent resulta que $n = ck$, siendo $c$ el número de clases distintas. En
consecuencia, $k$ es un divisor de $n$.

En un grupo con un número finito de elmentos, a $\card(G)$ se le denomina
\emph{orden} del grupo $C$.

Ejemplo. Entre los conjuntos cocientes que hemos estudiado, ya nos hemos
encontrado algunos que pueden ser considerados como conjuntos cocientes
asociados a un subgrupo dado. En concreto, si tomamos $n\zset$ como subgrupo
de $\zset$, o $2\pi\zset$ como subgrupo de $\rset$, véase el ejercicio 4.16
TKTK, obtenemos precisamente los conjuntos cocientes de los ejemplos 3.10 y
3.11 TKTK, los enteros módulo $n$, $\zset/n\zset$ y los números reales
módulo $2\pi$, $\rset/2\pi\zset$.









