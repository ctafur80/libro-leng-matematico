




Consideramos ahora conjuntos donde están definidas dos operaciones internas.
Por analogía con las operaciones internas de números, así como por
comodidad, denotaremos a la primera operación como suma, $+$, mientras que a
la segunda la llamaremos producto, $\cdot$. El utilizar otros signos, por
ejemplo, $\oplus$ y $\odot$, para representar las operaciones sería quizás
más correcto pero muy engorroso y no lo haremos en general. Solo
utilizaremos otros símbolos en algún ejemplo donde las operaciones, ya
conocidas, tienen su propio símbolo.

Definición. Anillo. Sea $A$ un conjunto y sean $+$ y $\cdot$ dos operaciones
internas definidas en $A$. Diremos que la terna ordenada $(A, +, \cdot)$ es
un \emph{anillo} (\emph{ring}) si se satisfacen:

\begin{enumerate}
  \item $(A, +)$ es un grupo conmutativo.
  \item La operación $\cdot$ es asociativa.
  \item Se cumple la propiedad distributiva de $\cdot$ respecto de $+$, esto
    es,

    $$ a(b + c) = ab + ac \quad \text{y} \quad (b + c)a = ba + ca $$
\end{enumerate}

Si además la operación $\cdot$ es conmutativa, se dice que $(A, +, \cdot)$
es un \emph{anillo conmutativo}. Advierta que con la suma sí debe cumplirse
para ser anillo.

Si $(A, +, \cdot)$ es un anillo con elemento neutro para el producto siendo
este distinto del elemento neutro de la suma, se dice que $(A, +, \cdot)$ es
un \emph{anillo unitario}.

Muchos autores engloban bajo el término \emph{anillo} a lo que nosotros
hemos llamado \emph{anillo unitario}.

Seguiremos las mismas notaciones aditiva y multiplicativa que utilizamos en
los grupos. En concreto:

\begin{itemize}
  \item El elemento neutro de la suma se llama \emph{elemento nulo} y se
    designa por $0$.
  \item El simétrico de $a$ para $+$ se denomina \emph{elemento opuesto} y
    se designa por ${-a}$.
  \item El elemento neutro del producto, si existe, se denomina
    \emph{elemento unidad} y se designa por $1$. Además, se cumple que $1
    \neq 0$.
  \item El simétrico de $a$ para $\cdot$, si existe, se denomina
    \emph{elemento inverso} de $a$ y se designa por $a^{-1}$. Creo que en
    inglés lo llaman \emph{reciprocal element}. En este caso se dice que $a$
    es un elemento \emph{invertible} o \emph{inversible}.
\end{itemize}

Se sigue un convenio muy extendido que hay sobre la notación cuando se
tienen operaciones suma y producto que consiste en otorgar prioridad a la
operación producto respecto de la otra. Así, podemos escribir las
expresiones anteriores: $ab + ac$ y $ba + ca$, con el significado al que
estamos todos acostumbrados, es decir, como si se tratase de $(ab) + (ac)$ y
$(ba) + (ca)$. Evidentemente, aunque se explique ahora, este es el sentido
de la propiedad mencionada antes. El propósito de esta regla en la notación
es evitar tener expresiones muy sobrecargadas de símbolos.

Si $n \in \nset^*$, las notaciones $na$ y $a^n$ se escriben para indicar

$$ na = \underbrace{a + a + \cdots + a}_{\text{$n$ veces}} \quad \text{y}
\quad a^n = \underbrace{a \cdot a \cdot \cdots \cdot a}_{\text{$n$ veces}}
$$

Ejemplo. Casos de anillos conocidos. Veremos en capítulos posteriores que
los conjuntos $(\zset, +, \cdot)$, $(\qset, +, \cdot)$, $(\rset, +, \cdot)$
y $(\cset, +, \cdot)$ son anillos conmutativos unitarios respecto de la suma
y el producto habituales.

El conjunto de matrices cuadradas de orden $n$ respecto de la suma y del
producto de matrices es un anillo unitario no conmutativo.

Ejercicio. Demuestre que el conjunto $\powset(\Omega)$ es un anillo
conmutativo y unitario respecto de la diferencia simétrica $\triangle$ como
``suma'' y la intersección $\cap$ como ``producto''.

Solución. Vimos en el ejercicio 4.10 que $(\powset(\Omega), \triangle)$ era
un grupo conmutativo. Vimos en el capítulo 2 que la intersección es
asociativa, conmutativa y con elemento unidad, que era $\Omega$. Solo nos
quedaría por comprobar si se cumple la propiedad distributiva de $\cap$
respecto de $\triangle$. En efecto, para todo $A,B,C \in \powset(\Omega)$ se
tiene:

\begin{align*}
  A \cap (B \triangle C)
    &= A \cap [(B \cap \overline{C}) \cup (\overline{B} \cap C)] \\
    &= (A \cap B \cap \overline{C}) \cup (A \cap \overline{B} \cap C) \\
    &= (A \cap B \cap \overline{A}) \cup (A \cap B \cap \overline{C}) \cup
      (\overline{A} \cap A \cap C) \\
    &= ((A \cap B) \cap (\overline{A} \cap \overline{C})) \cup
      ((\overline{A} \cap \overline{B}) \cap (A \cap C)) \\
    &= ((A \cap B) \cap (\overline{A \cap C})) \cup ((\overline{A \cap B})
      \cap (A \cap C)) \\
    &= (A \cap B) \triangle (A \cap C) \\
\end{align*}

\noindent con lo que queda confirmado.

Proposición. Propiedades en un anillo. Sea $(A, +, \cdot)$ un anillo. Se
tiene:

\begin{enumerate}
  \item Para todo $a \in A$, $a \cdot 0 = 0 \cdot a = 0$. Esta propiedad se
    expresa también diciendo que $0$ es \emph{absorbente} para el producto.

  \item Para todo $a, b \in A$, $({-a})b = a({-b}) = {-(ab)}$ y
    $({-a})({-b}) = ab$.

  \item Si además el anillo $A$ es \emph{conmutativo}, se satisfacen las
    igualdades:

    \begin{align*}
      (a + b)^2 &= a^2 + b^2 + 2ab \\
      (a + b)(a - b) &= a^2 - b^2 \\
      (a + b)^n &= \binom{n}{0} a^n + \binom{n}{1} a^{n-1}b + \dots +
        \binom{n}{n-1} ab^{n-1} + \binom{n}{n} b^n \\
          &= \sum_{p=0}^{n} \binom{n}{p} a^{n-p} b^p \quad \text{para todo }
          n \in \nset^*
    \end{align*}
\end{enumerate}

Demostración.

1. Usando la propiedad distributiva del producto respecto de la suma, $a
\cdot 0 = a(0 + 0) = a \cdot 0 + a \cdot 0$ y por la propiedad cancelativa
de todo grupo, aplicándola a la suma, se deduce que $a \cdot 0 = 0$. La otra
igualdad se hace de manera análoga.

2. De $(ab) + [(-a)b] = (a + (-a))b = 0 \cdot b = 0$, se deduce que $ab$ y
$(-a)b$ son opuestos, es decir, $(-a)b = -(ab)$. Las otras igualdades son
análogas.

3. Las dos primeras igualdades se obtienen teniendo en cuenta que $ab = ba$.
Si no se tratase de un anillo conmutativo, no se cumplirían. La tercera de
esas igualdades suele recibir el nombre de binomio de Newton. Haremos una
demostración haciendo uso del Principio de Inducción.

Para $n = 1$ el resultado es trivial.

Supongamos que la fórmula es cierta para $n$, esto es,

$$ (a + b)^n = \binom{n}{0} a^n + \binom{n}{1} a^{n-1}b + \cdots +
\binom{n}{n-1} ab^{n-1} + \binom{n}{n} b^n $$

\noindent y demostremos que

$$ (a + b)^{n+1} = \binom{n+1}{0} a^{n+1} + \binom{n+1}{1} a^n b + \cdots +
\binom{n+1}{n} ab^n + \binom{n+1}{n+1} b^{n+1} $$

\indent En efecto,

\begin{align*}
  (a + b)^{n+1}
    &= (a + b)^n (a + b) \\
    &= \left[ \binom{n}{0} a^n + \binom{n}{1} a^{n-1}b + \cdots +
      \binom{n}{n-1} ab^{n-1} + \binom{n}{n} b^n \right](a + b) \\
    &= \binom{n}{0} a^{n+1} + \binom{n}{1} a^n b + \cdots +
      \binom{n}{n-1} ab^n + \binom{n}{n} b^{n+1} \\
    &= \binom{n}{0} a^{n+1} + \left[\binom{n}{1} + \binom{n}{0}\right]
      a^n b + \left[\binom{n}{2} + \binom{n}{1}\right] a^{n-1}b^2 +
      \cdots \\
    &+ \left[\binom{n}{n} + \binom{n}{n-1}\right] ab^n + \binom{n}{n}
    b^{n+1} \\
\end{align*}

Teniendo en cuenta que

$$ 1 = \binom{n}{0} = \binom{n+1}{0} $$

\noindent y

$$ \binom{n}{n} = \binom{n+1}{n+1} $$

\noindent y

$$ \binom{n+1}{p} = \binom{n}{p} + \binom{n}{p-1} $$

\noindent conocida esta última como la Fórmula de Pascal, se obtiene

$$ (a + b)^{n+1} = \binom{n+1}{0} a^{n+1} + \binom{n+1}{1} a^n b +
\binom{n+1}{2} a^{n-1}b^2 + \dots + \binom{n+1}{n} ab^n + \binom{n+1}{n+1}
b^{n+1} $$

Aunque se haga uso aquí de la Fórmula de Pascal, no se demostrará. Puede
consultar cualquier texto sobre combinatoria para encontrar una demostración
de esta.




\subsection{Divisores de cero}

En un anillo $(A, +, \cdot)$, se dice que el elemento $a \in A$ siendo $a
\neq 0$ es un \emph{divisor de cero} si existe $b \in A$ siendo $b \neq 0$
tal que $ab = 0$.

Ejemplo. En los anillos $\zset$, $\qset$ y $\rset$ no existen divisores de
cero.

Sí existen divisores de cero en el anillo $(\powset(\Omega), \triangle,
\cap)$, pues todo subconjunto $A$ de $\Omega$ tal que $A \neq \emptyset$ y
$A \neq \Omega$ es un divisor de cero ya que $A \cap \overline{A} =
\emptyset$.

También hay divisores de cero en el anillo de las matrices cuadradas de
orden 2. Por ejemplo, se tiene que

$$
  A =
  \begin{pmatrix}
    1 & 1 \\
    {-1} & -1
  \end{pmatrix}
$$

\noindent es un divisor de cero, pues tomando

$$
  B =
  \begin{pmatrix}
    {-1} & 1 \\
    1 & -1
  \end{pmatrix}
$$

\noindent se tiene que

$$
  AB =
  \begin{pmatrix}
    1 & 1 \\
    {-1} & {-1}
  \end{pmatrix}
  \begin{pmatrix}
    {-1} & 1 \\
    1 & {-1}
  \end{pmatrix}
  =
  \begin{pmatrix}
    0 & 0 \\
    0 & 0
  \end{pmatrix}
$$

Ejercicio. Sea $(A, +, \cdot)$ un anillo unitario. Demuestre que si $a \in
A$ es un divisor de cero entonces $a$ no es invertible. En consecuencia, un
elemento invertible no puede ser un divisor de cero.

Solución. Vamos a tratar de demostrarlo por contradicción. Supongamos que $a
\in A$ es un divisor de cero invertible. En consecuencia, existe el inverso
de $a$, $a^{-1} \in A$, y existe $b \in A$ siendo $b \neq 0$ tal que $ab =
0$. Multiplicando ambos miembros a la izquierda por $a^{-1}$ se obtiene,
según el punto de vista,

$$ a^{-1}(ab) = a^{-1}0 = 0 $$

\noindent por un lado, y

$$ a^{-1}(ab) = (a^{-1}a)b = 1 \cdot b = b $$

\noindent por el otro. Con lo que se tiene que $b = 0$, cosa que contradice
la hipótesis de partida según la cual $a$ era invertible.

Un anillo sin divisores de cero se denomina \emph{anillo íntegro}. Hemos
considerado únicamente divisores de cero a la izquierda. De manera análoga
se definen los divisores de cero a la derecha.

Obsérvese si $a$ es un divisor de cero (a la izquierda), entonces cualquier
elemento $b \neq 0$ tal que $ab = 0$ es un divisor de cero a la derecha y
viceversa. Por tanto, un anillo íntegro tampoco tiene divisores de cero a la
derecha.




\subsection{Subanillos e ideales}

Definición. Subanillo. Sea $(A, +, \cdot)$ un anillo y sea $H$ un
subconjunto no vacío de $A$, $H \subseteq A$. Se dice que $H$ es un
\emph{subanillo} (\emph{subring}) de $A$ si $(H, +, \cdot)$ es a su vez un
anillo.

En particular, el subconjunto unitario $H = \{0\}$ con las operaciones
restringidas de $A$ y el propio anillo $A$ son subanillos de $A$.

Igual que ocurría en los grupos, algunas propiedades de un anillo se
satisfacen automáticamente en un subconjunto de este con las mismas
operaciones, como la propiedad asociativa del producto o la propiedad
distributiva del producto respecto de la suma. Es muy fácil demostrar la
proposición siguiente, que caracteriza a los subanillos de un anillo dado.

Proposición. Caracterización de Subanillo. Sean $(A, +, \cdot)$ un anillo y
$H$ un subconjunto no vacío de $A$. $H$ es un subanillo de $A$ si y solo si
para todo $a, b \in H$ se cumplen:

\begin{itemize}
  \item $a - b \in H$
  \item $ab \in H$
\end{itemize}

Como puede ver, la primera de estas condiciones asegura que $(H, +)$ es un
subgrupo de $(A, +)$, ya que no es más que la condición de la
caracterización de subgrupo. Por otro lado, la segunda de las condiciones de
la caracterización de subanillo significa que el producto es una operación
interna en $H$.

% TODO Hacer realmente la demostración.

Como en los grupos, a veces es más rápido demostrar que $(H, +, \cdot)$ es
un anillo por medio de la demostración de que es un subanillo de un anillo
conocido. Así nos evitamos las propiedades asociativa y conmutativa de la
suma, la propiedad asociativa del producto y la propiedad distributiva del
producto respecto a la suma. Todas estas propiedades se ``heredan'' de que
$(A, +, \cdot)$ es un anillo.

Ejercicio. Demuestre que $\zset[\sqrt{2}] = \{a + b\sqrt{2} \st a, b \in
\zset\}$ con la suma y producto usuales de $\rset$ es un subanillo de
$(\rset, +, \cdot)$.

Solución. La forma de demostrarlo se hará mediante la caracterización de
subanillo. Como ya vimos en el ejercicio 4.15, $(\zset[\sqrt{2}], +)$ era un
subgrupo de $(\rset, +)$. En cuanto a la propiedad conmutativa de este, esta
se ``hereda'' de la del grupo $(\rset, +)$.

Respecto a la otra operación, el producto, solo tenemos que demostrar que
esta operación es interna en $\zset[\sqrt{2}]$. Dados $a, a', b, b' \in
\zset$, operando tenemos

$$ (a + b\sqrt{2})(a' + b'\sqrt{2}) = (aa' + 2bb') + (ab' + a'b)\sqrt{2} $$

\noindent con lo que se cumple la segunda de las condiciones de
caracterización de subanillo. Por todo esto, $(\zset[\sqrt{2}], +, \cdot)$
será un subanillo de $(\rset, +, \cdot)$. Además, $\zset[\sqrt{2}]$ es un
subanillo unitario pues tiene elemento nulo y elemento unidad, siendo estos
distintos. El elemento unidad, es decir, el elemento neutro del producto, es
$1 = 1 + 0\sqrt{2} \in \zset[\sqrt{2}]$.

Ejercicio. ¿Es $n\zset = \{kn \st k \in \zset\}$ un subanillo unitario de
$\zset$ para $n \in \nset$ siendo $n \geq 2$?

Solución. Ya demostramos en el ejercicio 4.16 que $n\zset$ era un subgrupo
de $(\zset, +)$. Además, claramente, si $a, b \in n\zset$, entonces $ab \in
n\zset$. Sin embargo, $1 \notin n\zset$. En este caso, $(n\zset, +, \cdot)$
es un subanillo no unitario del anillo $(\zset, +, \cdot)$.

De entre los subconjuntos de un anillo, los ideales juegan un papel muy
relevante, véase por ejemplo, el ejercicio 9. Tal y como se ve en las
asignaturas del álgebra abstracta, los ideales son la subestructura del
anillo con cualidades más interesantes; más que los propios subanillos.

Para simplificar la introducción de los ideales, ya que esta no es una
asignatura de álgebra abstracta, nos limitaremos al caso de anillos
conmutativos.

Definición. Ideal. Sean $(A, +, \cdot)$ un anillo conmutativo e $I$ un
subconjunto no vacío de $A$. Se dice que $(I, +, \cdot)$ es un \emph{ideal}
del anillo $(A, +, \cdot)$, si se cumplen:

\begin{itemize}
  \item $a - b \in I$ para todo $a, b \in I$.
  \item $ac \in I$ para todo $a \in I$ y para todo $c \in A$.
\end{itemize}

Con la primera de las condiciones, se tiene que $(I, +)$ es un subgrupo de
$(A, +)$. La otra condición es un caso más TKTK.

La diferencia con los subanillos está en la segunda condición de la
caracterización. El ideal es un concepto más amplio, tal y como se ve, por
lo que todo subanillo es un ideal.

Ejemplo. El conjunto

$$ n\zset = \{kn \st k \in \zset\} \ \text{para} \ n \in \nset $$

\noindent es un ideal de $\zset$ con sus operaciones usuales. En lo que
respecta a la suma, es fácil demostrarlo por la primera de las condiciones.
Para la condición del prodcuto, si $a \in n\zset$ y $c \in \zset$, existe $k
\in \zset$ tal que $a = kn$ y, en consecuencia, $ac = (kn)c = (kc)n \in
n\zset$. Para $n = 0$, se obtiene $I = \{0\}$, mientras que para $n = 1$ se
obtiene $I = \zset$.

$\{0\}$ y $A$ son siempre ideales del anillo $A$.

$\zset[\sqrt{2}]$ no es un ideal de $\rset$ pues tomando $a = 1 \in
\zset[\sqrt{2}]$ y $c \in \rset \setminus \zset[\sqrt{2}]$, resulta que $ac
= c \notin \zset[\sqrt{2}]$.

Definición. Si $(A, +, \cdot)$ es un anillo conmutativo y $a \in A$ es un
elemento fijo, el conjunto

$$ aA = \{ak \st k \in A\} $$

\noindent con esas mismas operaciones, es un ideal de $A$ que se denomina
\emph{ideal principal} generado por $a$. También se le denota por $(a)$.

Veremos en el capítulo~\ref{ch:enteros} que todos los ideales de $\zset$ son
principales.



























