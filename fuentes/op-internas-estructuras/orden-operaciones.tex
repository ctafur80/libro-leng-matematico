


Si en un conjunto tenemos definidas una relación de orden y una operación
interna, el hecho de que se cumplan ciertas propiedades de compatibilidad
entre ambas permite trabajar con ``desigualdades'', en el sentido amplio del
término, de manera similar a como se trabaja con las desigualdades a las que
estamos acostumbrados. Por simplificar, en este apartado supondremos que
todas las operaciones son conmutativas.





\subsection{Grupo ordenado}

Definición. Grupo Ordenado. Supongamos que tenemos un grupo conmutativo $(G,
+)$ siendo $0$ el elemento neutro y ${-a}$ el elemento simétrico de $a$,
para todo $a \in G$. Sea una relación de orden $\preceq$ definida sobre $G$.
Se dice que $(G, +, \preceq)$ es un \emph{grupo ordenado} (\emph{ordered
group}) si la relación de orden es compatible con la suma, esto es:

$$ \text{para todo} \ a, b, c \in G. \ a \preceq b \implies a + c \preceq b
+ c $$

\noindent Advierta que, como se ha requerido que se trate de un grupo
conmutativo, se podría haber puesto la condición $a \preceq b \implies c + a
\preceq c + b$.

Por analogía con los números, se dice que el elemento $a \in G$ es
\emph{positivo} si se cumple $0 \preceq a$ y el conjunto de los elementos
positivos de $G$ se denota por $G_+$. Se dice que el elemento $a \in G$ es
\emph{negativo} si $a \preceq 0$, denotándose al conjunto de los números
negativos como $G_{-}$.

Indistintamente se escribe $b \succeq a$ para indicar $a \preceq b$. y se
lee como $b$ ``sucede'', ``es posterior'' o ``es mayor o igual'' a $a$. La
notación $a \prec b$ o $b \succ a$ indica $a \preceq b$ y $a \neq b$. Tal y
como hemos dicho al comienzo de esta sección, interpretaremos estas
expresiones en el sentido amplio. TKTK.

Podemos deducir fácilmente una propiedad de los grupos ordenados.

Propiedad. En un grupo ordenado $(G, +, \preceq)$, dados dos elementos
cualesquiera $a, b \in G$ positivos, es decir, tales que $0 \preceq a$ y $0
\preceq b$, se cumple

$$ 0 \preceq a + b $$

Demostración.

\begin{align*}
  a &\succeq 0 \\
  a + b &\succeq 0 + b = b \\
\end{align*}

\noindent El primer paso se justifica por la condición de grupo ordenado. De
aquí, como tenemos las condiciones

\begin{align*}
  0 &\preceq b \\
  b &\preceq a + b \\
\end{align*}

\noindent y, como toda relación de orden cumple la propiedad transitiva, al
``encadenarlas'' obetenemos que

$$ 0 \preceq a + b $$

Veamos algunas propiedades más de los grupos ordenados.

Proposición. En un grupo ordenado $(G, +, \preceq)$ se satisfacen las
propiedades siguientes, para todo $a, a', b, b' \in G$:

\begin{enumerate}
  \item $a \preceq b$ si y solo si $b + ({-a}) \in G_+$.
  \item Si $a \preceq b$ y $c \preceq d$, entonces $a + c \preceq b + d$.
  \item Si $a \preceq b$, entonces ${-b} \preceq {-a}$.
\end{enumerate}

Demostración.

1.

\begin{align*}
  a &\preceq b \\
  a + ({-a}) &\preceq b + ({-a}) \\
  0 &\preceq b + ({-a}) \\
\end{align*}

\noindent Esto es lo mismo que decir $b + ({-a}) \in G_+$.

Recíprocamente, partiendo de $b + ({-a}) \in G_+$, transformamos esta
expresión en una equivalente: $0 \preceq b + ({-a})$, y, operando con esta,

\begin{align*}
  0 &\preceq b + ({-a}) \\
  0 + a &\preceq [b + ({-a})] + a = b + [({-a}) + a] = b + 0 = b \\
  a &\preceq b \\
\end{align*}

\noindent Se ha hecho uso de la propiedad asociativa, que debe cumplirse al
tratarse de un grupo.

2. Al tratarse de un grupo ordenado, al sumar $c \in G$ en las dos partes de
la expresión $a \preceq b$, se tiene, $a + c \preceq b + c$. De forma
similar, al sumar $b \in G$ en los dos lados de la expresión $c \preceq d$,
se obtiene $b + c \preceq b + d$. Haciendo uso de la propiedad transitiva de
la relación de orden, se tiene que $a + c \preceq b + d$.

3. Tal y como vimos antes, de $a \preceq b$ se obtiene que $0 \preceq b +
({-a})$. Entonces, operando,

\begin{align*}
  0 &\preceq b + ({-a}) \\
  ({-b}) + 0 &\preceq ({-b}) + [b + ({-a})] \\
  {-b} &\preceq [({-b}) + b] + ({-a}) = 0 + ({-a}) = {-a} \\
\end{align*}

Observación. La notación numérica de $b - a$ por $b + ({-a})$ se extiende a
todos los grupos con notación aditiva. De esta manera, el primer punto de la
proposición anterior se puede expresar como

$$ a \preceq b \ \text{si y solo si} \ b - a \in G_+ $$

Si la relación de orden es total, se dice que el grupo es un \emph{grupo
totalmente ordenado}, y, en caso contrario, \emph{grupo parcialmente
ordenado}.

Ejemplo.

En los capítulos siguientes veremos que $(\zset, +, \leq)$, $(\qset, +,
\leq)$ y $(\rset, +, \leq)$ son grupos totalmente ordenados. Por otra parte,
$(\qset^*, \cdot, \leq)$ no es un grupo ordenado pues el orden no es
compatible con el producto, ya que $1 \leq 2$ y sin embargo para $c = {-1}$
no se cumple que $1({-1}) \leq 2({-1})$. En cambio, sí es un grupo
totalmente ordenado el conjunto de los números racionales estrictamente
positivos, $(\qset_+^*, \cdot, \leq)$.

Consideramos en $\rset^2$ la suma definida componente a componente, es
decir, $(a,b) + (c,d) = (a + c, b + d)$ y el orden producto definido en el
ejemplo 3.23,

$$ (a, b) \leq_P (c, d) \ \text{si y solo si} \ a \leq c \ \text{y} \ b \leq
d $$

% TODO No me queda claro que esto esté explicado aquí.

\noindent entonces $(\rset^2, +, \leq_P)$ es un grupo parcialmente ordenado,
pues, si $(a, b) \leq_P (c, d)$ y $(e, f) \in \rset^2$, entonces $(a, b) +
(e, f) \leq_P (c, d) + (e, f)$, puesto que de $a \leq c$ y $b \leq d$ se
deduce que $a + e \leq c + e$ y $b + f \leq d + f$.

Sea $\mathcal{F}([0,1], \rset)$ el conjunto de funciones reales de variable
en $[0,1] \subseteq \rset$ donde, como es habitual, se define la suma de
funciones y el orden para todo $f, g \in \mathcal{F}([0,1], \rset)$,
mediante:

\begin{itemize}
  \item $(f + g)(x) = f(x) + g(x)$ para todo $x \in [0,1]$.
  \item $f \leq g$ si y solo si $f(x) \leq g(x)$ para todo $x \in [0,1]$.
\end{itemize}

\noindent Se comprueba fácilmente que $(\mathcal{F}([0,1], \rset), +, \leq)$
es un grupo parcialmente ordenado. Veámoslo.

Lo primero que deseamos comprobar es que la operación suma así definida es
una operación interna. Supongamos dos funciones arbitrarias $f, g \in
\mathcal{F}([0, 1], \rset)$. Tal y como se ve en la definición de esa suma
de funciones, se están usando dos sumas en realidad, aunque se representen
ambas por el mismo signo. Vamos a crear un símbolo nuevo para evitar una
posible confusión de estas operaciones. Vamos a designar por $\oplus$ a la
suma de funciones, dejando con $+$ a la de números reales. Se tendría
entonces lo siguiente:

$$ (f \oplus g)(x) = f(x) + g(x) \ \text{para todo} \ x \in [0,1] $$

\noindent en lugar de lo que se escribió antes.

En lo que respecta a la demostración de que $\oplus$ es una operación
interna, es muy fácil ver que $f(x) + g(x)$ lo es.

Ahora, veamos si se cumple la propiedad asociativa.

\begin{align*}
  [(f \oplus g) \oplus h](x)
    &= [(f \oplus g)(x)] + h(x) \\
    &= [f(x) + g(x)] + h(x) \\
    &= f(x) + [g(x) + h(x)] \\
    &= f(x) + [(g \oplus h)(x)] \\
    &= [f \oplus (g \oplus h)](x)
\end{align*}

En cuanto al elemento neutro, creemos que se trata de la función constante
de valor 0, es decir,

$$
  \begin{array}{rccc}
    I:  & [0, 1]  & \longrightarrow & \rset \\
        & x       & \longmapsto     & 0
  \end{array}
$$

\noindent Se comprueba fácilmente que con esta se tiene

$$ (f \oplus I)(x) = (I \oplus f)(x) = f(x) $$

\noindent pues, por ejemplo,

$$ (f \oplus I)(x) = f(x) + I(x) = f(x) + 0 = f(x) $$

Para el elemento simétrico, veamos cuál sería esa función $f' \in
\mathcal{F}([0, 1], \rset)$. Debería cumplirse lo siguiente

$$ (f \oplus f')(x) = (f' \oplus f)(x) = I(x) $$

Intuimos que puede tratarse de $f'(x) = {-f(x)}$. Comprobemos si es esta en
realidad. Por un lado, tenemos

$$ (f \oplus f')(x) = f(x) + ({-f(x)}) = f(x) - f(x) = I $$

\noindent con lo que se cumple. En el otro sentido de operación sería lo
mismo.

Ahora, habría que ver si se trata de un grupo ordenado mediante la relación
de orden dada.

Por claridad, al igual que hemos hecho con la suma, con estas dos relaciones
de orden usaremos dos notaciones distintas. La relación de las funciones la
designaremos por $\preceq$, mientras que la de los números reales será
$\leq$.

La relación de orden dada sería, con esta notación, la siguiente:

$$ f \preceq g \ \text{si y solo si} \ f(x) \leq g(x) \ \text{para todo} \ x
\in [0,1] $$

La condición para que se trate de un grupo ordenado sería:

$$ \text{para todo} \ f, g, h \in \mathcal{F}([0, 1], \rset). \ f \preceq g
\implies f \oplus h \preceq g \oplus h $$

La hipótesis de partida, $f \preceq g$, es lo mismo que decir que $f(x) \leq
g(x)$ para todo $x \in [0, 1]$, tal y como dice la definición de esa
relación de orden $\preceq$. Entonces, partiendo de esta y operando,

\begin{align*}
  f &\preceq g \\
  f(x) &\leq g(x) \\
  f(x) + h(x) &\leq g(x) + h(x) \\
  (f \oplus h)(x) &\preceq (g \oplus h)(x) \\
\end{align*}

\noindent siendo $h \in \mathcal{F}([0, 1], \rset)$. Queda confirmado
entonces que se trata de un grupo ordenado.

Quedaría por comprobar si se trata de un grupo totalmente ordenado o no. Es
fácil ver que el orden no es total, pues pueden existir dos funciones en
$\mathcal{F}([0, 1], \rset)$ que no estén relacionada en ninguno de los dos
sentidos posibles. TKTK.





\subsection{Anillo ordenado}

Supongamos ahora que la relación de orden está definida sobre un conjunto
$A$ donde tenemos definida una estructura de anillo conmutativo $(A, +,
\cdot)$. Ya vimos cómo en $\qset$ el orden $\leq$ no es en general
compatible con el producto de números racionales, aunque sí es compatible
cuando nos restringimos a números positivos. Con el propósito de incluir a
casos como este en el conjunto de anillos ordenados, la definición de estos
será algo más laxa para el producto que para la suma.

Definición. Anillo Ordenado. Se dice que $(A, +, \cdot, \leq)$ es un
\emph{anillo ordenado} si se cumplen:

\begin{itemize}
  \item Para todo $a,b,c \in A$, si $a \preceq b$, entonces $a + c \preceq b
    + c$.
  \item Para todo $a,b \in A$, si $0 \preceq a$ y $0 \preceq b$, entonces $0
    \preceq ab$.
\end{itemize}

Debido a esto, para todo anillo ordenado $(A, +, \cdot, \preceq)$ se tiene
que $(A, +, \preceq)$ es un grupo ordenado. En consecuencia, en un anillo
ordenado se satisfacen todas las propiedades de la proposición 4.35. De
nuevo se designa por $A_+$ al conjunto de elementos positivos de $A$, $A_+ =
\{a \in A \st 0 \preceq a\}$.

Si la relación de orden es total, se dice que el anillo es un \emph{anillo
totalmente ordenado}. Si además el anillo es un cuerpo, hablaremos de un
\emph{cuerpo ordenado}. En un anillo totalmente ordenado, se define el
\emph{valor absoluto} de $a \in A$ mediante

$$
  |a| =
  \begin{cases}
    a & \text{si} \ 0 \preceq a \\
    {-a} & \text{si} \ a \prec 0
  \end{cases}
$$

Proposición. En un anillo totalmente ordenado $(A, +, \cdot, \preceq)$ se
satisfacen las propiedades siguientes:

\begin{enumerate}
  \item $a \preceq b$ si y solo si $b - a \in A_+$.
  \item Si $a \preceq b$ y $a' \preceq b'$, entonces $a + a' \preceq b + b'$.
  \item Si $a \preceq b$, entonces ${-b} \preceq {-a}$.
  \item Si $a \preceq b$ y $0 \preceq c$, entonces $ac \preceq bc$.
  \item Si $a \preceq b$ y $c \preceq 0$, entonces $bc \preceq ac$.
  \item Para todo $a \in A$, $a^2 \succeq 0$.
  \item Si $A$ es un anillo unitario, entonces $0 \prec 1$.
  \item $|a| \succeq 0$ para todo $a \in A$ y $|a| = 0$ si y solo si $a =
    0$.
  \item $|ab| = |a||b|$ para todo $a, b \in A$.
  \item $|a + b| \preceq |a| + |b|$ para todo $a, b \in A$. A esta propiedad
    se la conoce como la \emph{desigualdad triangular}.
\end{enumerate}

Si además $(A, +, \cdot)$ es un \emph{cuerpo}, también se cumplen:

\begin{enumerate}
  \setcounter{enumi}{10}
  \item Si $a \succ 0$, entonces $a^{-1} \succ 0$.
  \item Si $0 \prec a \preceq b$, entonces $b^{-1} \preceq a^{-1}$.
  \item Si $a \preceq b \prec 0$, entonces $b^{-1} \preceq a^{-1}$.
\end{enumerate}

% TODO Ir mejorando esto, cuando tenga tiempo.

Demostración. Las propiedades 1, 2 y 3 se deducen de la proposición 4.35. La
propiedad 8 se deduce sin ninguna dificultad.

4. Si $a \preceq b$ y $0 \preceq c$ entonces $0 \preceq b - a$ y $0 \preceq
c$. En consecuencia $0 \preceq (b-a)c = bc - ac$ y por tanto $ac \preceq
bc$.

5. Si $a \preceq b$ y $c \preceq 0$ entonces $0 \preceq b - a$ y $0 \preceq
{-c}$. En consecuencia $0 \preceq (b-a)({-c}) = {-bc} + ac$ y por tanto $bc
\preceq ac$.

6. Si $0 \preceq a$ de la propiedad ii) de la definición de anillo ordenado
se deduce que $0 \preceq a \cdot a = a^2$. Si $a \preceq 0$, multiplicando
ambos miembros por $a$ y aplicando la propiedad 5 se deduce que $0 \preceq a
\cdot a$.

7. Basta tener en cuenta que $1 = 1 \cdot 1 = 1^2$ y por tanto $1 \geq 0$.
Teniendo en cuenta que $1 \neq 0$ se obtiene que $0 < 1$.

8. Se comprueba sin dificultad en los cuatro casos posibles: i) $0 \preceq
a$ y $0 \preceq b$. ii) $a < 0$ y $0 \preceq b$. iii) $0 \preceq a$ y $b <
0$. iv) $a < 0$ y $b < 0$.

10. Observemos, en primer lugar, que para todo $a \in A$ se cumple
trivialmente que $a \preceq |a|$ y $-a \preceq |a|$.

\begin{itemize}
  \item Si $0 \preceq a + b$, entonces $|a + b| = a + b \preceq |a| + |b|$.
  \item En caso contrario, $a + b < 0$ y en consecuencia, $|a + b| = {-a} -
    b \preceq |a| + |b|$.
\end{itemize}

11. Supongamos $a > 0$. Si fuera $a^{-1} \preceq 0$, multiplicando por $a$
ambos términos se deduce que $1 = aa^{-1} \preceq a \cdot 0 = 0$, que
contradice la propiedad 7.

12 y 13. En ambos casos se obtiene que $ab > 0$. Por la propiedad anterior
se deduce que $(ab)^{-1} = b^{-1}a^{-1} = a^{-1}b^{-1} > 0$. Entonces, si $a
\preceq b$, multiplicando ambos miembros por $a^{-1}b^{-1}$, se obtiene
$aa^{-1}b^{-1} \preceq ba^{-1}b^{-1}$, esto es, $b^{-1} \preceq a^{-1}$.

En los capítulos 5 y 6, veremos que el anillo $(\zset, +, \cdot, \preceq)$ y
los cuerpos $(\qset, +, \cdot, \preceq)$ y $(\rset, +, \cdot, \preceq)$ son
ordenados. También veremos en el capítulo 7, como la propiedad 6 de la
proposición anterior, nos permite afirmar que no existe en el conjunto de
los números complejos ninguna relación de orden total compatible con la
estructura de cuerpo.







