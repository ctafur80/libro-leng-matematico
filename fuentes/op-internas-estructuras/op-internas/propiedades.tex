





\subsection{Propiedad asociativa}

Propiedad. Dado un conjunto $E$ y una operación interna $\star$ sobre este.
La operación $\star$ es \emph{asociativa} si para todo $a, b, c \in E$ se
cumple

$$ (a \star b) \star c = a \star (b \star c) $$

Una de las ventajas de la propiedad asociativa es que nos podemos permitir
eliminar los paréntesis de expresiones como las anteriores. Para estas,
tendríamos que son equivalentes a $a \star b \star c$, que también sería una
fórmula bien definida, es decir, tiene asociado un único significado, sin
ambigüedad.

Otra ventaja es que permite definir por recurrencia cómo se operan $n + 1$
elementos, ya que se cumple que

$$ a_1 \star a_2 \star \cdots \star a_n \star a_{n+1} = (a_1 \star a_2 \star
\cdots \star a_n) \star a_{n+1} $$

Ejemplo. De las operaciones del ejemplo de la sección anterior TKTK, hemos
visto en capítulos anteriores que son asociativas las leyes $\land$ y $\lor$
en el conjunto de proposiciones lógicas, $\cap$ y $\cup$ en
$\powset(\Omega)$ y la composición de aplicaciones en $\mathcal{F}(\Omega)$.
Veremos en los capítulos~\ref{ch:naturales} y~\ref{ch:enteros} que las
operaciones $+$ y $\cdot$ son asociativas en $\nset$, $\zset$, $\qset$ y
$\rset$. También son asociativas las operaciones $\land$ y $\lor$, máximo
común divisor y mínimo común múltiplo, en $\nset^*$; o suma, $+$, y
producto, $\times$, en el conjunto de matrices cuadradas de orden $n$. Sin
embargo, no son asociativa la resta ni la división de números, pues $(9 - 5)
- 1 \neq 9 - (5 - 1)$ y $(16/4)/2 \neq 16/(4/2)$.




\subsection{Propiedad conmutativa}

Propiedad. Dado un conjunto $E$ y una operación interna $\star$ sobre este.
La operación $\star$ es \emph{conmutativa} (\emph{commutative}) si para todo
$a, b \in E$ se cumple

$$ a \star b = b \star a $$

Si una operación $\star$ es asociativa y conmutativa, entonces $a_1 \star
a_2 \star \cdots \star a_n$ permanece invariable cuando se permutan o se
reagrupan de manera arbitraria los elementos. Tiene sentido, por tanto, la
notación

$$ \star_{i=1}^n a_i $$

\noindent para indicar

$$ a_1 \star a_2 \star \cdots \star a_n $$

\noindent siendo indiferente el orden de los mismos.

En particular, cuando se utiliza la notación aditiva o la multiplicativa,
cosa que veremos luego, para operaciones que sean asociativas y
conmutativas, se usan

$$ \sum_{i=1}^n a_i \quad \text{y} \quad \prod_{i=1}^n a_i $$

\noindent respectivamente, para la suma y el producto de $a_1, a_2, \ldots,
a_n$. En el caso de que todos los $a_i$ sean iguales a $a$, la suma se
indica por $na$, y, el producto, por $a^n$.

También, se utilizan para la intersección y unión de conjuntos las
notaciones siguientes:

$$ \bigcap_{i=1}^n A_i \quad \text{y} \quad \bigcup_{i=1}^n A_i $$

\noindent para indicar, respectivamente, la intersección y la unión de los
conjuntos $A_1, A_2, \ldots, A_n$.

Ejemplo. De las operaciones del ejemplo 4.1 TKTK, no son conmutativas la
composición de aplicaciones o el producto de matrices. Esto conlleva que
cuando se manejen igualdades, por ejemplo, de matrices, hay que proceder con
cautela a la hora de multiplicar los dos miembros de la igualdad,
multiplicando ambos miembros a la izquierda, o ambos a la derecha. Es decir,
de $A = B$ se deduce que $AC = BC$ o $CA = CB$ pero no se deduce que $AC =
CB$. Tampoco son conmutativas la resta ni la división en los conjuntos
numéricos $\zset$, $\rset$, etc.




\subsection{Elemento neutro}

Propiedad. Dado un conjunto $E$ y una operación interna $\star$ sobre este.
Se denomina \emph{elemento neutro} de la operación interna $\star$ en $E$ a
un elemento $e \in E$ que cumple, para todo $a \in E$,

$$ a \star e = e \star a = a $$

En lo que respecta a la terminología, en inglés, en lugar de llamarlo
\emph{neutral element} se le suele llamar \emph{identity element}, que en
español podríamos traducir como \emph{elemento identidad}. Si consulta
textos en inglés, conviene que conozca esto.

Ejemplo. Los elementos neutros de $+$ y $\cdot$ en $\nset$, $\zset$, $\qset$
y $\rset$ son, respectivamente, $0$ y $1$.

Los de $\cap$ y $\cup$ en $\powset(\Omega)$ son respectivamente $\Omega$ y
$\emptyset$, cosa que es fácil de demostrar. Con la intersección de los
elementos de $\powset(\Omega)$, $\cap$, se ha de obtener un elemento $B \in
\powset(\Omega)$ tal que se cumpla, para todo $A \in \powset(\Omega)$, $A
\cap B = A$. Es fácil ver que ese elemento $B$ es el conjunto $\Omega$. De
forma análoga, para $A \cup B = A$ podemos ver que el único elemento $B \in
\powset(\Omega)$ que lo comple es $\emptyset$.

El elemento neutro de la composición en $\mathcal{F}(\Omega)$ es la
aplicación identidad $I_\Omega$. Dada una función $f \in
\mathcal{F}(\Omega)$ y la función identidad $I_\Omega$,

\begin{center}
\begin{minipage}[t]{.45\textwidth}
  \centering
  $$
    \begin{array}{llll}
      f:  & \Omega  & \longrightarrow & \Omega \\
          & x       & \longmapsto     & f(x)
    \end{array}
  $$
\end{minipage}
\begin{minipage}[t]{.45\textwidth}
  \centering
  $$
    \begin{array}{llll}
      I_\Omega: & \Omega  & \longrightarrow & \Omega \\
                & x       & \longmapsto     & I_\Omega (x) = x
    \end{array}
  $$
\end{minipage}
\end{center}

\noindent se tiene que

\begin{align*}
  I_\Omega \circ f (x) &= I_\Omega (f(x)) = f(x) \\
  f \circ I_\Omega (x) &= f(I_\Omega (x)) = f(x) \\
\end{align*}

\noindent con lo que $I_\Omega$ será el elemento neutro de la composición de
funciones en $\mathcal{F}(\Omega)$.

El elemento neutro en el producto de matrices cuadradas de orden 2 es la
matriz siguiente

$$
  I =
  \begin{pmatrix}
    1 & 0 \\
    0 & 1
  \end{pmatrix}
$$

\noindent a la que se suele conocer como \emph{matriz identidad} (de orden
2).

No existe elemento neutro en la resta o división en $\rset^*$, $-$ y $/$, ni
en el máximo común divisor en $\nset^*$, $\land$.

Proposición. Unicidad del elemento neutro. En caso de que exista el elemento
neutro de una operación sobre un conjunto, dicho elemento será único.

Demostración. Lo único que hay que hacer para demostrarlo es cambiar el
punto de vista. Supongamos que $e$ y $e'$ son ambos elementos neutros del
conjunto $E$ con la operación interna $\star$. Se tiene entonces,
considerando que $e'$ es elemento neutro, que $e \star e' = e$. Por otro
lado, considerando que $e$ es elemento neutro, se tiene que $e \star e' =
e'$. De ambas conclusiones se deduce que $e = e'$.





\subsection{Elemento simétrico}

Definición. Elemento simétrico. Dado un conjunto $E$ con una operación
interna $\star$ sobre $E$ con elemento neutro $e \in E$. Se denomina
\emph{elemento simétrico} del elemento $a \in E$ a un elemento $a' \in E$
tal que

$$ a \star a' = a' \star a = e $$

Observación. De la propia definición del elemento simétrico se deduce que si
$a'$ es elemento simétrico de $a$, entonces $a$ es elemento simétrico de
$a'$. Es decir, que $(a')' = a$. TKTK. En realidad, creo que ha de darse la
propiedad asociativa, tal y como indica la propiedad que vemos más adelante.
De lo contrario, se podría tener más de uno.

% TODO Comprobar esto último. Por lo que tengo entendido, estructuras
% algebraicas no asociativas, como los loops o los pseudogrupos, no cumplen
% esta propiedad de $(a')' = a$.

Advierta que el elemento neutro es un elemento para todo el conjunto y la
operación, mientras que el simétrico será uno para cada elemento del
conjunto, bajo esa operación. Es decir,

\begin{align*}
  \exists e \in E \ \text{tal que} \ \forall a \in E, \, a \star e = e \star
    a = a \\
  \forall a \in E \, \exists a' \in E \ \text{tal que} \ a \star a' = a'
    \star a = e \\
\end{align*}

En la terminología, en inglés, no se suele ver que se hable del elemento
simétrico. A este lo suelen denominar \emph{inverse element}, que en español
a veces traducen como \emph{elemento inverso}. El problema está en que esto
último puede causar confusión, pues muchas veces se llama así a un inverso
en concreto: el de la operación producto en alguno de los conjuntos
numéricos $\nset$, $\zset$, etc. TKTK. Cuando se habla de forma concreta del
inverso por la división, en inglés lo llaman normalmente \emph{reciprocal
element}.

Advierta que, según la definición, está permitido que el elemento simétrico
de un elemento sea él mismo.

Ejemplo. En $\nset$, ningún elemento tiene simétrico respecto de la suma,
salvo $a = 0$, o el producto, salvo $a = 1$. En ambos casos el simétrico del
elemento sería él mismo.

En $\zset$, $\qset$ y $\rset$, el simétrico de $a$ para la suma es ${-a}$.
En $\zset$, no existe el simétrico de $a$ para el producto salvo si $a =
{-1}$ o $a = 1$. En $\qset^*$ y $\rset^*$, el simétrico de $a$ para el
producto es $1/a$.

Según vimos en el capítulo anterior, en el conjunto $\mathcal{F}(\Omega)$ de
las aplicaciones de un conjunto dado en sí mismo las aplicaciones biyectivas
son las únicas que tienen simétrico respecto de la composición. El simétrico
de la biyección $f$ es la biyección inversa $f^{-1}$.

En el conjunto de matrices cuadradas de orden $n$, solo tienen simétrico
respecto del producto las matrices cuyo determinante es distinto de $0$.

Proposición. Unicidad del Elemento Simétrico. Sea $\star$ una operación
interna asociativa en $E$ con elemento neutro $e \in E$. Si $a \in E$ tiene
elemento simétrico, este es único.

Demostración. Supongamos que $a'$ y $a''$ son ambos elementos simétricos de
$a$. Utilizamos la propiedad asociativa para calcular de dos maneras
distintas $a' \star a \star a''$:

\begin{align*}
  a' \star a \star a''  &= (a' \star a) \star a'' = e \star a'' = a'' \\
  a' \star a \star a''  &= a' \star (a \star a'') = a' \star e = a' \\
\end{align*}

\noindent En consecuencia, $a' = a''$.

Ahora, pasamos a ver las estructuras algebraicas.




