


El estudio de las estructuras algebraicas surgió cuando algunos matemáticos
pensaron que era una buena idea dejar de estudiar el comportamiento de las
distintas clases de objetos matemáticos para pasar poner el foco en su
comportamiento. Se fueron definiendo entonces algunas estructuras
algebraicas y esto permitió ``aglutinar'', o sintetizar, en cierto modo, el
conocimiento.

Así, por ejemplo, se pueden deducir algunas de las propiedades que conocemos
para un tipo concreto de objetos, como que se cumple $a, b \in \rset$, $a^2
- b^2 = (a + b)(a - b)$ para todo $a, b \in \rset$, y deducir esa propiedad
de la estructura algebraica anillo conmutativo, que definiremos más adelante
en este capítulo. Esto permite conocer que esa propiedad se da también en
otros objetos matemáticos distintos a los números reales.

Tal y como ya hemos mencionado anteriormente, el propósito principal de esta
asignatura es definir de forma rigurosa los distintos tipos de números que
ya conocemos: naturales, enteros, racionales, etc. Aunque ya conozca todos
estos y sepa manipular las expresiones en las que aparecen, seguramente los
conoce únicamente de una forma intuitiva. Una definición rigurosa y
satisfactoria de estos requiere del conocimiento de ciertas estructuras
algebraicas. Para presentar las estructuras algebraicas, se debe conocer el
concepto de \emph{operación}, que es el propósito de la próxima sección.




