



Para comenzar a hablar de estructuras algebraicas, debemos definir primero
las operaciones, que son un tipo de aplicaciones. Según las propiedades que
cumpla una operación sobre un conjunto, se tendrá un tipo de estructura
algebraica.

Definición. Operación interna. Dado un conjunto $E$, una \emph{operación
interna} $R$ en $E$ es una aplicación de $E \times E$ en $E$, es decir, $R
\in \mathcal{F}(E \times E, E)$.

Se trata, por tanto, de una aplicación que asocia un elemento de $E$ a todo
par ordenado $(a, b)$ de elementos de $E$.

Hay quien prefiere llamar \emph{ley de composición interna} a las
operaciones internas.

% TODO Operación binaria interna??

% TODO Operaciones externas.

En lugar de usar una notación de tipo función, como $R(a, b)$, lo usual es
usar la notación mediante un operador infijo, es decir, escribir los dos
argumentos entre un símbolo que represente a la operación, como $a \star b$,
$a \circ b$, $a + b$, etc. Esto es lo más común, aunque algunas operaciones
internas se representan en simbología de función.

Por cierto, si se fija, en uno de los ejemplos hemos usado el signo de la
suma, $+$, que evidentemente, ya conoce desde hace muchos años. La suma en
los números reales, así como en los otros conjuntos en los que se suele
definir, es una operación interna.

A este respecto, tal y como dijimos antes, vamos a aumentar el grado de
abstracción y puede que hablemos de la suma, `$+$', con un sentido diferente
al que suele tener, incluyendo aquí a operaciones distintas. Por ejemplo, la
operación suma existe tanto para números reales como para matrices
cuadradas. Tal y como veremos aquí, se trata en realidad de dos operaciones
distintas, aunque las representemos por el mismo signo, $+$, y las llamemos
igual. La prueba está en que algunas de sus propiedades son distintas. Así,
aunque para la suma de números reales se cumple lo que se conoce como la
propiedad cancelativa, que viene a decir que, para $a, b \in \rset$, si $ax
= ay$ y $a \neq 0$, entonces $x = y$, esta propiedad no se da para la suma
de matrices cuadradas. Bastaría el siguiente contraejemplo para demostrarlo:

$$
  \begin{pmatrix}
  1 & 2 \\
  3 & 6
  \end{pmatrix}
  \begin{pmatrix}
  5 & 3 \\
  -1 & 1
  \end{pmatrix}
  =
  \begin{pmatrix}
  1 & 2 \\
  3 & 6
  \end{pmatrix}
  \begin{pmatrix}
  3 & 1 \\
  0 & 2
  \end{pmatrix}
$$

\noindent siendo, evidentemente,

$$
  \begin{pmatrix}
  5 & 3 \\
  -1 & 1
  \end{pmatrix}
  \neq
  \begin{pmatrix}
  3 & 1 \\
  0 & 2
  \end{pmatrix}
$$

Se podría decir que el álgebra abstracta es un área muy básica de las
matemáticas. Básica no en el sentido de fácil, sino en que en esta nos
movemos cerca de la base matemática. Una prueba de ello es que estamos
constantemente hablando de conjuntos.

Veamos algunos ejemplos más de operaciones internas que ya conocemos. La
intersección y la unión en el conjunto $\powset(\Omega)$ de las partes de un
conjunto $\Omega$, que representamos por los símbolos `$\cap$' y `$\cup$',
respectivamente, serían operaciones internas, ya que, para todo $A, B \in
\powset(\Omega)$, se tiene que $A \cap B \in \powset(\Omega)$ y $A \cup B
\in \powset(\Omega)$.

También, la composición en el conjunto $\mathcal{F}(\Omega)$ de las
aplicaciones de un conjunto $\Omega$ en sí mismo; símbolo `$\circ$'.

La suma, la resta y el producto en los conjuntos $\nset$, $\zset$, $\qset$ o
$\rset$; `$+$', `$-$' y `$\cdot$', aunque el producto también se representa
a veces como `$\times$' o por la notación en aposición.

La división en los conjuntos $\qset^*$ o $\rset^*$; `$/$'.

La suma y el producto en el conjunto de matrices cuadradas de orden $n$;
`$+$' y `$\times$'.

La conjunción y la disyunción en el conjunto de proposiciones lógicas;
`$\land$' y `$\lor$'.

El máximo común divisor y el mínimo común múltiplo en $\nset^*$; `$\gcd()$'
y TKTK. También hay quien usa notación infija, como `$\land$' y `$\lor$',
para estas operaciones TKTK.

El producto vectorial en el espacio euclídeo tridimensional; `$\times$' o
`$\wedge$'.

La resta en $\nset$, `$-$', o el producto escalar en el espacio euclídeo
tridimensional, `$\cdot$', no son operaciones internas, sino externas. No
vamos a necesitar de las operaciones externas en esta asignatura.

% TODO Definir operación externa.




