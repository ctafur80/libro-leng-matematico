


Definición. Cuerpo. Un \emph{cuerpo} (\emph{field}) es un anillo conmutativo
unitario en el que todo elemento no nulo es invertible respecto del
producto.

Al añadir la propiedad del elemento inverso ---eso sí, solo para los
elementos no nulos---, se cumplirá también la del elemento neutro. Por
tanto, se podría dar también la definición siguiente. Sea $\mathbb{K}$ un
conjunto y sean $+$ y $\cdot$ dos operaciones internas definidas en
$\mathbb{K}$. La terna ordenada $(\mathbb{K}, +, \cdot)$ es un cuerpo si se
satisfacen las condiciones siguientes:

\begin{enumerate}
  \item $(\mathbb{K}, +)$ es un grupo conmutativo.
  \item $(\mathbb{K}^*, \cdot)$ es un grupo conmutativo.
  \item La operación $\cdot$ es distributiva respecto de la operación $+$ en
    $\mathbb{K}$.
\end{enumerate}

\noindent o, si desea la lista de todas las propiedades:

\begin{enumerate}
  \item Las operaciones $+$ y $\cdot$ son asociativas en $\mathbb{K}$.
  \item Las operaciones $+$ y $\cdot$ son conmutativas en $\mathbb{K}$.
  \item La operación $\cdot$ es distributiva respecto de la operación $+$ en
    $\mathbb{K}$.
  \item Existen los elementos neutros de la suma y el producto en
    $\mathbb{K}$, designados por 0 y 1, respectivamente.
  \item Para todo elemento $a$ de $\mathbb{K}$, existe el simétrico de $a$
    respecto de la suma; que se designa por ${-a}$ y se le llama
    \emph{opuesto} de $a$.
  \item Para todo elemento $a \neq 0$ de $\mathbb{K}$, existe el simétrico
    de $a$ para el producto; que se designa por $a^{-1}$ y se le llama el
    \emph{inverso} de $a$.
\end{enumerate}

En cualquier caos, algunas de estas no sería necesario demostrarlas. Por
ejemplo, al cumplirse la propiedad conmutativa del producto, bastaría con
demostrar uno de los dos casos de la propiedad distributiva del producto
respecto de la suma.

No existe un consenso ampliamente extendido sobre la definición de
\emph{cuerpo}. Así, en esta no siempre se incluye la conmutatividad del
producto. En esa terminología, cuando el producto es conmutativo lo indican
denominándolo \emph{cuerpo conmutativo}. Nosotros entenderemos que en un
cuerpo el producto es siempre conmutativo. Seguimos en este sentido la
terminología inglesa que denomina \emph{field} a lo que hemos denominado
\emph{cuerpo}, mientras que, si el producto no es conmutativo, se denomina
\emph{anillo de división} (\emph{division ring}).

Como ve, el conjunto de un cuerpo se suele representar en una tipografía
algo extraña; como con barras dobles. TKTK. Se trata de la misma que se usa
para los conjuntos numéricos típicos, como los enteros, $\zset$, los reales,
$\rset$, etc., sean o no cuerpos, que llevamos viendo a lo largo de toda la
asignatura. TKTK.

Definición. Subcuerpo. Si $(\mathbb{K}, +, \cdot)$ es un cuerpo y $H$ es un
subconjunto de $\mathbb{K}$, consideramos las restricciones a $H$ de las
operaciones en $\mathbb{K}$. Se dice que $(H, +, \cdot)$ es un
\emph{subcuerpo} de $(\mathbb{K}, +, \cdot)$ si es a su vez un cuerpo.

Ejemplo. En los capítulos siguientes, veremos que $(\qset, +, \cdot)$,
$(\rset, +, \cdot)$ y $(\cset, +, \cdot)$ son cuerpos. Sin embargo, $(\zset,
+, \cdot)$ o $(\zset[\sqrt{2}], +, \cdot)$ no son cuerpos, pues en estos no
todos los elementos no nulos son invertibles; por ejemplo, $x = 2$ no es
invertible para las operaciones típicas $+$ y $\cdot$, tanto en $\zset$ como
en $\zset[\sqrt{2}]$.

Ejemplo. Consideramos el conjunto cociente de los enteros módulo 3,
$\zset/3\zset = \{0, 1, 2\}$, de los ejemplos 3.10 y 4.17, y definimos las
operaciones $+$ y $\cdot$ tomando representantes en cada clase de
equivalencia; esto es:

$$ [a] + [b] = [a + b] \quad \text{y} \quad [a] \cdot [b] = [a \cdot b] $$

Se comprueba que las operaciones no dependen de los representantes escogidos
(véase el ejercicio 9 TKTK) y se obtienen las tablas siguientes:

$$
  \begin{array}{c|ccc}
  + & 0 & 1 & 2 \\
  \hline
  0 & 0 & 1 & 2 \\
  1 & 1 & 2 & 0 \\
  2 & 2 & 0 & 1
  \end{array}
  \qquad
  \begin{array}{c|ccc}
  \cdot & 0 & 1 & 2 \\
  \hline
  0 & 0 & 0 & 0 \\
  1 & 0 & 1 & 2 \\
  2 & 0 & 2 & 1
  \end{array}
$$

Es fácil comprobar que $(\zset/3\zset, +, \cdot)$ es un cuerpo. Veámoslo.

Las propiedades asociativa y conmutativa de ambas operaciones se ``heredan''
de las de la suma y del producto en $\zset$. Si desea demostrar de otra
forma la conmutativa, simplemente basta con que compruebe que las tablas son
simétricas respecto de la diagonal principal.

En cuanto a la propiedad distributiva del producto respecto a la suma,
aunque serían dos, para todo $a, b, c \in \zset/3\zset$,

\begin{align*}
  a(b + c) &= ab + ac \\
  (b + c)a &= ba + ca \\
\end{align*}

\noindent en realidad basta con demostrar una de las dos, tal y como
dijimos, ya que se cumple la propiedad conmutativa. Habría que ir probando
con valores. Por ejemplo, si $a = 0$, es fácil ver que se cumple. En
realidad, habría que comprobarlo para todos los casos posibles en la tabla.

Elemento neutro de la suma. Es el valor 0 ya que no altera al valor con el
que se le opere. En general, debe comprobarlo tanto en filas como en
columnas, pero, como se cumple la propiedad conmutativa para esta operación,
bastaría con hacerlo para las filas solamente, por ejemplo. Con el del
producto sería similar. Se trata del elemento 1.

Elemento simétrico de la suma. Serían $1' = 2$, $2' = 1$ y $0' = 0$, tanto
en filas y en columnas. El del producto también se daría, excepto para el 0,
pero en este no es neceario que se dé ya que es el elemento neutro de la
suma. Concretamente, se tendría que $1' = 1$ y $2' = 2$.

Todo esto hace que $(\zset/3\zset, +, \cdot)$ cumpla todas las condiciones
para ser un cuerpo.

Ejemplo. Consideremos ahora el conjunto cociente de los enteros módulo 4,
$\zset/4\zset = \{0,1,2,3\}$,  y definiendo de nuevo las operaciones $+$ y
$\cdot$ mediante

$$ [a] + [b] = [a + b] \quad \text{y} \quad [a] \cdot [b] = [a \cdot b] $$

\noindent se obtiene:

$$
  \begin{array}{c|cccc}
  + & 0 & 1 & 2 & 3 \\
  \hline
  0 & 0 & 1 & 2 & 3 \\
  1 & 1 & 2 & 3 & 0 \\
  2 & 2 & 3 & 0 & 1 \\
  3 & 3 & 0 & 1 & 2
  \end{array}
  \qquad
  \begin{array}{c|cccc}
  \cdot & 0 & 1 & 2 & 3 \\
  \hline
  0 & 0 & 0 & 0 & 0 \\
  1 & 0 & 1 & 2 & 3 \\
  2 & 0 & 2 & 0 & 2 \\
  3 & 0 & 3 & 2 & 1
  \end{array}
$$

En este caso, $(\zset/4\zset, +, \cdot)$ es un anillo conmutativo unitario
pero no es un cuerpo  pues 2 no es invertible para el producto. Basta
recorrer la fila o columna del 2 para observar que no existe ningún elemento
$a$ tal que $2 \cdot a = 1$.

Del ejercicio 4.23 se deduce que un cuerpo no puede tener divisores de cero.
Por tanto  si $\mathbb{K}$ es un cuerpo, entonces el producto es una
operación interna en $\mathbb{K}^* = \mathbb{K} \setminus \{0\}$.

Propiedades. Como todo cuerpo $(\mathbb{K}, +, \cdot)$ es un anillo
conmutativo, se satisfacen en particular todas las propiedades, válidas para
anillos, de la proposición 4.21. Asimismo, $(\mathbb{K}^*, \cdot)$ satisface
todas las propiedades válidas para grupos de la proposición 4.12. En
particular, en un cuerpo $(\mathbb{K}, +, \cdot)$ se cumple:

\begin{itemize}
  \item $a \cdot 0 = 0 \cdot a = 0$ para todo $a \in \mathbb{K}$.
  \item Si $a \cdot b = 0$ entonces $a = 0$ o $b = 0$. Es decir, no hay
    divisores de $0$.
  \item Si $ab = ac$ y $a \neq 0$ entonces $b = c$. Propiedad cancelativa en
    $(\mathbb{K}^*, \cdot)$.
  \item Si $a \neq 0$ y $b \in \mathbb{K}$, la ecuación $ax + b = 0$ tiene
    solución única en $\mathbb{K}$, siendo esta $x = -ba^{-1}$.
\end{itemize}

Al igual que para los anillos se introdujo el concepto de subanillo, el
concepto de subcuerpo es análogo.

Definición. Subcuerpo. Sea $(\mathbb{K}, +, \cdot)$ un cuerpo y sea $H$ un
subconjunto no vacío de $\mathbb{K}$ donde consideramos las restricciones de
las operaciones de $\mathbb{K}$. Si $(H, +, \cdot)$ cumple las condiciones
de cuerpo, será entonces un \emph{subcuerpo} (\emph{subfield}) de
$(\mathbb{K}, +, \cdot)$.

Teniendo en cuenta las proposiciones 4.14 y 4.33, resulta inmediata la
proposición siguiente.

Proposición. Caracterización de Subcuerpo. Sean $(\mathbb{K}, +, \cdot)$ un
cuerpo y $H$ un subconjunto de $\mathbb{K}$ con al menos dos elementos
distintos. $H$ es un subcuerpo de $\mathbb{K}$ si y solo si se cumplen:

\begin{itemize}
  \item $a - b \in H$ para todo $a, b \in H$.
  \item $ab^{-1} \in H$ para todo $a, b \in H^*$.
\end{itemize}

























